%%%%%%%
%%%%%%% Conformal twistor KIDs
%%%%%%% 
%%%%%%%
%%%%%%% 
%%%%%%%
%%%%%%% Started on: 26.05.2019
%%%%%%% 
%%%%%%%

\documentclass[10pt,a4paper]{article}
\usepackage{amssymb}
\usepackage{amsmath}
\usepackage{amsfonts}
\usepackage{amsthm}
\usepackage{latexsym}
\usepackage{mathrsfs}
\usepackage{stmaryrd}
\usepackage[dvips]{epsfig}
\usepackage{setspace}
\usepackage{float}
\usepackage{bm}
%\usepackage{showkeys}
%\usepackage[pdftex]{graphicx}
\usepackage{tikz}
\usepackage{enumerate}
\usepackage{subfigure}
\usepackage{nomencl}
%\usepackage{makeidx} 
%\makeindex
\usepackage{authblk}
\renewcommand\Affilfont{\itshape\small}
\usepackage{textcomp}
\usepackage{scrextend}% add KOMA-Script features to other classes
\usepackage[toc,page]{appendix}

%\usepackage[utf8]{inputenc}
%\usepackage[LGR,T1]{fontenc}
%\newcommand\Koppa{\begingroup\fontencoding{LGR}\selectfont\char21\endgroup}

\theoremstyle{plain}
\newtheorem{proposition}{Proposition}
\newtheorem{lemma}{Lemma}
\newtheorem{theorem}{Theorem}
\newtheorem{assumption}{Assumption}
\newtheorem*{conjecture}{Conjecture}
\newtheorem*{subconjecture}{Subconjecture}
\newtheorem{corollary}{Corollary}
\newtheorem*{main}{Theorem}
\newtheorem{definition}{Definition}
\newtheorem{remark}{Remark}

\setlength{\textwidth}{148mm}           % Width of text on page- max 148
\setlength{\textheight}{235mm}          % height of text on page-max 235
\setlength{\topmargin}{-10mm}            % Margin at top ofpage- max -5
\setlength{\oddsidemargin}{0mm}         % Odd page sidemargin max 15
\setlength{\evensidemargin}{0mm}

% Underlined lowcase latin letters
\def\es{{\bar{s}}}
\def\er{{\bar{r}}}

% Boldface mathmode lowcase latin letters
\def\bma{{\bm a}}
\def\bmb{{\bm b}}
\def\bmc{{\bm c}}
\def\bmd{{\bm d}}
\def\bme{{\bm e}}
\def\bmf{{\bm f}}
\def\bmg{{\bm g}}
\def\bmh{{\bm h}}
\def\bmi{{\bm i}}
\def\bmj{{\bm j}}
\def\bmk{{\bm k}}
\def\bml{{\bm l}}
\def\bmn{{\bm n}}
\def\bmm{{\bm m}}
\def\bmo{{\bm o}}
\def\bmq{{\bm q}}
\def\bms{{\bm s}}
\def\bmt{{\bm t}}
\def\bmu{{\bm u}}
\def\bmv{{\bm v}}
\def\bmw{{\bm w}}
\def\bmx{{\bm x}}
\def\bmy{{\bm y}}
\def\bmz{{\bm z}}

% Boldface mathmode numbers
\def\bmzero{{\bm 0}}
\def\bmone{{\bm 1}}
\def\bmtwo{{\bm 2}}
\def\bmthree{{\bm 3}}

% Boldface mathmode uppercase latin letters
\def\bmA{{\bm A}}
\def\bmB{{\bm B}}
\def\bmC{{\bm C}}
\def\bmD{{\bm D}}
\def\bmE{{\bm E}}
\def\bmF{{\bm F}}
\def\bmG{{\bm G}}
\def\bmH{{\bm H}}
\def\bmK{{\bm K}}
\def\bmL{{\bm L}}
\def\bmM{{\bm M}}
\def\bmN{{\bm N}}
\def\bmP{{\bm P}}
\def\bmQ{{\bm Q}}
\def\bmR{{\bm R}}
\def\bmS{{\bm S}}
\def\bmT{{\bm T}}
\def\bmX{{\bm X}}
\def\bmZ{{\bm Z}}


\def\Riem{{\bm R}{\bm i}{\bm e}{\bm m}}
\def\Ric{{\bm R}{\bm i}{\bm c}}
\def\Weyl{{\bm W}{\bm e}{\bm y}{\bm l}}
\def\RWeyl{{\bm R}{\bm W}{\bm e}{\bm y}{\bm l}}
\def\Sch{{\bm S}{\bmc}{\bm h}}
\def\Schouten{{\bm S}{\bmc}{\bm h}{\bm o}{\bm u}{\bm t}{\bm e}{\bm n}}
\def\Hessian{{\bm H}{\bm e}{\bm s}{\bm s}}

% Fracture letters
\def\fraka{{\frak a}}
\def\frakb{{\frak b}}
\def\frakc{{\frak c}}
\def\frakd{{\frak d}}
\def\frakf{{\frak f}}
\def\frakg{{\frak g}}
\def\fraki{{\frak i}}
\def\frakj{{\frak j}}
\def\frakk{{\frak k}}

% Mathbf letters
\def\mbfu{\mathbf{u}}

% Boldface mathmode lowcase latin letters
\def\bma{{\bm a}}
\def\bmb{{\bm b}}
\def\bmc{{\bm c}}
\def\bmd{{\bm d}}
\def\bme{{\bm e}}
\def\bmf{{\bm f}}
\def\bmg{{\bm g}}
\def\bmh{{\bm h}}
\def\bmi{{\bm i}}
\def\bmj{{\bm j}}
\def\bmk{{\bm k}}
\def\bml{{\bm l}}
\def\bmn{{\bm n}}
\def\bmm{{\bm m}}
\def\bmo{{\bm o}}
\def\bmq{{\bm q}}
\def\bmr{{\bm r}}
\def\bms{{\bm s}}
\def\bmt{{\bm t}}
\def\bmu{{\bm u}}
\def\bmv{{\bm v}}
\def\bmw{{\bm w}}
\def\bmx{{\bm x}}
\def\bmy{{\bm y}}
\def\bmz{{\bm z}}
\def\bmJ{{\bm J}}

% Boldface mathmode numbers
\def\bmzero{{\bm 0}}
\def\bmone{{\bm 1}}
\def\bmtwo{{\bm 2}}
\def\bmthree{{\bm 3}}

% Boldface mathmode uppercase latin letters
\def\bmA{{\bm A}}
\def\bmB{{\bm B}}
\def\bmC{{\bm C}}
\def\bmD{{\bm D}}
\def\bmE{{\bm E}}
\def\bmF{{\bm F}}
\def\bmG{{\bm G}}
\def\bmH{{\bm H}}
\def\bmK{{\bm K}}
\def\bmL{{\bm L}}
\def\bmM{{\bm M}}
\def\bmN{{\bm N}}
\def\bmP{{\bm P}}
\def\bmQ{{\bm Q}}
\def\bmR{{\bm R}}
\def\bmS{{\bm S}}
\def\bmT{{\bm T}}
\def\bmU{{\bm U}}
\def\bmV{{\bm V}}
\def\bmW{{\bm W}}
\def\bmX{{\bm X}}
\def\bmY{{\bm Y}}
\def\bmZ{{\bm Z}}


% Boldface mathmode lowcase greek letters
\def\bmalpha{{\bm \alpha}}
\def\bmbeta{{\bm \beta}}
\def\bmgamma{{\bm \gamma}}
\def\bmdelta{{\bm \delta}}
\def\bmepsilon{{\bm \epsilon}}
\def\bmeta{{\bm \eta}}
\def\bmzeta{{\bm\zeta}}
\def\bmxi{{\bm \xi}}
\def\bmchi{{\bm \chi}}
\def\bmiota{{\bm \iota}}
\def\bmomega{{\bm \omega}}
\def\bmlambda{{\bm \lambda}}
\def\bmmu{{\bm \mu}}
\def\bmnu{{\bm \nu}}
\def\bmphi{{\bm \phi}}
\def\bmvarphi{{\bm \varphi}}
\def\bmsigma{{\bm \sigma}}
\def\bmvarsigma{{\bm \varsigma}}
\def\bmtau{{\bm \tau}}
\def\bmupsilon{{\bm \upsilon}}

% Boldface mathmode uppercase greek letters
\def\bmGamma{{\bm \Gamma}}
\def\bmPhi{{\bm \Phi}}
\def\bmUpsilon{{\bm \Upsilon}}
\def\bmSigma{{\bm \Sigma}}

% Boldface operators
\def\bmpartial{{\bm \partial}}
\def\bmnabla{{\bm \nabla}}
\def\bmhbar{{\bm \hbar}}
\def\bmperp{{\bm \perp}}
\def\bmell{{\bm \ell}}

% Complex and real numbers
\font\SYM=msbm10
\newcommand{\Real}{\mbox{\SYM R}}
\newcommand{\Complex}{\mbox{\SYM C}}
\newcommand{\Natural}{\mbox{\SYM N}}
\newcommand{\Integer}{\mbox{\SYM Z}}
\newcommand{\Sphere}{\mbox{\SYM S}}


% ParallelPerp symbol
\newcommand{\parperp}{\mathbin{\text{\rotatebox[origin=c]{90}{$\models$}}}}
\newcommand{\perppar}{\mathbin{\text{\rotatebox[origin=c]{-90}{$\models$}}}}

% Projector symbols
%\newcommand{\proj2perp}{\pi^{\small{\perp}}}
%\newcommand{\proj2par}{\pi^{\small{\parallel}}}
%\newcommand{\proj4par}{\pi^{\small{\parallel}}}
%\newcommand{\proj4parperp}{\pi^{\small{\parperp}}}
%\newcommand{\proj4perppar}{\pi^{\small{\perppar}}}
%\newcommand{\proj4perp}{\pi^{\small{\perp}}}

%Counter variable for the margin notes
\newcounter{mnotecount}%[section]

\usepackage{cancel}

% This code generates the margin notes
\newcommand{\mnotex}[1]%{}
{\protect{\stepcounter{mnotecount}}$^{\mbox{\footnotesize $\bullet$\themnotecount}}$ 
\marginpar{%\color{red}%
\raggedright\tiny\em
$\!\!\!\!\!\!\,\bullet$\themnotecount: #1} }

\renewcommand\labelitemi{\tiny$\bullet$}

\begin{document}

\title{\textbf{Conformal twistor KIDs}}

\author[,1]{E. Gasper\'{i}n \footnote{E-mail address:{\tt
      edgar.gasperin@tecnico.ulisboa.pt}}}
\author[,2]{J.L. Williams \footnote{E-mail address:{\tt jlw31@bath.ac.uk}}}
%\author[1]{Con T. Ributor}
%\affil[1]{Autodesk Research?}
\affil[1]{Instituto Superior T\'{e}cnico }
\affil[2]{Department of Mathematical Sciences, University of Bath, Claverton Down, Bath BA2 7AY, United Kingdom.}

\maketitle
\begin{abstract}
    Write-up of Edgar's twistor KID calculations from the new notebook. 
\end{abstract}

\section{Definitions}

We are interested here in the twistor equation:
\[\nabla_{A'(A}\kappa_{B)}=0, \]
encoded in the vanishing of the \textit{zero quantity} $H_{A'AB}:=2\nabla_{A'(A}\kappa_{B)}$. It will prove useful to define the auxiliary quantity
\[\xi_{A'}:= \nabla^A{}_{A'}\kappa_A. \]
We also define the zero quantity $S_{A'B'A}:=\nabla_{QA'}H_{B'A}{}^Q.$ This may be expressed alternatively in terms of the auxiliary field\mnotex{I think we should change the sign in either the definition of $S$ or $\xi$, to be more consistent. } $\xi_{A'}$, by an easy computation, as follows:
\begin{equation}
S_{A'B'A} = -\nabla_{AB'}\xi_{A'} + 2\Lambda \epsilon_{A'B'}\kappa_A - 2\Phi_{AQA'B'}\kappa^Q. \label{Eq:SZeroQuantityInTermsOfXi}
\end{equation}
The other (symmetrised) contraction yields the \textit{Buchdahl constraint}:
\[0 = \nabla_{(A}{}^{A'}H_{\vert A'\vert BC)} = \Psi_{ABCD}\kappa^D, \]
though this won't feature in the calculations here. 

\section{Wave equations}

\subsection{For the twistor fields}

A short computation shows that 
\[\square \kappa_B = -2\Lambda \kappa_B + \tfrac{2}{3}\nabla^{AA'}H_{A'AB}. \]
Hence, if $\kappa_B$ solves the twistor equation, then it necessarily satisfies the wave equation 
\begin{equation}
    \square \kappa_B + 2\Lambda \kappa_B = 0. \label{Eq:WaveEqForKappa}
\end{equation}
We will choose to propagate a twistor candidate according to this equation. It is maybe worth noticing that $S_{A'}{}^{A'}{}_B=\nabla^{AA'}H_{A'AB}$. Hence, if $\kappa_A$ satisfies \eqref{Eq:WaveEqForKappa}, then necessarily $S_{A'B'A} = S_{(A'B')A}$, since it is assured that $\nabla^{AA'}H_{A'AB}=0$. 
\\

We will also need a wave equation for the auxiliary field $\xi_{A'}$. To get this, take the contracted derivative of \eqref{Eq:WaveEqForKappa} and commute derivatives, finally resulting in 
\begin{equation}
    \square \xi_{A'} + 2\Lambda \xi_{A'} - 8(\nabla_{AA'}\Lambda)\kappa^A = 0. \label{Eq:WaveEqForXi}
\end{equation}
Again, if $\kappa_A$ is a twistor, then the resulting $\xi_{A'}$ solves \eqref{Eq:WaveEqForXi}. Given a twistor candidate and its corresponding $\xi_{A'}$, we choose to propagate the latter according to this wave equation. 

\subsection{For the zero quantities}

In order to derive wave equations for the zero quantities, we assume that twistor candidate, $\kappa_A$ and its auxiliary spinor $\xi_{A'}$ satisfy the wave equations \eqref{Eq:WaveEqForKappa}--\eqref{Eq:WaveEqForXi}; at the end of the day, these will be satisfied by construction, since the candidate quantities will be propagated off the initial hypersurface using these wave equations.  
\\

To get a wave equation for $H_{A'AB}$, simply take the definition of $\bmS$ in terms of $\bmH$ and take a contracted derivative ---i.e. consider $\nabla_{A}{}^{D'}S_{D'A'B}$. Ultimately, we get
\[\square H_{A'AB} = 8\Lambda H_{A'AB} - 2\Psi_{ABCD}H_{A'}{}^{CD} - 2\Phi_{ADA'D'}H^{D'D}{}_B - 2\nabla_{AD'}S^{D'}{}_{A'B}.\]
It is important to note that this is expressible (in a regular way) in terms of the rescaled Weyl spinor $\phi_{ABCD} = \Theta^{-1}\Psi_{ABCD}$:
\[\square H_{A'AB} = 8\Lambda H_{A'AB} - 2\Theta\phi_{ABCD}H_{A'}{}^{CD} - 2\Phi_{ADA'D'}H^{D'D}{}_B - 2\nabla_{AD'}S^{D'}{}_{A'B}.\]
To close the system, we need a wave equation for $S_{A'B'A}$. To get this, we will apply the D'Alembertian to \eqref{Eq:SZeroQuantityInTermsOfXi}, commute derivatives, and substitute the wave equation for $\xi_{A'}$, equation \eqref{Eq:WaveEqForXi}. Finally, we arrive at\mnotex{Is it possible to simplify this a bit more?}
\begin{multline}\square S_{A'B'A} = 6\Lambda S_{A'B'A} - 4\Phi_{ABC'(A'}S_{B')}{}^{C'B} - 2\Theta \bar{\phi}_{A'B'C'D'}S^{C'D'}{}_A \\- \tfrac{2}{3}\Phi_{BCA'B'}(\nabla_{AC'}H^{C'BC}+2\nabla^C{}_{C'}H^{C'}{}_A{}^B) \\+ 4H_{(A'\vert AB\vert}\nabla^B{}_{B')}\Lambda - 2(\nabla_{CC'}\Phi_{ABA'B'})H^{C'BC}.
\end{multline}
Note that the terms on the right-hand-side are homogeneous in $\bmS,~\bmH$ and $\nabla\bmH$.

\section{What goes wrong in the higher-valence case?}

Define also the ``Buchdahl zero quantity":
\[ B_{ABCD} = \phi_{F(ABC}\kappa_{D)}}{}^F.\]
Note that 
\[\nabla_{(A}{}^{A'}H_{\vertA'\vert BCD)} = 6\Theta B_{ABCD} \]
Can derive equations of the form
\begin{eqnarray}
    && \square H_{A'ABC} = (\bmH, \nabla\bmS),\\
    && \square H_{A'ABC} = (\bmH, \bmB, \nabla\bmB),\\
    && \square S_{AA'BB'} = (\bmH, \bmS, \Theta\bmB, \nabla\Theta\cdot\nabla\bmB) = (\bmH, \bmS, \nabla\bmH, \nabla\Theta\cdot\nabla\bmB),\\
    && \square B_{ABCD} = (\bmH, \bmB) + \tfrac{2}{3}\nabla_{\bm\xi}\phi_{ABCD} = (\bmH, \bmB) + \tfrac{2}{3}\mathcal{L}_{\bm\xi}\phi_{ABCD}.\label{Eq:TempEqForBuchdahl}
\end{eqnarray}
The final equality follows from the fact that $\nabla_{(A}{}^{A'}\xi_{B)A'}=0$, as a consequence of the assumed wave equation for $\kappa_{AB}$.\mnotex{Generally speaking, do we need to worry about the fact that $\kappa_{AB},\xi_{AA'}$ are being propagated independently (though consistently)?}
\\

\textbf{Note}: In the equation for $\bmS$, we couldn't replace the $\nabla \bmB$ with $\nabla\nabla\bmH$ terms even if we wanted to, because then we would have $\Theta^{-1}$ factors appearing. 
\\

\textbf{Proposal:} Define $F_{A'BCD}:=\nabla^A{}_{A'}B_{ABCD}$. Then we get a wave equation for $B$ trivially:
\[\square B_{ABCD} \propto \nabla_{(A}{}^{A'}F_{\vert A'\vert BCD)} + \text{curv.}\times \bmB. \]
To get the remaining equation for $F_{A'BCD}$ take a contracted derivative of \eqref{Eq:TempEqForBuchdahl}, commute derivatives on the $\nabla_{\bm\xi}\bm\phi$ term and use the fact that $\nabla^A{}_{A'}\phi_{ABCD}=0.$

\section{A closed system for the Killing spinor case}

Zero quantities:
\[H_{A'ABC}, \qquad B_{ABCD}, \qquad F_{A'BCD}:=\nabla^A{}_{A'}B_{ABCD} \]
From the previous section, $\nabla_{(A}{}^{A'}H_{\vertA'\vert BCD)} = 6\Theta B_{ABCD}$. Additionaly, the wave equation for $\kappa_{AB}$ is equivalent to $\nabla^{AA'}H_{A'ABC}=0$, hence we have $\nabla_{A}{}^{A'}H_{A'BCD} = 6\Theta B_{ABCD}$. Contracting with $\nabla^A{}_{B'}$ we then derive the following wave equation:
\begin{equation}
    \square H_{A'ABC} = ...
\end{equation}
Similarly, substituting the definition of $F_{A'ABC}$ in terms of $B_{ABCD}$, it is straightforward to verify the following wave equation for $B_{ABCD}$: 
\begin{equation}
     \square B_{ABCD} = 12\Lambda B_{ABCD} - 6\Theta \phi_{(AB}{}^{FG}B_{CD)FG} + 2\nabla_{AA'}F^{A'}{}_{BCD}. \label{Eq:FirstWaveEqForB}
\end{equation}
The task remaining is to derive a wave equation for $F_{A'ABC}$. Let us first consider some useful identities:
\begin{align}
     \kappa^{DG}\phi_{(ABC}{}^H\phi_{F)HDG}&=\kappa_A{}^D\phi_{(BC}{}^{GH}\phi_{FD)GH}+\kappa_B{}^D\phi_{(AC}{}^{GH}\phi_{FD)GH}\nonumber\\
    &+\kappa_C{}^D\phi_{(AB}{}^{GH}\phi_{FD)GH}+\kappa_F{}^D\phi_{(BC}{}^{GH}\phi_{AD)GH} \nonumber\\
    & =2\phi_{(AB}{}^{GH}B_{CF)GH}
\end{align}
Using this identity, we can derive the following alternative (non-homogeneous) wave equation for $B_{ABCD}$:
\begin{equation}
    \square B_{ABCD} = \tfrac{2}{3}\nabla_{\bm\xi}\phi_{ABCD} + 8\Lambda B_{ABCD} - 14\Theta \phi_{(AB}{}^{FG}B_{CD)FG} + \tfrac{2}{3}(\nabla_{FA'}\phi_{G(ABC})H^{A'}{}_{D)}{}^{FG}\label{Eq:SecondWaveEqForB}
\end{equation}
where, for convenience, we have defined $\xi_{AA'}:=\nabla^B{}_{A'}\kappa_{AB}$, as usual. Contrary to previous approaches, however, we won't propagate $\xi_{AA'}$ independently of $\kappa_{AB}$; it is simply a convenient shorthand. 
\begin{remark}{\em}
As an aside, note that \eqref{Eq:SecondWaveEqForB} expresses the fact that, \textbf{given a Killing spinor} $\kappa_{AB}$, the quantity 
\[\mathcal{L}_{\bm\xi}\phi_{ABCD}\equiv \nabla_{\bm\xi}\phi_{ABCD} + \phi_{F(ABC}\nabla_{D)A'}\xi^{FA'} = \nabla_{\bm\xi}\phi_{ABCD} \]
---the equality being by virtue of the wave equation for $\kappa_{AB}$--- vanishes, and in particular that in the physical spacetime the Lie derivative of the Weyl curvature along the Killing vector $\tilde{\bm\xi}$ vanishes, as we know should be the case.}
\end{remark}

At this point we note that there are no derivatives of zero quantities appearing on the right-hand-side of \eqref{Eq:SecondWaveEqForB}. This, combined with the fact that $\nabla^A{}_{A'}\phi_{ABCD}=0$, makes it seems plausible that by applying $\nabla^A{}_{A'}$ to \eqref{Eq:SecondWaveEqForB} we will be able to derive a wave equation for $F_{A'BCD}$. We will see that this does indeed work; the only essential difficulty is in showing that $\nabla^A{}_{A'}\nabla_{\bm\xi}\phi_{ABCD}$ is expressible in terms of the zero quantities $\bmH, \bmB, \bmF$ and their first derivatives only. This is done in the following identity: 
\begin{equation}
    \nabla^A{}_{A'}\nabla_{\bm\xi}\phi_{ABCD} = 
\end{equation}

\end{document}
