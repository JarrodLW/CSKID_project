%%%%%%%
%%%%%%% Matter Aligment condition
%%%%%%% 
%%%%%%% arXiv version.
%%%%%%%
%%%%%%%
%%%%%%% Started on: 11.9.2016
%%%%%%% Current version: 18.Nov.2021
%%%%%%%

\documentclass[10pt,a4paper]{article}
\usepackage{amssymb}
\usepackage{amsmath}
\usepackage{amsfonts}
\usepackage{amsthm}
\usepackage{latexsym}
\usepackage{mathrsfs}
\usepackage{stmaryrd}
\usepackage[dvips]{epsfig}
\usepackage{setspace}
\usepackage{float}
\usepackage{bm}
%\usepackage{showkeys}
%\usepackage[pdftex]{graphicx}
\usepackage{tikz}
\usepackage{enumerate}
\usepackage{subfigure}
\usepackage{nomencl}
%\usepackage{makeidx} 
%\makeindex
\usepackage{authblk}
\renewcommand\Affilfont{\itshape\small}
\usepackage{textcomp}
\usepackage{scrextend}% add KOMA-Script features to other classes
\usepackage[toc,page]{appendix}
\usepackage{comment}
\usepackage{xcolor}


\theoremstyle{plain}
\newtheorem{proposition}{Proposition}
\newtheorem{lemma}{Lemma}
\newtheorem{theorem}{Theorem}
\newtheorem{assumption}{Assumption}
\newtheorem*{conjecture}{Conjecture}
\newtheorem*{subconjecture}{Subconjecture}
\newtheorem{corollary}{Corollary}
\newtheorem*{main}{Theorem}
\newtheorem*{definition}{Definition}
\newtheorem{remark}{Remark}

\setlength{\textwidth}{148mm}           % Width of text on page- max 148
\setlength{\textheight}{235mm}          % height of text on page-max 235
\setlength{\topmargin}{-10mm}            % Margin at top ofpage- max -5
\setlength{\oddsidemargin}{0mm}         % Odd page sidemargin max 15
\setlength{\evensidemargin}{0mm}

% Underlined lowcase latin letters
\def\es{{\bar{s}}}
\def\er{{\bar{r}}}

% Boldface mathmode lowcase latin letters
\def\bma{{\bm a}}
\def\bmb{{\bm b}}
\def\bmc{{\bm c}}
\def\bmd{{\bm d}}
\def\bme{{\bm e}}
\def\bmf{{\bm f}}
\def\bmg{{\bm g}}
\def\bmh{{\bm h}}
\def\bmi{{\bm i}}
\def\bmj{{\bm j}}
\def\bmk{{\bm k}}
\def\bml{{\bm l}}
\def\bmn{{\bm n}}
\def\bmm{{\bm m}}
\def\bmo{{\bm o}}
\def\bmq{{\bm q}}
\def\bms{{\bm s}}
\def\bmt{{\bm t}}
\def\bmu{{\bm u}}
\def\bmv{{\bm v}}
\def\bmw{{\bm w}}
\def\bmx{{\bm x}}
\def\bmy{{\bm y}}
\def\bmz{{\bm z}}

% Boldface mathmode numbers
\def\bmzero{{\bm 0}}
\def\bmone{{\bm 1}}
\def\bmtwo{{\bm 2}}
\def\bmthree{{\bm 3}}

% Boldface mathmode uppercase latin letters
\def\bmA{{\bm A}}
\def\bmB{{\bm B}}
\def\bmC{{\bm C}}
\def\bmD{{\bm D}}
\def\bmE{{\bm E}}
\def\bmF{{\bm F}}
\def\bmG{{\bm G}}
\def\bmH{{\bm H}}
\def\bmK{{\bm K}}
\def\bmL{{\bm L}}
\def\bmM{{\bm M}}
\def\bmN{{\bm N}}
\def\bmP{{\bm P}}
\def\bmQ{{\bm Q}}
\def\bmR{{\bm R}}
\def\bmS{{\bm S}}
\def\bmT{{\bm T}}
\def\bmX{{\bm X}}
\def\bmZ{{\bm Z}}


\def\Riem{{\bm R}{\bm i}{\bm e}{\bm m}}
\def\Ric{{\bm R}{\bm i}{\bm c}}
\def\Weyl{{\bm W}{\bm e}{\bm y}{\bm l}}
\def\RWeyl{{\bm R}{\bm W}{\bm e}{\bm y}{\bm l}}
\def\Sch{{\bm S}{\bmc}{\bm h}}
\def\Schouten{{\bm S}{\bmc}{\bm h}{\bm o}{\bm u}{\bm t}{\bm e}{\bm n}}
\def\Hessian{{\bm H}{\bm e}{\bm s}{\bm s}}

% Fracture letters
\def\fraka{{\frak a}}
\def\frakb{{\frak b}}
\def\frakc{{\frak c}}
\def\frakd{{\frak d}}
\def\frakf{{\frak f}}
\def\frakg{{\frak g}}
\def\fraki{{\frak i}}
\def\frakj{{\frak j}}
\def\frakk{{\frak k}}

% Mathbf letters
\def\mbfu{\mathbf{u}}

% Boldface mathmode lowcase greek letters
\def\bmalpha{{\bm \alpha}}
\def\bmbeta{{\bm \beta}}
\def\bmgamma{{\bm \gamma}}
\def\bmdelta{{\bm \delta}}
\def\bmepsilon{{\bm \epsilon}}
\def\bmeta{{\bm \eta}}
\def\bmzeta{{\bm\zeta}}
\def\bmxi{{\bm \xi}}
\def\bmchi{{\bm \chi}}
\def\bmiota{{\bm \iota}}
\def\bmomega{{\bm \omega}}
\def\bmlambda{{\bm \lambda}}
\def\bmmu{{\bm \mu}}
\def\bmnu{{\bm \nu}}
\def\bmphi{{\bm \phi}}
\def\bmvarphi{{\bm \varphi}}
\def\bmsigma{{\bm \sigma}}
\def\bmvarsigma{{\bm \varsigma}}
\def\bmtau{{\bm \tau}}
\def\bmupsilon{{\bm \upsilon}}

% Boldface mathmode uppercase greek letters
\def\bmGamma{{\bm \Gamma}}
\def\bmPhi{{\bm \Phi}}
\def\bmUpsilon{{\bm \Upsilon}}
\def\bmSigma{{\bm \Sigma}}

% Boldface operators
\def\bmpartial{{\bm \partial}}
\def\bmnabla{{\bm \nabla}}
\def\bmhbar{{\bm \hbar}}
\def\bmperp{{\bm \perp}}
\def\bmell{{\bm \ell}}

% Complex and real numbers
\font\SYM=msbm10
\newcommand{\Real}{\mbox{\SYM R}}
\newcommand{\Complex}{\mbox{\SYM C}}
\newcommand{\Natural}{\mbox{\SYM N}}
\newcommand{\Integer}{\mbox{\SYM Z}}
\newcommand{\Sphere}{\mbox{\SYM S}}

% ParallelPerp symbol
\newcommand{\parperp}{\mathbin{\text{\rotatebox[origin=c]{90}{$\models$}}}}
\newcommand{\perppar}{\mathbin{\text{\rotatebox[origin=c]{-90}{$\models$}}}}

%%%%% for piecewise functions right hand side brace %%%%%
\newenvironment{rcases}
  {\left.\begin{aligned}}
  {\end{aligned}\right\rbrace}

% Projector symbols
%\newcommand{\proj2perp}{\pi^{\small{\perp}}}
%\newcommand{\proj2par}{\pi^{\small{\parallel}}}
%\newcommand{\proj4par}{\pi^{\small{\parallel}}}
%\newcommand{\proj4parperp}{\pi^{\small{\parperp}}}
%\newcommand{\proj4perppar}{\pi^{\small{\perppar}}}
%\newcommand{\proj4perp}{\pi^{\small{\perp}}}

%Counter variable for the margin notes
\newcounter{mnotecount}%[section]

% This code generates the margin notes
\newcommand{\mnotex}[1]%{}
{\protect{\stepcounter{mnotecount}}$^{\mbox{\footnotesize $\bullet$\themnotecount}}$ 
\marginpar{%\color{red}%
\raggedright\tiny\em
$\!\!\!\!\!\!\,\bullet$\themnotecount: #1} }

\renewcommand\labelitemi{\tiny$\bullet$}

\newcommand{\notimplies}{%
  \mathrel{{\ooalign{\hidewidth$\not\phantom{=}$\hidewidth\cr$\implies$}}}}


%%%%%%%%%%%%%%%%%%%%%%% needed for long lists of equations
\allowdisplaybreaks
%%%%%%%%%%%%%%%%%%%%%%%%

\begin{document}


\title{\textbf{Matter alignment condition notes}}

\author[1]{E. Gasper\'in \footnote{E-mail address:{\tt edgar.gasperin@tecnico.ulisboa.pt}}}
\author[2]{J.L. Williams \footnote{E-mail address:{\tt jlw31@bath.ac.uk}}}
\affil[1]{CENTRA, Departamento de F\'isica,
  Instituto Superior T\'ecnico IST, Universidade de Lisboa UL, Avenida
  Rovisco Pais 1, 1049 Lisboa, Portugal.}
\affil[2]{Department of Mathematical Sciences, University of Bath, Claverton Down, Bath BA2 7AY, United Kingdom.}



\maketitle
\begin{abstract}
Note on the matter alignment condition
\end{abstract}

\section{Matter alignment condition}


The physical Einstein-Maxwell equations read:
\[
\tilde{\Phi}_{ABA'B'}=2\tilde{\phi}_{AB}\bar{\tilde{\phi}}_{A'B'}
\]
where $\tilde{\Phi}_{ABA'B'}$ and $\tilde{\phi}_{AB}$ denote
respectively the physical trace-free Ricci spinor and the physical
Maxwell spinor.  The physical matter alignment zero-quantity defined as
\[
\tilde{\Theta}_{AB}:=2 \tilde{\kappa}_{(A}{}^{Q}\tilde{\phi}_{B)Q}
\]
The relation between the physical Killing spinor, the physical Maxwell
spinor and their unphysical counterparts is given by
\[
\kappa_{AB}=\Xi^2\tilde{\kappa}_{AB}, \qquad
\phi_{AB}=\Xi^{-1}\tilde{\phi}_{AB}
\]
Hence, using that the relation between the physical and unphysical
$\epsilon$ spinors is
\[
\epsilon_{AB}=\Xi \tilde{\epsilon}_{AB}, \qquad
\epsilon^{AB}=\Xi^{-1}\tilde{\epsilon}^{AB}
\]
and spinor indices of physical quantities, should be moved using the
$\tilde{\bm\epsilon}$ spinor one has that
\[
\tilde{\kappa}_{A}{}^{Q}=\Xi^{-1}\kappa_{A}{}^Q
\]
Hence, the matter alignment condition is conformally invariant, in the
sense that
\[
 \tilde{\kappa}_{(A}{}^{Q}\tilde{\phi}_{B)Q} =
 \kappa_{(A}{}^{Q}{\phi}_{B)Q}
 \]
 This only means that if the physical matter alignment
 condition is satisfied then $\bm\phi \propto \bm\kappa$.
 For notational purposes define the unphysical matter alignment condition zero-quantity as
\[
\Theta_{AB}:=2 \kappa_{(A}{}^{Q}\phi_{B)Q}
\]
Observe that $\Theta_{AB}=0$
does not  imply that the auxiliary vector $\xi_{AA'}$ is a Killing or conformal
 Killing vector of $(\mathcal{M},\bmg)$.
 To see this recall that the conformal
 transfomation formula for the trace-free Ricci spinor is given by
 \[
\tilde{\Phi}_{ABA'B'}=\Phi_{ABA'B'} +
\Xi^{-1}\nabla_{A(A'}\nabla_{B')B}\Xi
 \]
 Defining the shorthand
  \[
 S_{AA'BB'} := \nabla_{QA'}H_{B'}{}^{Q}{}_{AB}
 \]
 where the auxiliary spinors $\xi_{AA'}$, $Q_{AB}$ and the
 zero-quantity $H_{A'ABC}$ are as defined as in GasWil22, then a calculation gives
  \[
 S_{AA'BB'} = \tfrac{1}{2} Q_{AB}\epsilon _{A'B'}  -
 \nabla_{AA'}\xi _{BB'} - \nabla_{BB'}\xi _{AA'} -6 \kappa _{(A}{}^{Q} \Phi _{B)QA'B'}  
 \]
 Using the conformal transformation law for the trace-free Ricci spinor, and the relation
 between the pysical Killing spinors gives
 \[
\kappa_A{}^{Q}\Phi_{BQA'B'}=\Xi\tilde{\kappa}_{A}{}^Q\tilde{\Phi}_{BQA'B'}-\Xi^{-1}\kappa_{A}{}^Q\nabla_{B(A'}\nabla_{B')Q}\Xi
\]
Thus
%\[
% \kappa_{(A}{}^{Q}\Phi_{B)QA'B'}=\tfrac{1}{2}\Xi\Theta_{AB}\bar{\tilde{\phi}}_{A'B'}
%-\Xi^{-1}\kappa_{A}{}^Q\nabla_{B(A'}\nabla_{B')Q}\Xi
%\]
\[
\kappa_{(A}{}^{Q}\Phi_{B)QA'B'}=\Theta_{AB}\bar{\phi}_{A'B'}-\Xi^{-1}\kappa_{(A}{}^Q\nabla_{B)(A'}\nabla_{B')Q}\Xi
\]
Altogether
  \[
 S_{AA'BB'} = \tfrac{1}{2} Q_{AB}\epsilon _{A'B'} -6 \Theta_{AB}\bar{\phi}_{A'B'}  -
 \nabla_{AA'}\xi _{BB'} - \nabla_{BB'}\xi _{AA'}  + 6 \Xi^{-1}\kappa_{(A}{}^Q\nabla_{B)(A'}\nabla_{B')Q}\Xi
 \]
 Thus even when the unphysical matter alignment condition $\Theta_{AB}=0$ and the
 Killing spinor equation $H_{A'ABC}=0$ is satisfied (and hence their concomitants $Q_{AB}=0$ and $S_{AA'BB'}=0$) the auxiliary vector $\xi_{AA'}$ is not a Killing nor a conformal Killing vector of $(\mathcal{M}, \bmg)$.
 



\end{document}
