%%%%%%%
%%%%%%% Conformal Killing Spinor Initial Data 
%%%%%%% 
%%%%%%% arXiv version.
%%%%%%%
%%%%%%%
%%%%%%% Started on: 11.9.2016
%%%%%%% Current version: 09.11.2021
%%%%%%%

\documentclass[10pt,a4paper]{article}
\usepackage{amssymb}
\usepackage{amsmath}
\usepackage{amsfonts}
\usepackage{amsthm}
\usepackage{latexsym}
\usepackage{mathrsfs}
\usepackage{stmaryrd}
\usepackage[dvips]{epsfig}
\usepackage{setspace}
\usepackage{float}
\usepackage{bm}
%\usepackage{showkeys}
%\usepackage[pdftex]{graphicx}
\usepackage{tikz}
\usepackage{enumerate}
\usepackage{subfigure}
\usepackage{nomencl}
%\usepackage{makeidx} 
%\makeindex
\usepackage{authblk}
\renewcommand\Affilfont{\itshape\small}
\usepackage{textcomp}
\usepackage{scrextend}% add KOMA-Script features to other classes
\usepackage[toc,page]{appendix}
\usepackage{comment}
\usepackage{xcolor}


\theoremstyle{plain}
\newtheorem{proposition}{Proposition}
\newtheorem{lemma}{Lemma}
\newtheorem{theorem}{Theorem}
\newtheorem{assumption}{Assumption}
\newtheorem*{conjecture}{Conjecture}
\newtheorem*{subconjecture}{Subconjecture}
\newtheorem{corollary}{Corollary}
\newtheorem*{main}{Theorem}
\newtheorem*{definition}{Definition}
\newtheorem{remark}{Remark}

\setlength{\textwidth}{148mm}           % Width of text on page- max 148
\setlength{\textheight}{235mm}          % height of text on page-max 235
\setlength{\topmargin}{-10mm}            % Margin at top ofpage- max -5
\setlength{\oddsidemargin}{0mm}         % Odd page sidemargin max 15
\setlength{\evensidemargin}{0mm}

% Underlined lowcase latin letters
\def\es{{\bar{s}}}
\def\er{{\bar{r}}}

% Boldface mathmode lowcase latin letters
\def\bma{{\bm a}}
\def\bmb{{\bm b}}
\def\bmc{{\bm c}}
\def\bmd{{\bm d}}
\def\bme{{\bm e}}
\def\bmf{{\bm f}}
\def\bmg{{\bm g}}
\def\bmh{{\bm h}}
\def\bmi{{\bm i}}
\def\bmj{{\bm j}}
\def\bmk{{\bm k}}
\def\bml{{\bm l}}
\def\bmn{{\bm n}}
\def\bmm{{\bm m}}
\def\bmo{{\bm o}}
\def\bmq{{\bm q}}
\def\bms{{\bm s}}
\def\bmt{{\bm t}}
\def\bmu{{\bm u}}
\def\bmv{{\bm v}}
\def\bmw{{\bm w}}
\def\bmx{{\bm x}}
\def\bmy{{\bm y}}
\def\bmz{{\bm z}}

% Boldface mathmode numbers
\def\bmzero{{\bm 0}}
\def\bmone{{\bm 1}}
\def\bmtwo{{\bm 2}}
\def\bmthree{{\bm 3}}

% Boldface mathmode uppercase latin letters
\def\bmA{{\bm A}}
\def\bmB{{\bm B}}
\def\bmC{{\bm C}}
\def\bmD{{\bm D}}
\def\bmE{{\bm E}}
\def\bmF{{\bm F}}
\def\bmG{{\bm G}}
\def\bmH{{\bm H}}
\def\bmK{{\bm K}}
\def\bmL{{\bm L}}
\def\bmM{{\bm M}}
\def\bmN{{\bm N}}
\def\bmP{{\bm P}}
\def\bmQ{{\bm Q}}
\def\bmR{{\bm R}}
\def\bmS{{\bm S}}
\def\bmT{{\bm T}}
\def\bmX{{\bm X}}
\def\bmZ{{\bm Z}}


\def\Riem{{\bm R}{\bm i}{\bm e}{\bm m}}
\def\Ric{{\bm R}{\bm i}{\bm c}}
\def\Weyl{{\bm W}{\bm e}{\bm y}{\bm l}}
\def\RWeyl{{\bm R}{\bm W}{\bm e}{\bm y}{\bm l}}
\def\Sch{{\bm S}{\bmc}{\bm h}}
\def\Schouten{{\bm S}{\bmc}{\bm h}{\bm o}{\bm u}{\bm t}{\bm e}{\bm n}}
\def\Hessian{{\bm H}{\bm e}{\bm s}{\bm s}}

% Fracture letters
\def\fraka{{\frak a}}
\def\frakb{{\frak b}}
\def\frakc{{\frak c}}
\def\frakd{{\frak d}}
\def\frakf{{\frak f}}
\def\frakg{{\frak g}}
\def\fraki{{\frak i}}
\def\frakj{{\frak j}}
\def\frakk{{\frak k}}

% Mathbf letters
\def\mbfu{\mathbf{u}}

% Boldface mathmode lowcase greek letters
\def\bmalpha{{\bm \alpha}}
\def\bmbeta{{\bm \beta}}
\def\bmgamma{{\bm \gamma}}
\def\bmdelta{{\bm \delta}}
\def\bmepsilon{{\bm \epsilon}}
\def\bmeta{{\bm \eta}}
\def\bmzeta{{\bm\zeta}}
\def\bmxi{{\bm \xi}}
\def\bmchi{{\bm \chi}}
\def\bmiota{{\bm \iota}}
\def\bmomega{{\bm \omega}}
\def\bmlambda{{\bm \lambda}}
\def\bmmu{{\bm \mu}}
\def\bmnu{{\bm \nu}}
\def\bmphi{{\bm \phi}}
\def\bmvarphi{{\bm \varphi}}
\def\bmsigma{{\bm \sigma}}
\def\bmvarsigma{{\bm \varsigma}}
\def\bmtau{{\bm \tau}}
\def\bmupsilon{{\bm \upsilon}}

% Boldface mathmode uppercase greek letters
\def\bmGamma{{\bm \Gamma}}
\def\bmPhi{{\bm \Phi}}
\def\bmUpsilon{{\bm \Upsilon}}
\def\bmSigma{{\bm \Sigma}}

% Boldface operators
\def\bmpartial{{\bm \partial}}
\def\bmnabla{{\bm \nabla}}
\def\bmhbar{{\bm \hbar}}
\def\bmperp{{\bm \perp}}
\def\bmell{{\bm \ell}}

% Complex and real numbers
\font\SYM=msbm10
\newcommand{\Real}{\mbox{\SYM R}}
\newcommand{\Complex}{\mbox{\SYM C}}
\newcommand{\Natural}{\mbox{\SYM N}}
\newcommand{\Integer}{\mbox{\SYM Z}}
\newcommand{\Sphere}{\mbox{\SYM S}}

% ParallelPerp symbol
\newcommand{\parperp}{\mathbin{\text{\rotatebox[origin=c]{90}{$\models$}}}}
\newcommand{\perppar}{\mathbin{\text{\rotatebox[origin=c]{-90}{$\models$}}}}

%%%%% for piecewise functions right hand side brace %%%%%
\newenvironment{rcases}
  {\left.\begin{aligned}}
  {\end{aligned}\right\rbrace}

% Projector symbols
%\newcommand{\proj2perp}{\pi^{\small{\perp}}}
%\newcommand{\proj2par}{\pi^{\small{\parallel}}}
%\newcommand{\proj4par}{\pi^{\small{\parallel}}}
%\newcommand{\proj4parperp}{\pi^{\small{\parperp}}}
%\newcommand{\proj4perppar}{\pi^{\small{\perppar}}}
%\newcommand{\proj4perp}{\pi^{\small{\perp}}}

%Counter variable for the margin notes
\newcounter{mnotecount}%[section]

% This code generates the margin notes
\newcommand{\mnotex}[1]%{}
{\protect{\stepcounter{mnotecount}}$^{\mbox{\footnotesize $\bullet$\themnotecount}}$ 
\marginpar{%\color{red}%
\raggedright\tiny\em
$\!\!\!\!\!\!\,\bullet$\themnotecount: #1} }

\renewcommand\labelitemi{\tiny$\bullet$}

\newcommand{\notimplies}{%
  \mathrel{{\ooalign{\hidewidth$\not\phantom{=}$\hidewidth\cr$\implies$}}}}


%%%%%%%%%%%%%%%%%%%%%%% needed for long lists of equations
\allowdisplaybreaks
%%%%%%%%%%%%%%%%%%%%%%%%

\begin{document}


\title{\textbf{The conformal Killing spinor initial data equations}}

\author[1]{E. Gasper\'in \footnote{E-mail address:{\tt edgar.gasperin@tecnico.ulisboa.pt}}}
\author[2]{J.L. Williams \footnote{E-mail address:{\tt jlw31@bath.ac.uk}}}
\affil[1]{CENTRA, Departamento de F\'isica,
  Instituto Superior T\'ecnico IST, Universidade de Lisboa UL, Avenida
  Rovisco Pais 1, 1049 Lisboa, Portugal.}
\affil[2]{Department of Mathematical Sciences, University of Bath, Claverton Down, Bath BA2 7AY, United Kingdom.}



\maketitle
\begin{abstract}
 We obtain necessary and sufficient conditions for an initial data set
 for the \emph{vacuum conformal Einstein field equations} to give rise
 to a spacetime development in possession of a Killing spinor.
%%%%%%%%%%%%%%%%%%%%%%%%%%%%%%%%%%%%%%%%
% This  constitutes the conformal analogue of the Killing spinor initial data
% equations derived in \cite{GarVal08c}.
 %%%%%%%%%%%%%%%%%%%%%%%%%%%%%%%%%%%%%%
 The fact that the conformal Einstein
 field equations are used in our derivation allows for the possibility
 that the initial hypersurface be (part of) the conformal boundary
 $\mathscr{I}$. For conciseness, these conditions are derived assuming
 that the initial hypersurface is spacelike. Consequently, these
 equations encode necessary and sufficient conditions for the
 existence of a Killing spinor in the development of asymptotic
 initial data on spacelike components of $\mathscr{I}$.
\end{abstract}


\section{Introduction}

The discussion of symmetries  in General
Relativity is ubiquitous. From the question of integrability of the geodesic
equations to the existence of explicit solutions to the Einstein field
equations and the black hole uniqueness problem, symmetries always
  play an important role.   Symmetry assumptions are usually incorporated into
 the Einstein field equations ---which in vacuum read
\begin{equation}
\tilde{R}_{ab}=\lambda \tilde{g}_{ab},
\label{EFEVacuum}
\end{equation} 
 through the use of Killing vectors.  From the spacetime point
 of view, the existence of Killing vectors allows one to perform
 \emph{symmetry reductions} of the Einstein field equations ---see for
 instance \cite{Wei90a}. This approach has been exploited in classical
 uniqueness results such as \cite{Rob75b}.  Closely related to the black
 hole uniqueness problem, characterisations and classifications of
  solutions to the Einstein field equations usually exploit the
 symmetries of the spacetime in one way or another, e.g., in the
 characterisations of the Kerr spacetime via the \emph{Mars-Simon
   tensor} ---see \cite{Mar99,Mar00,Sim84}.  On the other hand, from
 the point of view of the Cauchy problem, symmetry assumptions should
 be imposed only at the level of initial data. In this regard,
 symmetry assumptions can be phrased in terms of the \emph{Killing
   vector initial data}.  The Killing vector initial data equations
 constitute a set of conditions that an initial data set
 $(\tilde{\mathcal{S}},\tilde{\bmh},\tilde{\bmK})$ for the Einstein
 field equations has to satisfy to ensure that the development will
 contain a Killing vector ---see \cite{BeiChr97b}.  Nevertheless,
 despite the fact that the existence of Killing vector plays a
central  role in the discussion of the symmetries, the
 existence of Killing vectors is sometimes not enough to encode all
 the symmetries and conserved quantities that a spacetime can posses,
 e.g., the Carter constant in the Kerr spacetime. To unravel some of
 these \emph{hidden symmetries} one has analyse the existence of a
 more fundamental type of objects; \emph{Killing spinors}
 $\tilde{\kappa}_{AB}$ ---in vacuum spacetimes, the existence of a
 Killing spinor directly implies the existence of a Killing
 vector. The \emph{Killing spinor initial data equations} have been
 derived in the \emph{physical framework} ---governed by the Einstein field
   equations--- in \cite{GarVal08c}.  These equations have been
 successfully employed in the construction of a geometric invariant
 which detects whether or not an initial data set corresponds to
 initial data for the Kerr spacetime ---see
 \cite{BaeVal10a,BaeVal10b,BaeVal11b}.
This analysis has also been extended to include suitable classes of
matter ---see \cite{ValCol16} for an analogous characterisation of
initial data for the Kerr-Newman spacetime.  In these
characterisations, some asymptotic conditions on the initial data are
required. These conditions  usually take the form of decay assumptions
on   $\tilde{\bmh}$, $\tilde{\bmK}$ and $\tilde{\bm\kappa}$ on
$\tilde{\mathcal{S}}$, given in terms
of asymptotically Cartesian coordinates.  Nonetheless, in other
approaches, the asymptotic behaviour of the spacetime can be studied in
a geometric way through conformal compactifications. The latter is
sometimes referred as the Penrose proposal. In this approach one
starts with a \emph{physical spacetime}
$(\tilde{\mathcal{M}},\tilde{\bmg})$ where $\tilde{\mathcal{M}}$ is a
4-dimensional manifold and $\tilde{\bmg}$ is a Lorentzian metric which
is a solution to the Einstein field equations.  Then, one introduces a
\emph{unphysical spacetime} $(\mathcal{M},\bmg)$ into which
$(\tilde{\mathcal{M}},\tilde{\bmg})$ is conformally embedded.
Accordingly, there exists an embedding $\varphi: \tilde{\mathcal{M}}
\rightarrow \mathcal{M}$ such that
\begin{equation} \label{eqn:Chapter:Introduction:ConformalRescaling}
\varphi^{*}\bmg=\Xi^2\tilde{\bmg}.
\end{equation}
 By suitably choosing the \emph{conformal factor} $\Xi$ the metric
 $\bmg$ may be well defined at the points where $\Xi=0$. In such
 cases, the set of points for where the conformal factor vanishes is
 at infinity from the physical spacetime perspective.
\noindent The set
\[
 \mathscr{I} \equiv \big\{p \in \mathcal{M} \hspace{0.2cm}| \hspace{0.2cm} \Xi(p)=0
 , \hspace{0.2cm} \mathbf{d}\Xi(p) \neq0\big\}
\]
is called the conformal boundary.  However, it can be readily
verified that the Einstein field equations are not conformally
invariant. Moreover, a direct computation using the conformal
transformation formula for the Ricci tensor shows that the vacuum
Einstein field equations \eqref{EFEVacuum}, lead to an equation which
is formally singular at the conformal boundary.  An approach to deal
with this problem was given in \cite{Fri81a} where a regular set of
equations for the unphysical metric was derived. These equations are
known as the \emph{conformal Einstein field equations}.  The crucial
property of these equations is that they are regular at the points
where $\Xi=0$ and a solution thereof implies whenever $\Xi\neq 0$ a
solution to the Einstein field equations ---see \cite{Fri81a,Fri83}
and \cite{CFEbook} for an comprehensive discussion.  There are three
ways in which these equations can be presented, the metric, the frame
and spinorial formulations. These equations have been mainly used in
the stability analysis of spacetimes ---see for instance \cite{Fri86b,
  Fri86c} for the proof of the global and semiglobal non-linear
stability of the de Sitter and Minkowski spacetimes, respectively.

\medskip

A conformal version of the Killing vector initial data equations using
the metric formulation of the conformal Einstein field equations has
been obtained in \cite{Pae14a}.  In the latter reference, intrinsic
conditions on an initial hypersurface $\mathcal{S}\subset \mathcal{M}$
of the unphysical spacetime are found such that the development of the
data ---in the unphysical setting the evolution is governed by the
conformal Einstein field equations--- gives rise to a conformal
Killing vector of the unphysical spacetime $(\mathcal{M},\bmg)$ which, in
turn, corresponds to a Killing vector of the physical spacetime
$(\tilde{\mathcal{M}},\tilde{\bmg})$.  Notice that this approach, in
particular, allows $\mathcal{S}$ to be determined by $\Xi=0$ so that
it to corresponds to the conformal boundary $\mathscr{I}$.  The
unphysical Killing vector initial data equations have been derived for
the characteristic initial value problem on a cone in \cite{Pae14a}
and on a spacelike conformal boundary in \cite{Pae14}.


\medskip
For applications involving the the conformal Einstein field equations
---say in its spinorial formulation, one frequently has to fix the
gauge and write the equations in components.  Despite the fact that,
at first glance, the conformal Einstein field equations expressed in
components with respect to an arbitrary spin frame seem to be
overwhelmingly complicated, as shown in \cite{GasVal17}, symmetry
assumptions (spherical symmetry in the latter case) greatly reduce the
number of equations to be analysed.  In the case of Petrov type D
spacetimes, e.g.the Kerr-de Sitter spacetime, the symmetries of the
spacetime are closely related to the existence of Killing spinors.
Therefore, a natural question in this setting is whether a conformal
version of the Killing spinor initial data equations introduced in
\cite{GarVal08c} can be found. In other words, what are the extra
conditions that one has to impose on an initial data set for the
conformal Einstein field equations so that the arising development
contains a Killing spinor?  This question is answered in this article
by deriving such conditions which we call the \emph{conformal Killing
  spinor initial data equations}

Despite the fact that the
Killing spinor equation is conformally invariant, it is not a priori
clear whether the conditions 
of \cite{GarVal08c, BaeVal10b} may be translated directly into the unphysical setting.
Indeed, one expects this not to be the case, since the Einstein field equations
are not conformally invariant. Moreover, one consideration that is
exploited in the discussion of \cite{GarVal08c} is based on the fact
that, on an Einstein spacetime $(\tilde{\mathcal{M}},\tilde{\bmg})$, a
Killing spinor $\tilde{\kappa}_{AB}$ gives rise to a Killing vector
$\tilde{\xi}_{a}$ whose spinorial counterpart is given by
$\tilde{\xi}_{AA'}=\tilde{\nabla}_{A'}{}^{Q}\tilde{\kappa}_{QA}$. Nevertheless,
this property does not hold in general.


%% In other words, if
%% $(\mathcal{M},\bmg)$, where $\bmg$ is not assumed to satisfy the
%% Einstein field equations, possess a Killing spinor $\kappa_{AB}$, then
%% the analogous concomitant $\xi_{AA'}=\nabla_{A'}{}^{Q}\kappa_{QA}$
%% does not correspond to a Killing vector ---not even a conformal Killing vector.
%% This situation is not ameliorated if one assumes that
%% $(\mathcal{M},\bmg)$ satisfies the conformal Einstein field equations.

%% Nevertheless, as discussed in this article, in the latter
%% case one can show that it corresponds to
%% a Weyl curvature collineation and using the conformal factor $\Xi$, the Killing
%% spinor $\kappa_{AB}$ and the \emph{auxiliary vector} $\xi_{a}$, one
%% can construct a conformal Killing vector $X_{a}$ associated to a
%% Killing vector $\tilde{X}_{a}$ of the physical spacetime
%% $(\tilde{\mathcal{M}},\tilde{\bmg})$.

%% The conditions of \cite{GarVal08c, BaeVal10b}
%% may be recovered from the results presented here by setting $\Xi = 1$.


\medskip


{\color{blue}{ 

The analysis carried out in this article can be considered as the
conformal analogue of the Killing spinor initial data equations
derived in \cite{GarVal08c}. Although the results of \cite{GarVal08c}
may be recovered from the analysis presented here by setting $\Xi =
1$, an important difference is that the set of variables that allow to
obtain a closed system of homogeneous wave equations in the present
case are different. The need for a different set of
\emph{Killing spinor zero-quantities} %to be propagated
in the conformal case, can be
traced back to the previous observation that in $(\mathcal{M},\bmg)$
the vector
    $\xi_{AA'}=\nabla_{A'}{}^{Q}\kappa_{QA}$ does not correspond to a
    (conformal) Killing vector. However, as by product of the present
    analysis it is shwon that $\xi_{AA'}$ is a Weyl collineation ---see
    \cite{KatLevDav69} for definitions of curvature collineations.
    Additionally, it is shown that using the conformal factor $\Xi$, the Killing spinor
    $\kappa_{AB}$ and $\xi_{AA'}$, one can
    construct a conformal Killing vector $X_{a}$ associated to a
    Killing vector $\tilde{X}_{a}$ of the physical spacetime
    $(\tilde{\mathcal{M}},\tilde{\bmg})$.  In the analysis of
    \cite{GarVal08c} the fact that
    $\tilde{\xi}_{AA'}=\tilde{\nabla}_{A'}{}^{Q}\tilde{\kappa}_{QA}$
    is a Killing vector is crucial since one propagates off the initial hypersurface,
    simultaneously,   $\tilde{\kappa}_{AA'}$ and the Killing vector $\tilde{\xi}^a$
    by introducing $\tilde{S}_{ab} \equiv \tilde{\nabla}_{(a}\tilde{\xi}_{b)}$ as a zero-quantity.
    Similarly, in the work of \cite{ValCol16} where the results of \cite{GarVal08c} are
    generalised to the case where $(\tilde{\mathcal{M}},\tilde{\bmg})$ satisfies the
    Einstein-Maxwell equations, the condition
    $\tilde{S}_{ab}=0$ is also verified by virtue of the so-called
    \emph{matter aligment condition}.  In the conformal setting analysed in this article,
     the analogous quantity $S_{ab}$ is not as geometrically motivated as in the previous cases
    and its usage as a variable in the system does not lead to a closed system
    of explicitly regular homogeneous wave equations. Here the adjective regular
     refers to the absence of formally singular terms, such as $\Xi^{-1}$, in the
    equations. Instead, the variable that is central for the present analysis
    turns out to be the so-called \emph{Buchdahl constraint} (and derivatives
    thereof), which links directly the existence of Killing spinors
    with the Petrov type of $(\mathcal{M},\bmg)$.

    \medskip

    Although the main
   objective of the present paper is deriving the valence-2 Killing spinor initial
   data in the conformal setting $(\mathcal{M},\bmg)$,
    the analogous conditions encoding the
    existence of a valence-1 Killing spinor are also derived.   The latter serves as a
    warm up exercise for the valence-2 case where one can
    already observe the above discussed features and understand
    differences between the derivation of the conditions on
    $(\tilde{\mathcal{M}},\tilde{\bmg})$ and those on
    $(\mathcal{M},\bmg)$ in a simpler arena.
}}

\medskip

   %% An interesting feature of our analysis is the fact that we make use of an
   %% alternative representation of the conformal Einstein field
   %% equations.  In
   %% principle, one could use the standard representation of the
   %% conformal Einstein field equations, however, some
   %% experimentation reveals that the latter approach leads to 
   %% Fuchsian systems of equations ---formally singular at the conformal
   %% boundary--- for quantities associated to the Killing spinor.


   For
   conciseness, the conformal Killing initial data equations are
   obtained on a spacelike hypersurface $\mathcal{S}$. Nonetheless, a
   similar computation can be performed on an hypersurface
   $\mathcal{S}$ with a different causal character.  The conditions
   found in this article have potential applications for the black
   hole uniqueness problem.  In particular, they can be used for an
   asymptotic characterisation of the Kerr-de Sitter spacetime analogous to
   \cite{MarPaeSenSim16} in terms of the existence of Killing spinors at
   the conformal boundary $\mathscr{I}$.

\medskip

The main results of this article are summarised  informally
 in the following:


\begin{main}\label{TheoremSummary}
If the conformal Killing spinor initial data equations
 \eqref{CS-KID1}-\eqref{CS-KID3} are satisfied
on an open set $\mathcal{U}\subset \mathcal{S}$, where
 $\mathcal{S}$ is a spacelike hypersurface on which initial data for 
the conformal Einstein field equations has been prescribed,
 then, the domain of dependence of $\mathcal{U}$  possesses a Killing spinor.
%% Moreover, assuming conditions \eqref{CS-KID1}, \eqref{CS-KID2} to hold, 
%% condition \eqref{CS-KID3} is equivalent to the vanishing of certain components
%%  of the Cotton spinor, with respect to a suitably-adapted spin dyad.
\end{main}

 A precise formulation is the content of Theorem
\ref{} and Proposition \ref{}.

\medskip

Involved computations throughout this article were facilitated through
the suite {\tt xAct} in {\tt Mathematica}.
Note that since the existence of a spinor structure is guaranteed for
globally-hyperbolic spacetimes ---see Proposition $4$ in
\cite{CFEbook}--- the use of spinors is not overly restrictive.

\newpage

\subsection*{Overview of the article}
{\color{blue}
Section \ref{NotationAndSpinorFormalism} establishes the
conventions and notation to be used in the rest of the paper.  It also
gives an abbridged discussion of the main spinorial identities to be
used and the space spinor formalism.  Section \ref{Sec:KillingSpinors}
gives an overview of Killing spinors and their conformal
properties. In Section \ref{Sec:CFEs}  the conformal
Einstein field equations are given for later use.  In Section
\ref{conformalTwistorKID} the conformal (valence-1 Killing spinor)
twistor initial data equations are obtained. In Section
\ref{} the conformal (valence-2) Killing initial data equations are
derived and discussed.}


%% Section
%% \ref{Sec:KillinSpinorZeroQuantities} introduces the main objects of
%% interest in the propagation of Killing spinor data, namely the
%% \emph{Killing spinor zero-quantities}. In Section
%% \ref{Sec:PropagationEquations} we construct conformally-regular wave
%% equations for the zero-quantities, leading to necessary and sufficient
%% conditions for the existence of a Killing spinor ---see Proposition
%% \ref{Prop:Propagation}. In Section \ref{Sec:IntrinsicConditions} the
%% latter conditions and the space spinor formalism are used to obtain 
%% the conformal Killing spinor initial data equations
%%  on spacelike hypersurfaces---see
%%  Theorem \ref{MainTheorem}.
%% % % Finally, in Section
% % \ref{Sec:FurtherAnalysis} the latter equations are analysed with respect to
% % an adapted spin dyad and the implied restrictions on the Cotton
% % spinor are presented ---see Proposition
% % \ref{PropositionRestrictionOnCotton}.

\section{Notation and spinorial formalism in a nutshell}
\label{NotationAndSpinorFormalism}

\subsection*{Spacetime spinor formalism}
Upper case Latin indices ~$_{ABC\cdots A'B'C'}$~ will be used as
abstract indices of the \emph{spacetime spinor} algebra, and the bold
numerals ~$_{\bm0\bm1\bm2\cdots}$~ denote components with respect to a
fixed spin dyad $ o^A\equiv
\epsilon_{\bm0}{}^A,\iota^A\equiv\epsilon_{\bm1}{}^A $ ---see Penrose
\& Rindler \cite{PenRin84} for further details.  Although spinor
notation will be preferred, for certain computations tensors will be
employed. Lower case Latin indices $_{a,b,c...}$ will be used as
abstract tensor indices.  For tensors, our curvature conventions are
fixed by
\[\nabla_{a}\nabla_{b}\kappa^c-\nabla_{b}\nabla_{a}\kappa^c=R_{ab}{}^{c}{}_{d}\kappa^{d}.\]
For spinors, the curvature conventions are fixed via the spinorial
Ricci identities which will be written in accordance with the above
convention for tensors.  To see this, recall that the commutator of
covariant derivatives $[ \nabla_{AA'},\nabla_{BB'}]$ can be expressed
in terms of the symmetric operator $\square_{AB}$ as
\[
[ \nabla_{AA'},\nabla_{BB'}]= \epsilon_{AB}\square_{A'B'} +
\epsilon_{A'B'}\square_{AB}
\]
where
\[
\square_{AB} \equiv \nabla_{Q'(A} \nabla_{B)}{}^{Q'}.
\]
 The action of the symmetric operator $\square_{AB}$ on valence-1
 spinors is encoded in the spinorial Ricci identities
\begin{subequations}
\begin{eqnarray}
&& \square_{AB}\xi_{C}=-\Psi_{ABCD} \xi^{D} +
  2\Lambda\xi_{(A}\epsilon_{B)C},
 \label{SpinorialRicciIdentities1} \\
&& \square_{A'B'}\xi_{C}=-\xi^{A}\Phi_{CA A' B'},
\label{SpinorialRicciIdentities2}
\end{eqnarray}
\end{subequations}
where $\Psi_{ABCD}$ and $\Phi_{AA'BB'}$ and $\Lambda$ are curvature spinors.
 The above identities can be extended to higher valence spinors in an
 analogous way ---see \cite{Ste91} for further discussion on these
 identities using different conventions to the ones used in this
 article. A related identity which will be systematically used in the
 following discussion is
\begin{equation}\label{DecomposeDoubleDerivativeContracted}
\nabla_{AQ'}\nabla_{B}{}^{Q'}=\square_{AB}+
\frac{1}{2}\epsilon_{AB}\square,
\end{equation}
where $\square_{AB}$ is the symmetric operator defined above and
$\square \equiv \nabla_{AA'}\nabla^{AA'}.$


\subsection*{Space spinor formalism}

To have a self-contained discussion in this section the space spinor
formalism, originally introduced in \cite{Som80}, is briefly recalled
---see also \cite{GarVal08c,BaeVal10b,CFEbook}.  Let $\tau^{AA'}$
denote the spinorial counterpart of a timelike vector $\tau^{a}$,
normal to a spacelike hypersurface $\mathcal{S}$ and normalised so
that $\tau_{a}\tau^{a}=2$.  Then, it follows that
$\tau_{AA'}\tau^{AA'}=2$ and, consequently,
\[\tau_{AA'}\tau_B{}^{A'}=\epsilon_{AB}.\]  
The covariant derivative $\nabla_{AA'}$ is then decomposed into the
\emph{normal} and \emph{Sen} derivatives:
\begin{align*}
  %& \mathcal{P}
  &\nabla_\tau \equiv \tau^{AA'}\nabla_{AA'},\\ & \mathcal{D}_{AB}\equiv
  \tau_{(A}{}^{A'}\nabla_{B)A'}.
\end{align*}
The \emph{Weingarten} spinor and the \emph{acceleration} of the
congruence are then defined by
\begin{align*}
& K_{ABCD} \equiv \tau_{D}{}^{C'} \mathcal{D}_{AB}\tau_{CC'},\\ &
  K_{AB} \equiv \tau_{B}{}^{C'} \mathcal{P}\tau_{AC'}.
\end{align*}
The above can be inverted to obtain the following formulae which will
prove useful in the sequel
\begin{align*}
  %& \mathcal{P}
  & \nabla_\tau \tau_{CC'}=- K_{CD} \tau^{D}{}_{C'},\\ &
  \mathcal{D}_{AB}\tau_{CA'} = - K_{ABCD} \tau^{D}{}_{A'}.
\end{align*}
The distribution induced by $\tau_{AA'}$ is integrable if and only
$K^D{}_{(AB)D}=0$, in which case $K_{ABCD}$ describes the extrinsic
curvature of the resulting foliation.
 Nevertheless, this is not required for our subsequent discussion.
In other words, we will allow
 the possibility that the distribution is non-integrable
---i.e. the spinor $ K^D{}_{(AB)D}$ will not be assumed
to vanish. 

\medskip

Defining the spinors $\chi_{AB}\equiv K^D{}_{(AB)D}$,
$\chi_{ABCD}\equiv K_{(ABCD)}$ and $\chi\equiv K_{AB}{}^{AB}$, the
Weingarten spinor decomposes as follows
\begin{equation}
\label{ExtrinsicCurvatureSplit}
K_{ABCD} = \chi_{ABCD} - \tfrac{1}{2} \epsilon_{A(C}\chi_{D)B} -
\tfrac{1}{2} \epsilon_{B(C}\chi_{D)A} - \tfrac{1}{3} \chi
\epsilon_{A(C} \epsilon_{D)B}.
\end{equation}
For the following discussion we will also need the commutators form with
%$\mathcal{P},~\mathcal{D}_{AB}$.
$\nabla_\tau,~\mathcal{D}_{AB}$.
To write these commutators in a
succinct way, first define
\[\widehat{\square}_{AB}\equiv \tau_A{}^{A'}\tau_B{}^{B'}\square_{A'B'}\]
from which, proceeding analogously as in \cite{BaeVal10b}, one obtains
\begin{align}
 \left[\nabla_\tau ,\mathcal{D}_{AB}
   \right]&=-\tfrac{1}{2}\chi_{AB}-\square_{AB}+\widehat{\square}_{AB}
+K_{(A}{}^D\mathcal{D}_{B)D}-K_{AB}{}^{FG}\mathcal{D}_{FG}, \label{CommutatorNormalSenDeriv}
\\ \left[\mathcal{D}_{AB},\mathcal{D}_{CD}\right]&=
\tfrac{1}{2}\left(\epsilon_{A(C}\square_{D)B}+\epsilon_{B(C}\square_{D)A}\right)
+\tfrac{1}{2}\left(\epsilon_{A(C}\widehat{\square}_{D)B}+\epsilon_{B(C}\widehat{\square}_{D)A}\right)
 \nonumber \\ &
 +\tfrac{1}{2}\left(K_{CDAB}-K_{ABCD}\right)\nabla_\tau
+K_{CDF(A}\mathcal{D}_{B)}{}^F-K_{ABF(C}\mathcal{D}_{D)}{}^F \label{CommutatorSenSenDeriv}
\end{align}


%% \begin{align}
%%  \left[\mathcal{P},\mathcal{D}_{AB}
%%    \right]&=-\tfrac{1}{2}\chi_{AB}-\square_{AB}+\widehat{\square}_{AB}
%% +K_{(A}{}^D\mathcal{D}_{B)D}-K_{AB}{}^{FG}\mathcal{D}_{FG}, \label{CommutatorNormalSenDeriv}
%% \\ \left[\mathcal{D}_{AB},\mathcal{D}_{CD}\right]&=
%% \tfrac{1}{2}\left(\epsilon_{A(C}\square_{D)B}+\epsilon_{B(C}\square_{D)A}\right)
%% +\tfrac{1}{2}\left(\epsilon_{A(C}\widehat{\square}_{D)B}+\epsilon_{B(C}\widehat{\square}_{D)A}\right)
%%  \nonumber \\ &
%%  +\tfrac{1}{2}\left(K_{CDAB}\mathcal{P}-K_{ABCD}\mathcal{P}\right)
%% +K_{CDF(A}\mathcal{D}_{B)}{}^F-K_{ABF(C}\mathcal{D}_{D)}{}^F \label{CommutatorSenSenDeriv}
%% \end{align}

%% It will also prove convenient to decompose the tracefree Ricci spinor,
%% $\Phi_{AA'BB'}$ in space spinor form. To do so, introduce its space
%% spinor counterpart $\Phi_{ABCD}\equiv\tau_{B}{}^{B'} \tau_{D}{}^{D'}
%% \Phi_{ACB'D'}$.  The latter can be decomposed as
%% \begin{align}
%% \label{RicciSpaceSpinorSplit}
%% \Phi_{ABCD}& = \Theta_{ABCD} + \tfrac{1}{2}
%% \left(\epsilon_{C(B}\Phi_{D)A} + \epsilon_{A(B}\Phi_{D)C}\right) -
%% \tfrac{1}{3} \Phi \epsilon_{A(B}\epsilon_{D)C}
%% \end{align}
%% where
%% \[ \Phi \equiv \Phi_{A}{}^{A}{}_{B}{}^{B},\qquad \Phi_{AB} \equiv \Phi_{(AB)C}{}^{C},\qquad
%%  \Theta_{ABCD} \equiv \Phi_{(ABCD)}\]


\section{Killing spinors}\label{Sec:KillingSpinors}

To start the discussion it is convenient to introduce some notation
and definitions. Let $(\tilde{\mathcal{M}},\tilde{\bmg})$ be a
4-dimensional manifold equipped with a Lorentzian metric
$\tilde{\bmg}$ and denote by $\tilde{\nabla}$ its associated
Levi-Civita connection.  For the time being $\tilde{\bmg}$ is not
assumed to be a solution to the Einstein field equations
\eqref{EFEVacuum}.

{\color{blue}
A totally symmetric
$\tilde{\kappa}_{A_1...A_p}=\tilde{\kappa}_{(A_1...A_p)}$ valence$-p$
spinor is said to be a (valence$-p$) \emph{Killing spinor}
if the following equation is
satisfied
\begin{equation}\label{qValenceKillingspinor}
\tilde{\nabla}_{Q'(Q}\tilde{\kappa}_{A_1...A_p)}=0.
\end{equation}
An important property of the Killing spinor equation is that it is
conformally-invariant, in other words if $\bmg$ is conformally related
to $\tilde{\bmg}$, namely $\bmg=\Xi^2\tilde{\bmg}$ then
${\kappa}_{A_1...A_q}=\Xi^2 \tilde{\kappa}_{A_1...A_q}$ satisfies
\[{\nabla}_{Q'(Q}{\kappa}_{A_1...A_p)}=0,\]
where ${\nabla}$ is the Levi--Civita connection of ${\bmg}$.

\medskip
\noindent In this paper we will only focus only the case $p=1$ and $p=2$.
If $p=1$, the equation
%%%%
%then a valence$-1$ Killing spinor $\tilde{\kappa}_A$ satisfies the equation
%%%
\begin{equation}\label{TwistorEq}
  \tilde{\nabla}_{Q'(Q}\tilde{\kappa}_{A)}=0.
\end{equation}
%%%
% which
%%%
is usually referred as the \emph{twistor equation}. We will follow
this naming convention and refer to a valence$-1$ spinor satisfying
equation \eqref{TwistorEq} as twistor.  Since only the cases $p=1$ and
$p=2$ will be discussed in this paper, we will refer to the case $p=1$
as the twistor case and the $p=2$ as the Killing spinor case. Namely,
we will say that a symmetric valence$-2$ spinor,
$\tilde{\kappa}_{AB}=\tilde{\kappa}_{(AB)}$, is a \textit{Killing
  spinor} if it satisfies the equation
\begin{equation}
\tilde{\nabla}_{A'(A}\tilde{\kappa}_{BC)}=0.
\end{equation}

The Killing spinor equation and twistor equations are, in general,
overdetermined; in particular, they imply the so-called
\textit{Buchdahl constraint}.  In the twistor case ($p=1$) this has
the form
\[
\tilde{\kappa}^D\Psi_{ABCD}=0,
\]
while in the Killing spinor case ($p=2$) the Buchdahl constraint
adquires the form
\[
\tilde{\kappa}^Q{}_{(A}\Psi_{BCD)Q}=0,
\]
where $\Psi_{ABCD}$ denotes the conformally invariant Weyl spinor.
The latter condition restricts $\Psi_{ABCD}$ to be algebraically
special.  In the twistor case the spacetime is necessarity of Petrov
type N or O, hence restricting its applicability for characterisation of
black holes.  In the Killing spinor case the spacetime is only
restricted to be of Petrov type D, N or O.

\medskip

At first glance, the conformal invariance property of the
%(valence-$q$)
Killing spinor equation would seem to indicate that the
approach leading to the Killing spinor initial data conditions derived
in \cite{GarVal08c} would identically apply for $(\mathcal{M},\bmg)$
with $\tilde\bmg=\Xi^2\tilde{\bmg}$. This is not the case simply
because the Einstein field equations are not conformally invariant.
In other words, in the analysis of \cite{GarVal08c} the vacuum
Einstein field equations \eqref{EFEVacuum}
%%%%%%%%%%%%%%%%%
%\mnotex{Did Juan and Alfonso's set up was with $\lamda$ arbitrary? or $\lambda=0$}
%with vanishing cosmological constant
%$\tilde{R}_{ab}=0$
%%%%%%%%%%%%%%%
were used, and, despite
that one can relate $R_{ab}$ with $\tilde{R}_{ab}$ this leads to
formally singular terms (terms containing $\Xi^{-1}$). Moreover, even if one
is willing to work with formally singular equations it is not apriori clear
that the choice of variables made in \cite{GarVal08c} will form a
closed homogeneous system in the conformal seeting.  To understand
this second point further, notice that for general manifold with
metric $(\tilde{\mathcal{M}},\tilde{\bmg})$ ---namely $\tilde{\bmg}$
not satifying any field equation--- the existence of a Killing spinor
$\tilde{\kappa}_{AB}$ is not related directly to the existence of a
Killing vector.
Nevertheless, if one assumes that $\tilde{\bmg}$
satisfies the vacuum Einstein field equations \eqref{EFEVacuum} then
the concomitant
\begin{equation*}
\tilde{\xi}_{AA'} \equiv \tilde{\nabla}^{B}{}_{A'}\tilde{\kappa}_{AB},
\end{equation*}
represents the spinorial counterpart of a complex Killing vector of
the spacetime $(\tilde{\mathcal{M}},\tilde{\bmg})$ ---see
\cite{GarVal08c} for further discussion. This point is subtle and even
in the physical (non-conformal) set up if one is to include matter
such as the Maxwell field and the analysis of \cite{GarVal08c} does
not straighfowardly apply since further conditions (the matter
alignment conditions) ---see \cite{ValCol16}--- need to be propagated.
}


\begin{remark}
  \emph{
  The notion of Killing spinors is related to that
  of Killing--Yano tensors. If a Killing spinor
$\tilde{\xi}_{AA'}$ is Hermitian, i.e.,
$\bar{\tilde{\xi}}_{AA'}=\tilde{\xi}_{AA'}$, then one can construct the
spinorial counterpart of a \emph{Killing--Yano tensor}
$\tilde{\Upsilon}_{ab}$ ---i.e. an antisymmetric $2-$tensor satisfying
$\tilde{\nabla}_{(a}\tilde{\Upsilon}_{b)c}=0$--- as follows
\[\tilde{\Upsilon}_{AA'BB'}=i(\tilde{\kappa}_{AB}\bar{\tilde{\epsilon}}_{A'B'}
-\bar{\tilde{\kappa}}_{A'B'}\tilde{\epsilon}_{AB}).\] Conversely,
given a Killing--Yano tensor, one can construct a Killing spinor
---see \cite{ValCol16,McLBer93,PenRin86}.}
\end{remark}


\medskip

In the sequel $(\tilde{\mathcal{M}},\tilde{\bmg})$ will be reserved to
denote the \emph{physical spacetime}, in other words, the symbol
$\tilde{ \quad}$ will be added to those fields associated with a
solution $\tilde{\bmg}$ to the vacuum Einstein field equations
\eqref{EFEVacuum}.  Similarly $(\mathcal{M},\bmg)$ will be used to
represent the \emph{unphysical spacetime} related to
$(\tilde{\mathcal{M}},\tilde{\bmg})$ via $\bmg=\Xi^2\tilde{\bmg}$.
---in a slight abuse of notation $\varphi(\tilde{\mathcal{M}})$ and
$\mathcal{M}$ will be identified so that the mapping $\varphi:
\tilde{\mathcal{M}}\rightarrow\mathcal{M}$ can be omitted.








\section{The  conformal Einstein field equations}
\label{Sec:CFEs}

%% As discussed in the introduction, for the derivation of the conformal
%% Killing initial data equations, an appropriate formulation of the
%% conformal Einstein field equations will be required.

 %In this section
 %we begin, for the sake of completeness, with a discussion of the standard
 %conformal field equations (CFEs) originally introduced in
 %\cite{Fri81a} by H. Friedrich ---see also \cite{CFEbook}.
 

%% Then, an
%% alternative formulation to these equations are presented.  The main
%% benefit of these equations, which we refer to as the \emph{alternative
%%   CFEs}, in our context is that the so-called \emph{rescaled Weyl
%%   tensor} is replaced by the \emph{Weyl tensor} through the
%% introduction of the \emph{Cotton tensor} as an additional unknown.
%% This latter approach was first proposed in \cite{Pae14a} by T. Paetz.


%%  The use of the alternative CFEs is vindicated by our final result
%% which indicates that the existence of a Killing spinor necessarily
%% places restrictions on the components of the Cotton spinor at
%% the level of the initial data ---see Theorem \ref{TheoremSummary},
%% Proposition \ref{PropositionRestrictionOnCotton}.

%%%%%%%%
%\subsection{The standard conformal Einstein field equations}
%%%%%%%


\medskip

This section contains an abriged discussion of the CFEs in first and second order form.
At the end of this section the main technical tool from the theory of partial differential
equations to be used for deriving the Killing spinor intial data equations is given.

\medskip
 

The conformal Einstein field equations are a conformal formulation of
the Einstein field equations. In other words, given a spacetime
$(\tilde{\mathcal{M}},\tilde{\bmg})$ satisfying the Einstein field
equations, the conformal Einstein field equations encode a system of
differential conditions for the curvature and concomitants of the
conformal factor associated with $(\mathcal{M},\bmg)$ where
$\bmg=\Xi^2\tilde{\bmg}$. The key property of these equations is that
they are regular even at the conformal boundary $\mathscr{I}$, where
$\Xi=0$.  This formulation of the conformal Einstein field equations
was first given in \cite{Fri81a} ---see also \cite{CFEbook} for a
comprehensive discussion.

\medskip

The metric version of the standard vacuum conformal 
Einstein field equations are encoded in the following zero-quantities
  ---see \cite{Fri81a,Fri81b,Fri82,Fri83}:
\begin{subequations}\label{CFE_tensor_zeroquants}
\begin{eqnarray}
&& Z_{ab} \equiv \nabla_{a}\nabla_{b}\Xi  +\Xi L_{ab} - s g_{ab}=0 ,
 \label{StandardCEFEsecondderivativeCF}\\
&& Z_{a} \equiv \nabla_{a}s +L_{ac} \nabla ^{c}\Xi=0 , \label{standardCEFEs}\\
&& \delta_{bac} \equiv \nabla_{b}L_{ac}-\nabla_{a}L_{bc} -
 d_{abcd}\nabla^d{}\Xi =0 , \label{standardCEFESchouten}\\
&& \lambda_{abc}\equiv \nabla_{e}d_{abc}{}^{e}=0 , \label{standardCEFErescaledWeyl}\\
&& Z \equiv \lambda - 6 \Xi s + 3 \nabla_{a}\Xi \nabla^{a}\Xi
\label{standardCFEconstraintFriedrichScalar}
\end{eqnarray}
\end{subequations}
where $\Xi$ is the conformal factor, $L_{ab}$ is the Schouten tensor,
defined in terms of the Ricci tensor $R_{ab}$ and the Ricci scalar $R$
via
\begin{equation}\label{SchoutenDefinition}
L_{ab}=\frac{1}{2}R_{ab}-\frac{1}{12}Rg_{ab},
\end{equation}
 $s$ is the so-called \emph{Friedrich scalar} defined as
\begin{equation}\label{s-definition}
s\equiv \tfrac{1}{4}\nabla_{a}\nabla^{a}\Xi + \tfrac{1}{24}R\Xi
\end{equation}
and $d^{a}{}_{bcd}$ denotes the \emph{rescaled Weyl tensor}, defined
as
\[d^{a}{}_{bcd}=\Xi^{-1}C^{a}{}_{bcd},\]
where $C^{a}{}_{bcd}$ denotes the Weyl tensor.  The geometric meaning
of these zero-quantities is the following: The equation $Z_{ab}=0$
encodes the conformal transformation law between ${R}_{ab}$ and
$\tilde{R}_{ab}$.  The equation $Z_{a}=0$ is obtained considering
$\nabla^{a}Z_{ab}$ and commuting covariant derivatives.  Equations
$\delta_{abc}=0$ and $\lambda_{abc}=0$ encode the contracted second
Bianchi identity. Finally, $Z=0$ is a constraint in the sense that if
it is verified at one point $p\in\mathcal{M}$ then $Z=0$ holds in
$\mathcal{M}$ by virtue of the previous equations.  A solution to the
metric conformal Einstein field equations consist of a collection of
fields
\[
\{g_{ab}, \; \Xi, \; \nabla_{a}\Xi,s\;,L_{ab},\; d_{abcd}\}
\]
satisfying

\begin{equation}\label{vanishing_CFEs_tensorial_zq}
  Z_{ab}=0, \quad Z_{a}=0, \quad \delta_{abc}=0, \quad \lambda_{abc}=0, \quad Z=0.
\end{equation}

{ \color{blue}

\begin{remark}
  \emph{ If one opts to use the Ricci tensor $R_{ab}$ instead of the Schouten tensor $L_{ab}$ then
    the Ricci scalar $R$ appears in the right-hand side of
    equations but no equation for it has been provided.
    In the CFEs the Ricci scalar encodes the \emph{conformal gauge source function}, hence
    there is no equation to fix that variable as it represents a gauge quantity of the formulation.}
\end{remark}

%% \begin{remark}
%% \emph{ In the metric formulation of the standard conformal Einstein field
%%   equations one needs to supplement the system encoded in the zero
%%   quantities defined above with an equation for the unphysical metric
%%   $g_{ab}$. To do so, one considers equation
%%   \eqref{SchoutenDefinition} expressed in some local coordinates
%%   $(x^{\mu})$. Recalling that in local coordinates the components of
%%   the Ricci tensor can be written as second order derivatives of the
%%   metric, one obtains the required equation for the unphysical metric.
%% % % This observation applies also for the subsequent
%% % % discussion of the alternative conformal Einstein field equations.
%% }
%% \end{remark}

Since the structural properties of the CFEs are better expressed in
spinorial formalism and due to the nature of the applications in this
article, the spinorial version of the CFEs will be used. The spinorial
 translation of the above CFEs zero-quantities render
---see \cite{CFEbook} for further details.

\begin{subequations}
\begin{flalign}
    Z_{AA'BB'}  = & - \Xi \Phi _{ABA'B'}  - s \epsilon _{AB} \epsilon
  _{A'B'} + \Xi \Lambda \epsilon _{AB} \epsilon _{A'B'} +
  \nabla_{BB'}\nabla_{AA'}\Xi \label{Def_ConfFactor_CFE_zeroquant}\\
  Z_{AA'}  =& \Lambda  \nabla_{AA'}\Xi  + \nabla_{AA'}s  - \Phi _{ABA'B'} \nabla^{BB'}\Xi \label{Def_s_CFE_zeroquant}\\
  %%  Z_{AA'BB'CC'}  =& \epsilon _{BC} \epsilon _{B'C'} \nabla_{AA'}\Lambda
  %% - \nabla_{AA'}\Phi _{BCB'C'}   - \epsilon _{AC} \epsilon _{A'C'} \nabla_{BB'}\Lambda
  %% \\ &   + \nabla_{BB'}\Phi _{ACA'C'}
  %% - \bar{\phi }_{A'B'C'D'} \epsilon _{AB} \nabla_{C}{}^{D'}\Xi
  %% - \phi _{ABCD} \epsilon _{A'B'} \nabla^{D}{}_{C'}\Xi\\
  \delta_{ABCC'} = &  \nabla_{A'(A}\Phi _{B)CC'}{}^{A'} - \epsilon _{C(A} \nabla_{B)C'}\Lambda   +  \phi _{ABCD} \nabla^{D}{}_{C'}\Xi \label{Def_delta_CFE_zeroquant} \\
  \Lambda _{CC'AB}  =& \nabla_{DC'}\phi _{ABC}{}^{D} \label{Def_Lambda_CFE_zeroquant}\\
    Z  =& \lambda   -6 \Xi  s + 3 \nabla_{AA'}\Xi  \nabla^{AA'}\Xi \label{Def_cons_CFE_zeroquant}
\end{flalign}
\end{subequations}
\mnotex{I think the original $Z_{AA'BB'CC'}$ does not
  contain more information than what $\delta_{ABCC'}$ encodes.   }
Similar to the tensorial case, one can choose  the
Schouten (tensor) spinor or the Ricci (tensor) spinor as a variable.
Here the equations have been expressed
using the standard curvature spinors of the NP formalism,
namely, the trace-free Ricci spinor $\Phi_{ABA'B'}$, the Ricci scalar
$\Lambda$ ---in fact $R= 24\Lambda$--- and the Weyl spinor
$\Psi_{ABCD}$ ---see \cite{Ste91, PenRin84}.
The rescaled Weyl spinor $\phi_{ABCD}$ is defined as
\begin{equation}\label{Def_rescaled_Weyl_spinor}
\phi_{ABCD} \equiv \Xi^{-1} \Psi_{ABCD}.
\end{equation}


%%%%%
%\subsection{The Conformal Einstein field equations in second order form}
%%%%

The CFEs as previously presented can be regarded as a set of covariant
conditions for geometric fields on $(\mathcal{M},\bmg)$ and, hence, they do
not have a particular PDE character.  However, there are, depending on
the gauge fixing procedure, different hyperbolic reduction strategies
to extract a set of evolution and constraint equations.
For the subsequent discussion only the evolution and constraint equations
implied by the $\Lambda_{CC'AB}=0$ equation will play a role.
A direct calculation using the space spinor formalism shows that
$\Lambda_{CC'AB}=0$ can be recasted as the following system of evolution equation and
constraint equations
\begin{align}\label{RescaledWeyl_evo_const}
  %\nabla_\tau \phi _{ABCD} = -2 \mathcal{D} _{DF}\phi _{ABC}{}^{F}
  & \nabla_\tau \phi _{ABCD} =  2 \mathcal{D} _{(A}{}^{F}\phi _{BCD)F}, % \label{RescaledWeyl_evo}\\
  %& \qquad
  \qquad \mathcal{D} _{CD}\phi _{AB}{}^{CD} = 0. 
  %\label{RescaledWeyl_const}
\end{align}
The evolution and constraint equations associated to the
other zero-quantities depend on the particular gauge fixing strategy
and will not play a relevant role for the discussion in the next sections.

\medskip

The CFEs are usually presented as the first order system
\eqref{vanishing_CFEs_tensorial_zq} with the definitions
\eqref{CFE_tensor_zeroquants}, however, for several applications it is
convenient to use a second order formulation of the equations. In
\cite{Pae13} the tensorial version of the CFEs was recasted as a set
of (tensorial) wave equations.
%%%%%
%However, as
%previously discussed several structural properties of the CFEs are
%better expressed in spinorial formalism, with this motivation,
%%%%%
Similarly, in \cite{GasVal15} a second order form of the spinorial
formulation of the CFEs was obtained.  This version of the CFEs is
particularly suited for the applications of this article, and, in
fact, only one of those equations ---that for the rescaled Weyl
spinor--- will be needed \mnotex{Double check this is true}.
%%%%%%%%%%%
%Since, only not all of the equations
%will be needed, this section presents only those which are used in the
%calculations of the subsequent sections of this paper.
%%%%%%%%%%%
The wave equation for the rescaled Weyl spinor can be succintly obtained from
considering $\nabla^{QC'}\Lambda _{CC'AB}$.  A direct calculation using the identity
\eqref{DecomposeDoubleDerivativeContracted} shows that if $\Lambda _{CC'AB}=0$ then,
\begin{eqnarray}
  \square \phi _{ABCF} = 12 \Lambda  \phi _{ABCF}  -6 \Xi  \phi _{(AB}{}^{DG}\phi _{CF)DG}
  \label{Wave_eq_CFE_Weyl}
\end{eqnarray}
A similar calculation can be carried out for the other equations in comprising the CFEs.
A full discussion of the \emph{spinorial CFE wave equations} and their equivalence
with the standard first order formulation CFEs can be found in \cite{GasVal15}.
%
%\medskip
%
One of the tools used in \cite{GasVal15} to show the
equivalence between these two set of equations is the uniqueness
property to a certain class of wave equations. This same result from
the theory of partial differential equations will be used to obtain
the main theorem of this article and is presented in the following
%%%%%%%
%One of the main tools used in the latter reference
%to show the equivalence between these two set of equations
%is the following result from the theory of partial differential equations which
%will be also used to obtain the main result of this article.
%%%%%
%This theorem will also be used in for de
% %The main tool from the theory of partial differential equations
%that will be used in the next sections is that of the uniqueness property for
%solutions to a certain class of wave equations.
%
%This theorem will be also used in the proof of the main result of this article.
%%%%%

}

\mnotex{Technical pde theorem moved here}
\begin{theorem}
\label{TheoremHomogeneousWave}
 Let $\mathcal{M}$ be a smooth manifold equipped with a 
Lorentzian metric $\bmg$ and consider the wave equation
\[\square \underline{u}=h\left(\underline{u},~\partial\underline{u}\right)\]
where $\underline{u}\in\mathbb{C}^m$ is a complex vector-valued function
on $\mathcal{M}$, 
$h:\mathbb{C}^{2m}\rightarrow\mathbb{C}^m$ is a smooth homogeneous
function of its arguments and $\square=g^{ab}\nabla_{a}\nabla_{b}$.
Let $\mathcal{U}\subset\mathcal{S}$ be an open set and $\mathcal{S}\subset \mathcal{M}$ be a spacelike hypersurface with normal
$\tau^{a}$ respect to $\bmg$. Then the Cauchy problem
\begin{align*}
\square
\underline{u}&=h\left(\underline{u},~\partial\underline{u}\right),\\ \underline{u}\left|_{\mathcal{U}}\right.&=\underline{u}_0,
\quad
\mathcal{P}\underline{u}\left|_{\mathcal{U}}\right.=\underline{u}_1,
\end{align*} 
where $\underline{u}_{0}$ and $\underline{u}_{1}$  are 
smooth on $\mathcal{U}$ and $\mathcal{P}\equiv \tau^\mu\nabla_\mu$,
 has a unique solution $\underline{u}$ in the domain of dependence of $\mathcal{U}$.
\end{theorem}
We refer the reader to
\cite{CFEbook,Tay96c} for a proof ---see also Theorem 1 in \cite{GarVal08c}.  

\begin{remark}
\emph{ Recall that an equation of the
above form are said to be \textit{homogeneous in} $\underline{u}$
\textit{and its first derivatives} if
$h\left(\lambda\underline{u},~\lambda\partial\underline{u}\right)=\lambda
h\left(\underline{u},~\partial\underline{u}\right)$ for all
$\lambda\in\mathbb{C}$. }
\end{remark}



%% \subsection{The alternative conformal Einstein field equations}
%% \label{AltCFEs}
%% %
%% %
%% In the current formulations of the conformal Einstein field equations
%% the rescaled Weyl tensor $d^{a}{}_{bcd}$ plays a central role in the
%% discussion.  Nevertheless in \cite{Pae13} an alternative approach was 
%% outlined, whereby the central object of interest is the Weyl tensor 
%% itself. In doing so, one must also introduce the Cotton tensor as an 
%% unknown. In \cite{Pae13} a set of wave equations for these unknowns is 
%% constructed. Here we follow a similar approach, but rather than 
%% deriving second-order equations for the conformal fields, we will obtain 
%% a closed system of equations which are (apart from the equation for the
%% conformal factor, \eqref{CEFEsecondderivativeCFAlt}) of first-order. We 
%% will call the resulting equations, along with their
%% spinorial equivalent, the \emph{alternative CFEs}.   


%% \medskip

%% Considering $\nabla^{a}\delta_{abc}=0$, with $\delta_{abc}$ as given
%% in in expression \eqref{standardCEFESchouten}, one obtains the
%% following wave equation for the Schouten tensor
%% \begin{equation}\label{WaveEqSchoutenForAlternativeCEFE}
%% \square L_{bc} = 4 L_{b}{}^{a} L_{ca} - L_{ad} L^{ad} g_{bc} - 2
%% L^{ad} C_{bacd} + \tfrac{1}{6} \nabla_{c}\nabla_{b}R.
%% \end{equation}
%% Recalling the definition of the Cotton tensor in terms of the Schouten
%% tensor
%% \begin{equation}\label{DefCotton}
%% Y_{abc} = 2 (- \nabla_{a}L_{bc} + \nabla_{b}L_{ac})
%% \end{equation}
%%  and using equations \eqref{WaveEqSchoutenForAlternativeCEFE} and
%%  \eqref{DefCotton} a computation shows that
%% \begin{equation} \label{DivY}
%% \nabla_{a}Y_{b}{}^{a}{}_{c} = -2 L^{ad} C_{bacd}.
%% \end{equation}
%% Therefore, to close the system one needs to find an equation for the
%% Weyl tensor. To do so, one can use the second Bianchi identity
%% \[
%% \nabla_{[a}R_{bf]c}{}^{d} = 0
%% \]
%% and the decomposition of the Riemann tensor in terms of the Weyl and
%% Schouten tensors
%% \[
%% R_{abc}{}^{d} = \delta_{b}{}^{d} L_{ac} - \delta_{a}{}^{d} L_{bc} +
%% L_{b}{}^{d} g_{ac} - L_{a}{}^{d} g_{bc} + C_{abc}{}^{d},
%% \]
%%  to obtain
%% \begin{equation}\label{SecondBianchiTemporaryBetter}
%%  \nabla_{[a}C_{bf]c}{}^{d} - 2 g_{[a|c|}\nabla_{b}L_{f]}{}^{d} + 2
%%  g_{[a}{}^{d}\nabla_{b}L_{f]c}=0.
%% \end{equation}
%% Using equations \eqref{SecondBianchiTemporaryBetter} and
%% \eqref{DefCotton} one can rewrite equation
%% \eqref{SecondBianchiTemporaryBetter} as
%% \begin{equation}\label{SecondBianchiForAlt}
%%  \nabla_{[a}C_{bf]c}{}^{d} = \tfrac{1}{2}Y_{[ab|c|}g_{f]}{}^{d} -
%%  \tfrac{1}{2} Y_{[ab}{}^{d}g_{f]c}.
%% \end{equation}
%% Consequently, one can replace the zero-quantities associated with the
%% rescaled Weyl tensor using equations \eqref{SecondBianchiForAlt},
%% \eqref{DefCotton} and \eqref{DivY}, obtaining then an alternative
%% version of the conformal Einstein field equations. The equations
%% so obtained are encoded in the vanishing of the following zero-quantities.
%% \begin{subequations}
%% \begin{eqnarray}
%% && Z_{ab} \equiv \nabla_{a}\nabla_{b}\Xi +\Xi L_{ab} - s g_{ab}=0 ,
%%  \label{CEFEsecondderivativeCFAlt}\\
%% && Z_{a} \equiv \nabla_{a}s +L_{ac} \nabla ^{c}\Xi=0
%%  , \label{CEFEsAlt}\\ && Z \equiv \lambda - 6 \Xi s + 3 \nabla_{a}\Xi
%%  \nabla^{a}\Xi,
%% \label{standardCFEconstraintFriedrichScalarAlt}\\
%% && \Delta_{bac} \equiv
%% \nabla_{b}L_{ac}-\nabla_{a}L_{bc}-\frac{1}{2}Y_{abc}, \label{CEFESchouten}
%% \\ && \Pi_{bc} \equiv \nabla_{a}Y_{b}{}^{a}{}_{c}+2
%% L^{ad}C_{bacd}, \label{CEFECotton}\\ && \Lambda_{abcd}{}^{e}=
%% 2\nabla_{[a}C_{bc]d}{}^{e} - Y_{[ab|d|}g_{c]}{}^{e} +
%% Y_{[ab}{}^{e}g_{c]d}
%% \label{CEFELambda}
%% \end{eqnarray}
%% \end{subequations}
%% Notice that in this alternative representation of the conformal
%% Einstein field equations the rescaled Weyl tensor does not
%% appear. Instead, the Weyl tensor $C_{abcd}$ and the Cotton tensor
%% $Y_{abc}$ are now part of the unknowns. Observe that the definition of
%% the Cotton tensor in terms of derivatives of the Schouten tensor is
%% encoded in equation \eqref{CEFESchouten}.  A solution to the alternative
%% conformal Einstein field equations consists of a collection of fields
%% \begin{equation}\label{UnknownsCFE}
%% \{g_{ab}, \; \Xi,\;\nabla_{a}\Xi \;,s\;,L_{ab},\; C_{abcd},\; Y_{abc}\}
%% \end{equation}
%% satisfying
%% \[ Z_{ab}=0, \quad Z_{a}=0, \quad \Delta_{abc}=0, \quad \Pi_{bc}=0,\quad \Lambda_{abcde}=0.   \]
%% \begin{remark}\emph{
%% Note that, by construction, any solution to the (alternative) CFEs
%% with $\Xi= 1$ corresponds to a solution of the Einstein field
%% equations. Conversely, given a solution to the Einstein field
%% equations, there corresponds a family of conformally-related solutions
%% to the (alternative) CFEs.}
%% \end{remark}

%% In view of the subsequent analysis of the Killing spinor equation it
%% is convenient to formulate the above system in spinorial form. Similar
%% to the case of the standard conformal Einstein field equations, the
%% spinorial formulation allows one to identify in a clearer way the
%% structure of the equations.  To obtain the spinorial formulation of
%% the the zero-quantities
%% \eqref{CEFEsecondderivativeCFAlt}-\eqref{CEFELambda} recall that the
%% spinorial counterpart of the Weyl tensor can be decomposed as
%% \[
%% C_{AA'BB'CC'DD'}=\bar{\Psi}_{A'B'C'D'} \epsilon_{AB} \epsilon_{CD} +
%% \Psi_{ABCD} \bar{\epsilon}_{A'B'} \bar{\epsilon}_{C'D'},
%% \]
%% where $\Psi_{ABCD}$ is the Weyl spinor.  Similarly, the irreducible
%% decomposition of the Cotton spinor $Y_{AA'BB'CC'}$ given by
%% \begin{equation}
%% Y_{AA'BB'CC'} = \bar{Y}_{A'B'C'C} \epsilon_{AB} + Y_{ABCC'}
%% \bar{\epsilon}_{A'B'},
%% \end{equation}
%% where
%% \begin{equation}
%% Y_{ABCC'} = \tfrac{1}{2} Y_{(A}{}_{|Q'|B}{}^{Q'}{}_{C)C'}.
%% \end{equation}
%% Additionally, the Schouten spinor $L_{AA'BB'}$ can be expressed in
%% terms of the tracefree Ricci spinor $\Phi_{AA'BB'}$ and
%% $\Lambda=\frac{1}{24}R$;
%% \begin{equation}
%% L_{AA'BB'}=-\Phi_{ABA'B'}+ \Lambda \epsilon_{AB}\epsilon_{A'B'} .
%% \end{equation} 
%% With these decompositions at hand, the spinorial formulation of the
%% above equations can be expressed as
%% \begin{equation}\label{CFEZeroQuantitiesEqualToZero}
%% Z_{AA'BB'}=0, \quad Z_{AA'}=0, \quad \Delta_{ABCC'}=0, \quad
%% \Pi_{BB'CC'}, \quad \Lambda_{C'BCF}=0,
%% \end{equation}
%% where
%% \begin{subequations}
%% \begin{flalign}
%% & Z_{AA'BB'} \equiv \nabla_{BB'}\nabla_{AA'}\Xi - \Xi
%%   \Phi_{ABA'B'}\epsilon_{A'B'} + \Xi \Lambda
%%   \epsilon_{AB}\epsilon_{A'B'} - s \epsilon_{AB}\epsilon_{A'B'},
%% \label{ZeroQuantitySecondDerivativeConformalFactor}  \\
%% & Z_{AA'} \equiv \nabla_{AA'}s - \Phi_{AA'}{}^{BB'} \nabla_{BB'}\Xi +
%% \Lambda\nabla_{AA'}\Xi, \label{ZeroQuantityDerivativeFriedrichScalar}
%% \\ &\Delta_{ABCC'} \equiv - Y_{ABCC'} + \nabla_{AA'}\Phi_{BCC'}{}^{A'} +
%% \epsilon_{BC} \nabla_{AC'}\Lambda + \nabla_{BA'}\Phi_{ACC'}{}^{A'} +
%% \epsilon_{AC} \nabla_{BC'}\Lambda,
%% \label{ZeroQuantitySchouten}\\
%% & \Pi_{AA'BB'} \equiv -2 \Phi^{CD}{}_{A'B'} \Psi_{ABCD} + -2
%% \Phi_{AB}{}^{C'D'} \overline{\Psi}_{A'B'C'D'} +
%% \nabla_{CC'}Y_{AA'}{}^{CC'}{}_{BB'} ,\label{ZeroQuantityCotton}
%% \\ &\Lambda_{C'BCF}
%% \equiv - \tfrac{1}{2} Y_{BCFC'} + \nabla_{AC'}\Psi_{BCF}{}^{A}.
%% \label{ZeroQuantityWeyl}
%% \end{flalign}
%% \end{subequations}
%% Notice that the zero-quantity $\Pi_{AA'BB'}$ can alternatively be
%% written in terms of the reduced Cotton spinor $ Y_{ABCC'}$ as
%% \begin{equation*}
%% \Pi_{AA'BB'} = -2 \Phi^{CD}{}_{A'B'} \Psi_{ABCD} - 2
%% \Phi_{AB}{}^{C'D'} \bar{\Psi}_{A'B'C'D'} +
%% \nabla_{AC'}\bar{Y}_{A'}{}^{C'}{}_{B'B} +
%% \nabla_{CA'}Y_{A}{}^{C}{}_{BB'}.
%% \end{equation*}
%% Furthermore, observe that a trace of the latter equation implies
%% \begin{equation*}
%%  \Pi^{A}{}_{A'AB'}= -\nabla_{AC'}\bar{Y}_{A'B'}{}^{C'A} .
%% \end{equation*}
%% Its worth noticing that in the formulation of the conformal Einstein
%% field equations, the Ricci scalar $\Lambda$ is not part of the
%% unknowns as it represents, the so-called, the \emph{conformal gauge
%%   source function} ---see \cite{Fri83, Fri91, CFEbook} for further
%% discussion.  The geometric meaning of the zero-quantities
%% \eqref{ZeroQuantitySecondDerivativeConformalFactor}-\eqref{ZeroQuantityWeyl}
%% is analogous to their tensorial counterparts.  In particular, the
%% Bianchi identities may be recovered from the alternative conformal
%% Einstein field equations by taking suitable contractions of the
%% zero-quantities $\Delta_{ABCC'}$ and $\Lambda_{ABB'C}$:
%% \begin{subequations}
%% \begin{eqnarray}
%% && \nabla_{AA'}\Phi_{BC'}{}^{AA'}  + 3 \nabla_{BC'}\Lambda =-\Delta_{B}{}^{A}{}_{AC'}
%% \label{BianchiIdentityRicciZeroQuantities} \\
%% && \nabla_{AC'}\Psi_{BCF}{}^{A} + \nabla_{(B}{}^{Q'}\Phi_{CF)C'Q'} = - \tfrac{1}{2} \Delta_{(BCF)C'} + \Lambda_{C'(BCF)}
%% \label{BianchiIdentityWeylZeroQuantities}
%% \end{eqnarray}
%% \end{subequations}


%% By \emph{initial data for the alternative CFEs} we mean the restriction to an
%% hypersurface $\mathcal{S}\subset \mathcal{M}$ of a collection of
%% fields \eqref{UnknownsCFE}, satisfying the constraint equations
%% implied by (\ref{CFEZeroQuantitiesEqualToZero}).
%% It will not be necessary for our purposes to study the constraints
%% equations in detail; indeed, the only
%% constraint that will be of interest ---see Section
%% \ref{Sec:IntrinsicConditions}--- is the following
%% \[\mathcal{D}^{PQ}\Psi_{ABPQ}-\tfrac{1}{2}Y_{AB}{}^{Q}{}_{Q}=0\]
%% which follows from $\Lambda_{A'ABC}=0$.  Since the (alternative)
%% conformal Einstein field equations imply a solution to the Einstein
%% field equations \eqref{EFEVacuum} whenever $\Xi\neq 0$, we will refer
%% to the development such an initial data set simply as a
%% \emph{spacetime development}.


{\color{blue}
  %\section{Conformally-Einstein vacuum twistor initial data}
  \section{Conformal twistor initial data}
  \label{conformalTwistorKID}
  In this section, the conformal twistor initial data equations are
  derived.  Although the main result of this article is on the
  conformal valence-2 Killing spinor initial data equations, the twistor
  case discussed in this section serves as a test case where the main
  features of the calculation of the next section can be understood in simpler setting.
\subsection{Twistor zero-quantities}
\label{Sec:TwistorZeroQuantities}

For the following discussion is convenient to make the following
\emph{zero-quantities}
\begin{subequations}
  \begin{eqnarray}
   && H_{A'AB} \equiv 2
    \nabla_{A'(A}\kappa_{B)},\label{Def_H_twistor}\\ && B_{ABC}
    \equiv \phi_{ABCD}\kappa^D.\label{Def_B_twistor}
    \end{eqnarray}
\end{subequations}
The spinors $H_{A'AB}$ and $B_{ABC}$ will be denoted in index free
notation as $\bmH$ and $\bmB$ and will be called the twistor
zero-quantity and the Buchdahl zero-quantity respectively.  The
Buchdahl zero-quantity arises as an integrability condition of the
twistor equation.  To see this, notice that, taking the following
derivative of $\bmH$ and substituting the definition
\eqref{Def_H_twistor} one obtains
  \begin{equation}\label{curl_H_twistor}
  \nabla_{AA'}H^{A'}{}_{BC}= 2 \nabla_{AA'}\nabla_{(B}{}^{A'}\kappa
  _{C)} = \tfrac{1}{2} \epsilon _{AB} \square \kappa _{C}  +
  \tfrac{1}{2}  \epsilon _{AC} \square \kappa _{B} +
  \square_{BA}\kappa _{C} + \square_{CA}\kappa _{B}.
  \end{equation}
  Symmetrsing and using equation \eqref{SpinorialRicciIdentities1} renders
  \[
  \nabla_{(A|A'|}H^{A'}{}_{BC)}= - 2\Psi_{ABCD}\kappa^D.
  \]
  The vanishing of the right-hand side of latter equation encodes the
  Buchdahl constraint, namely the fact that if $(\mathcal{M},\bmg)$
  admits a twistor then it is necesarilly of Petrov type N or O. To
  write this in the variables appearing in the conformal Einstein
  field equations, using the definition of the rescaled Weyl spinor
  yields
  \begin{equation}\label{Curl_H_sym_toB_twistor}
  \nabla_{(A}{}^{A'}H_{|A'|BC)} = 2\Xi B_{ABC},
  \end{equation}
  which motivates the name for the zero-quantity $\bmB$.
  Thus, with this notation, it is clear that if the unphysical spacetime
  $(\mathcal{M},\bmg)$ admits a twistor (valence-1 Killing spinor) then following
  zero-quantities vanish
  \begin{equation}
H_{A'AB}=0, \qquad B_{ABC}=0.
  \end{equation}
  \subsection{Twistor auxiliary quantities and the twistor candidate equation}
A useful bookkeeping device for the subsequent
calculations are the definitions of following \emph{auxiliary quantities}:
\begin{subequations}
  \begin{eqnarray}
      && Q_{A}  \equiv \nabla^{QA'}H_{A'QA} \label{def_Q_twistor} \\
      && \xi_{A'} \equiv \nabla^B{}_{A'}\kappa_B \label{def_xi_twistor}
  \end{eqnarray}
\end{subequations}
  The \emph{auxiliary spinor} $\xi_{A'}$ is merely a convenient
  placeholder for making irreducible decompositions of derivatives of
  $\kappa_A$ such as
  \begin{align}\label{decomp_Der_kappa}
    \nabla_{AA'}\kappa _{B} & = \tfrac{1}{2} \epsilon _{AB}
    \nabla_{CA'}\kappa ^{C} + \nabla_{(A|A'|}\kappa _{B)}
 \\ & = \tfrac{1}{2} H_{A'AB} - \tfrac{1}{2} \xi _{A'} \epsilon_{AB}.
  \end{align}
  and in principle one can carry out all the calculations without this
  definition. It is nevertheless illustrative to introduce this
  shorthand since the analogous quantity in the Killing spinor case
  ($p=2$) will have some geometrical significance.

  \medskip
  
  On the other hand, the \emph{auxiliary quantity}
  $Q_A$ will be central for the following discussion since it
  encodes a wave equation for $\kappa_A$. To see this, observe that
  tracing the identity \eqref{curl_H_twistor} and substituting the
  definition \eqref{def_Q_twistor} gives,
\begin{equation}\label{Q_to_box_twistor_candidate}
Q_{A} = 3 \Lambda \kappa _{A} + \tfrac{3}{2} \square \kappa _{A}.
\end{equation}
Solving for $\square \kappa _{A}$ one has
\[
\square \kappa _{A} = \tfrac{2}{3} Q_{A} -2 \Lambda \kappa _{A}.
\]
If the equation $Q_{A}=0$ is imposed, then
the latter expression can be read as a wave equation for $\kappa_A$.
This motivates the following definition: a valence-1 spinor $\eta_A$ satisfying
\begin{align} \label{Wave_eq_twistor_candidate}
\square \eta _{A} = -2 \Lambda  \eta _{A}
\end{align}
will be called a \emph{twistor candidate}. To understand the
motivation for this definition and its name, notice that in general,
any twistor $\kappa_A$ trivially satisfies the twistor candiadate
equation but not every twistor candidate $\eta_A$ will solve the
twistor equation. In other words,
\[
\bmH=0 \implies \bmQ =0, \qquad \text{but in general} \qquad \bmQ =0 \notimplies \bmH=0.
\]
However,  the initial data $(\nabla_\tau
\eta_A, \eta_A)|_{\mathcal{S}}$ for the wave equation
\eqref{Wave_eq_twistor_candidate} has not been fixed yet. The aim of the following
calculations is to determine the conditions on the initial data for the
twistor candidate such that if propagated off $\mathcal{S}$, using
equation \eqref{Wave_eq_twistor_candidate}, then the corresponding twistor
candidate $\eta_A$ is, in fact, a twistor. Namely,
\begin{equation}
\bmQ =0 \;\;\&\;\; \text{twistor initial data} \;\;\implies \bmH=0.
\end{equation}
The strategy to obtain such conditions on the intial data
$(\nabla_\tau \eta_A, \eta_A)|_{\mathcal{S}}$  is to derive a closed
system of homogeneous wave equations for the zero-quantities $\bmH$
and $\bmB$ to show that, if trivial initial data for such
equations is given, then, using Theorem \ref{TheoremHomogeneousWave},
$\bmH=0$ and $\bmB=0$ in the domain of dependence of the data.

\subsection{Wave equations for the zero-quantities}

A wave equation for the zero-quantity $\bmH$ can be constructed as
follows.  From the irreducible decomposition of
$\nabla_D{}^{A'}H_{A'AB}$,
\[
\nabla_{D}{}^{A'}H_{A'AB} = \tfrac{1}{3} \epsilon _{BD}
\nabla_{CA'}H^{A'}{}_{A}{}^{C} + \tfrac{1}{3} \epsilon _{AD}
\nabla_{CA'}H^{A'}{}_{B}{}^{C} + \nabla_{(A}{}^{A'}H_{|A'|BD)},
\]
and the definitions \eqref{def_Q_twistor} and equation
\eqref{Curl_H_sym_toB_twistor} one has that
\begin{align}\label{derH_twistor_toBandQ}
\nabla_{D}{}^{A'}H_{A'AB} = 2 B_{ABD} \Xi + \tfrac{1}{3} Q_{B}
\epsilon _{AD} + \tfrac{1}{3} Q_{A} \epsilon _{BD}
\end{align}
Applying $\nabla_{D}{}^{B'}$ to the last expression, and using the
identity \eqref{DecomposeDoubleDerivativeContracted} along with the
spinorial Ricci identities
\eqref{SpinorialRicciIdentities1}-\eqref{SpinorialRicciIdentities2},
renders
\begin{equation}\label{wave_H_twistor}
  \square H_{B'AB} = 6 \Lambda H_{B'AB} + 4 \Xi
  \nabla_{DB'}B_{AB}{}^{D} -4 B_{ABD} \nabla^{D}{}_{B'}\Xi -4
  \Phi_{(A}{}^{D}{}_{|B'}{}^{A'}H_{A'|B)D} + \tfrac{4}{3}
  \nabla_{(A|B'|}Q_{B)}
\end{equation}

\noindent To derive a wave equation for $\bmB$, one applies the D'Alembertian operator
$\square$ to the definition in equation \eqref{Def_B_twistor} to obtain
\begin{align}\label{pre_wave_B_twistor}
\square B_{ABC} = \kappa ^{D} \square \phi _{ABCD} + \phi _{ABCD}
\square \kappa ^{D} + 2 \nabla_{FA'}\phi _{ABCD} \nabla^{FA'}\kappa
^{D}
\end{align}
Substituting the definition \eqref{Def_B_twistor}, the identity
\eqref{Q_to_box_twistor_candidate}, and the wave equation satisfied by
the rescaled Weyl spinor \eqref{Wave_eq_CFE_Weyl} into the last expression gives
\begin{equation}\label{wave_B_twistor}
\square B_{ABC} = 10 B_{ABC} \Lambda + H^{A'DF} \nabla_{FA'}\phi _{ABCD}  -6 \Xi B_{(A}{}^{DF}\phi
_{BC)DF} + \tfrac{2}{3} \phi _{ABCD} Q^{D}
\end{equation}
Observe that if $Q_{A}=0$, namely if the twistor candidate wave equation is imposed then,
$\bmH$ and $\bmB$ satisfy the following set of wave equations
\begin{subequations}
\begin{eqnarray}
  && \square H_{B'AB} = 6 \Lambda H_{B'AB} + 4 \Xi
  \nabla_{DB'}B_{AB}{}^{D}  -4 B_{ABD} \nabla^{D}{}_{B'}\Xi   -4 \Phi_{(A}{}^{D}{}_{|B'}{}^{A'}H_{A'|B)D}
   \label{Hom_wave_HandB1} \\
 && \square B_{ABC} = 10 B_{ABC} \Lambda + H^{A'DF} \nabla_{FA'}\phi _{ABCD}  -6 \Xi B_{(A}{}^{DF}\phi
_{BC)DF}  \label{Hom_wave_HandB2}
\end{eqnarray}
\end{subequations}
Notice that the only place where the CFEs ---in their wave equation form---
have been used was in equation \eqref{pre_wave_B_twistor}
to substitute the term $\square \phi _{ABCD}$.

The relevant observation about equations
\eqref{Hom_wave_HandB1}-\eqref{Hom_wave_HandB2} is that they
constitute a closed system of \emph{regular and homogeneous}
wave equations for $\bmH$ and $\bmB$.
Hence prescribing trivial intial data
\[
H_{A'AB}=0, \qquad \nabla_\tau H_{A'AB}=0, \qquad B_{ABC}=0, \qquad \nabla_\tau B_{ABC}=0 \qquad \text{on} \qquad \mathcal{S}
\]
and using Theorem \ref{TheoremHomogeneousWave}, which stablishes the
uniquess of solutions to wave equations of the type
of \eqref{Hom_wave_HandB1}-\eqref{Hom_wave_HandB2}, on has that 
\begin{equation}\label{ID_trivial_H_B_twistor}
H_{A'AB}=0, \qquad B_{ABC}=0 \qquad \text{on} \qquad \mathcal{D}^{+}(\mathcal{S}) .
\end{equation}
In turn, substituting the definitions for the zero-quantities $\bmH$
and $\bmB$ into the conditions \eqref{ID_trivial_H_B_twistor} render a
prescription to fix the intial data $(\nabla_\tau \eta_A,
\eta_A)|_\mathcal{S}$ for  twistor
candidate wave equation \eqref{Wave_eq_twistor_candidate} that ensures that
 the twistor candidate $\eta_A$ will
corresponds to an actual twistor $\kappa_A$ in $\mathcal{D}^{+}(\mathcal{S})$.

This discussion is summarised in the following
\begin{proposition}\label{Prop:Propagation_twistor}
  Given initial data for the conformal field equations on $\mathcal{U}\subseteq\mathcal{S}$
  where $\mathcal{S}$ is a spacelike
hypersurface $\mathcal{S}$ with normal vector $\tau^{AA'}$, and
associated normal derivative $\nabla_\tau \equiv
\tau^{AA'}\nabla_{AA'}$, the corresponding spacetime development
admits a twistor (valence-1 Killing spinor) in $\mathcal{D}^{+}(\mathcal{U})$ ---the future domain of dependence of $\mathcal{U}$---  if and only if
\begin{subequations}
\begin{eqnarray}
  && H_{A'AB}=0,\label{eq:VanishingOfH_twistor}\\ && \nabla_\tau
  H_{A'AB}=0,\label{eq:VanishingOfNormalDerivH_twistor}\\ &&
  B_{ABC}=0,\label{eq:VanishingOfB_twistor}\\ &&\nabla_\tau
  B_{ABC}=0, \label{eq:VanishingOfNormalDerivB_twistor}
\end{eqnarray}
\end{subequations}
 hold on $\mathcal{U}$.
\end{proposition}
\begin{proof}
The \emph{only if} direction is immediate. Suppose, on the other hand,
that
\eqref{eq:VanishingOfH_twistor}-\eqref{eq:VanishingOfNormalDerivB_twistor}
hold on some $\mathcal{U}\subset\mathcal{S}$ ---that is to say, there
exist a spinor field $\kappa_{A}$ for which
\eqref{eq:VanishingOfH_twistor}-\eqref{eq:VanishingOfNormalDerivB_twistor}
are satisfied on $\mathcal{U}$. The latter is then used as initial
data for the twistor candidate wave equation
\begin{align} \label{Wave_eq_twistor_candidate_prop}
\square \kappa _{A} = -2 \Lambda \kappa _{A}.
\end{align}
As the zero-quantities $H_{A'AB},~B_{ABC}$ satisfy the homogeneous
wave equations \eqref{Hom_wave_HandB1}-\eqref{Hom_wave_HandB2} then
the uniqueness result for homogeneous wave equations, given in
Theorem \ref{TheoremHomogeneousWave},
ensures that
\[ H_{A'ABC}=0,\qquad B_{ABC}=0,\]
in $\mathcal{D}^{+}(\mathcal{U})$. In other words,
$\kappa_{A}$ solves the twistor equation on $\mathcal{D}^{+}(\mathcal{U})$.
\end{proof}


\subsection{Comparisson between the Killing initial data conditions
  in the physical and unphysical pictures}

The main advantage of the conformal (unphysical) approach to the
Einstein field equations is that the conformal boundary $\mathscr{I}$
determined by $\Xi=0$ is a submanifold of $(\mathcal{M},\bmg)$. This
allows, in particular, to consider the $\Xi=0$ hypersurface as a
legitimate hypersurface to prescribe data which can be evolved using
regular ---without $\Xi^{-1}$-terms--- evolution equations.  This set up is
particularly atractive to study spacetimes with $\lambda>0$ in which
---given the appropriate conditions--- the conformal boundary $\mathscr{I}$
is a spacelike hypersurface and hence one can pose \emph{an asymptotic
initial value problem}: an initial value problem where the initial
hypersurface is $\mathscr{I}$.
On the other hand, the conformal (valence-1 Killing spinor) twistor conditions
of proposition \eqref{Prop:Propagation_twistor} allows to identify
asymptotic initial data  whose development will
contain a twistor. Although the twistor case is too restrictive to
characterise black hole spacetimes
%%%%
% (since the Buchdahl constraint forces the manifold $(\bmg, \mathcal{M})$
% to be of Petrov type N orO),
%%%
it is still illustrative to compare the derivation of the physical
twistor initial data conditions on
$(\tilde{\mathcal{M}},\tilde{\bmg})$ and that leading to proposition
\eqref{Prop:Propagation_twistor}.

For the twistor case, one important difference between discussion in
\cite{GasVal15} using the vacuum Einstein field equations in  $(\mathcal{M},\bmg)$
is that the system closes with $\tilde{H}_{A'AB}$ alone and there is no need
to introduce the analogous physical Buchdahl zero-quantity $\tilde{B}_{ABC}$.
Therefore it is interesting to check if in the conformal case discussed
in the previous sections one can also close the system with $H_{A'AB}$ alone.

Applying the D'Alembertian $\square$ to equation
\eqref{Def_H_twistor}, using the definition of the auxiliary quantity
$Q_A$ in equation \eqref{def_Q_twistor}, a direct calculation exploiting the
identities \eqref{SpinorialRicciIdentities1}-\eqref{SpinorialRicciIdentities2}
and \eqref{DecomposeDoubleDerivativeContracted}, gives

\begin{align}\label{Box_H_alone}
\square H_{A'AB} = & -2 \Psi _{ABCD} H_{A'}{}^{CD} + 6 \Lambda
H_{A'AB} -4 \Phi _{(A}{}^{C}{}_{|A'}{}^{B'}H_{B'|B)C} \nonumber \\ &
-2 \kappa _{(A}\nabla_{B)A'}\Lambda -2
\kappa^{C}\nabla_{(A}{}^{B'}\Phi _{B)CA'B'} + 2 \kappa ^{C}
\nabla_{DA'}\Psi _{ABC}{}^{D} + \tfrac{4}{3} \nabla_{(A|A'|}Q_{B)}.
\end{align}
If one were discussing the physical case ---adding a tilde to every
term in \eqref{Box_H_alone}--- in which the fields are defined on
$(\tilde{\mathcal{M}},\tilde{\bmg})$ which satisfies the vacuum
Einstein field equations
\begin{align}\label{vaccumEFE}
  \tilde{\Lambda}=0, \qquad \tilde{\Phi}_{AA'BB'}=0,
\end{align}
then, using the Bianchi identity $\tilde{\nabla}^{A}{}_{B'}\Psi
_{ABCD}=\tilde{\nabla}^{A'}{}_{(B}\tilde{\Phi}_{CD)A'B'}$ and
equations \eqref{vaccumEFE},
% it follows
%from the physical version of
%equation \eqref{Box_H_alone}
the physical version of equation \eqref{Box_H_alone} ---formally adding a tilde---
reduces to
\begin{align}\label{Box_H_alone_physical}
  \tilde{\square} \tilde{H}_{A'AB} = & -2 \Psi _{ABCD} \tilde{H}_{A'}{}^{CD}
  + \tfrac{4}{3} \tilde{\nabla}_{(A|A'|}\tilde{Q}_{B)}.
\end{align}
Hence, imposing $\tilde{Q}_A=0$, the system
closes with $\tilde{H}_{A'AB}$ alone.


\medskip

In the other hand, if one tries to follow the same strategy in the unphysical set up
---with $(\mathcal{M},\bmg)$ satisfying the CFEs--- one ends up with a
formally singular equation. To see this, observe that starting from
the identity \eqref{Box_H_alone} and using equation \eqref{Def_rescaled_Weyl_spinor}
along the CFEs zero-quantities
\[ \delta_{ABCC'}=0, \qquad \Lambda_{CC'AB}=0,\]
as defined in equations \eqref{Def_delta_CFE_zeroquant}-\eqref{Def_Lambda_CFE_zeroquant},
a calculation gives

\begin{align}\label{WaveH_twistor_singular}
  \square H_{A'AB} = & - \frac{2 \nabla^{C}{}_{A'}\Xi
    \nabla_{(A}{}^{B'}H_{|B'|BC)}}{\Xi } + 6 \Lambda H_{A'AB} -2 \Xi
  \phi _{ABCD}H_{A'}{}^{CD} -4
  \Phi_{(A}{}^{C}{}_{|A'}{}^{B'}H_{B'|B)C} \nonumber \\ & +
  \tfrac{4}{3} \nabla_{(A|A'|}Q_{B)}.
\end{align}

%\noindent
Hence, setting $Q_{A}=0$, renders an homogeneous but
\emph{singular equation} ---due to the $\Xi^{-1}$ coefficient---
for $H_{A'AB}$, for which the theory of behind Theorem
\ref{TheoremHomogeneousWave} does not apply. Arguably, one could try
to use the theory of Fuchsian systems to see if the analogous of
Theorem \ref{TheoremHomogeneousWave} applies for the singular equation
\eqref{WaveH_twistor_singular}.  However, one of the advantages of the
conformal approach of the CFEs respect to other approaches to include
$\mathscr{I}$ is that one deals with formally regular equations.
Therefore, from this perspective,
%%%%%%
%, in the spirit of the CFEs,
%%%%%
it is preferable to work with explicitly regular equations and hence,
it is necessary to introduce $B_{ABC}$ as a further zero-quantity to
be propagated.  A analogous observation holds for the conformal valence-2
Killing spinor initial data discussion of the following sections,
where, to close the system in a regular way, one needs to introduce not
only the Buchdahl zero-quantity but also its derivative.

%% 
%% Making formally identical definition as that of \eqref{Def_H_twistor}
%% to define the physical twistor zero-quantity $\tilde{H}_{A'AB}$ and applying $\square$
%% one obtains

%% \[
%% \square H_{A'AB} =-2 \square_{(A}{}^{C}\nabla_{|CA'|}\kappa _{B)}  -2 \square_{A'}{}^{B'}\nabla_{(A|B'|}\kappa _{B)} + 2 \nabla^{C}{}_{A'}\square_{(A|C|}\kappa _{B)} + 2 \nabla_{(A}{}^{B'}\square_{|A'B'|}\kappa _{B)}
%% + 2 \nabla_{(A|A'|}\square \kappa_{B)}
%% \]
%% and the physical auxiliary quantity $\tilde{Q}_A$



\subsection{Intrinsic conformal twistor initial data conditions}

In this section, the conformal twistor initial data conditions of
proposition \ref{Prop:Propagation_twistor} are written in terms of
intrinsic quantities at $\mathcal{S}$.  To understand the need of the
calculation to be carried out in this section observe that, although
the conditions of proposition \ref{Prop:Propagation_twistor} are given
on $\mathcal{S}$, they contains not only derivatives tangential to
$\mathcal{S}$ but also normal to it. Hence, to obtain genuine
intrinsic conditions on $\mathcal{S}$ one needs to remove these normal
derivatives.

\medskip

Introducing the following definitions:
\begin{align}
  %H_{CAB} \eqref \tau _{C}{}^{A'} H_{A'AB}, \qquad
  %\mathcal{H} _{ABC} \equiv H_{(ABC)},
  %\qquad \mathcal{H} _{A} \equiv H^{D}{}_{AD}.
  \mathcal{H} _{ABC}  \equiv \tau _{(A}{}^{A'}H_{|A'|BC)}, \qquad
  \mathcal{H}_{A}  \equiv  \tau^{QA'} H_{A'AQ},
\end{align}
the space spinor split of $H_{A'AB}$ reads
\begin{align}
  H_{A'AB} = - \tfrac{1}{2} \tau ^{C}{}_{A'} \mathcal{H} _{ABC}  -
  \tfrac{1}{6} \tau ^{C}{}_{A'} \mathcal{H} _{B} \epsilon _{AC}  -
  \tfrac{1}{6} \tau ^{C}{}_{A'} \mathcal{H} _{A} \epsilon _{BC}.
\end{align}
Hence, the space spinors $\mathcal{H} _{ABC}$ and $\mathcal{H}_{A}$
contain all the information of $H_{A'AB}$. In other words,
\[
H_{A'ABC}=0 \quad                   %|_{\mathcal{S}}=0
\iff \quad \mathcal{H} _{A}=0      %|_{\mathcal{S}} =0
\quad
\& \quad \mathcal{H}_{ABC}=0  %|_{\mathcal{S}}=0
\]
Substituting the definition \eqref{Def_H_twistor} one obtains
\begin{align}\label{spacespinordecompHtotwistorders}
\mathcal{H} _{A} = \tfrac{3}{2} \nabla_\tau \kappa_{A} - \mathcal{D} _{AB}\kappa^{B}, \qquad \mathcal{H} _{ABC} = 2 \mathcal{D} _{(AB}\kappa _{C)},
\end{align}
Then, $H_{A'AB}|_{\mathcal{S}}=0$  imposes
following conditions on the
the initial data
$(\kappa_A,\nabla_\tau\kappa_A)|_{\mathcal{S}}$
for the twistor candidate wave equation
\eqref{Wave_eq_twistor_candidate_prop}:
\begin{align}\label{H_twistor_vanishes_ID}
 \nabla_\tau \kappa _{A} = \tfrac{2}{3} \mathcal{D} _{AB}\kappa ^{B}, \qquad
 \quad \mathcal{D} _{(AB}\kappa _{C)}=0 \qquad \text{on} \qquad \mathcal{S}.
\end{align}
Another set of constraints arise from the conditions $\nabla_\tau
H_{A'BC}|_{\mathcal{S}}=0$ and $B_{ABC}|_{\mathcal{S}}=0$. These two
conditions can be analysed in tandem since $\bmB$ is related to the
derivative of $\bmH$. Using the space spinor split of $\nabla$ it
follows, exploting the identity \eqref{derH_twistor_toBandQ}, that
\begin{align}
  \tau _{D}{}^{A'}\nabla_\tau H_{A'AB} -2 \tau ^{CA'} \mathcal{D}
  _{DC}H_{A'AB} = 4 B_{ABD} \Xi + \tfrac{4}{3} Q_{(A}\epsilon
  _{B)D}\quad
\end{align}
Transvecting with $\tau^{D}{}_{B'}$ and rearranging gives
\begin{align}
\nabla_\tau H_{B'AB} = -4 B_{ABD} \Xi \tau ^{D}{}_{B'} -2 \tau ^{CA'}
\tau ^{D}{}_{B'} \mathcal{D} _{DC}H_{A'AB} - \tfrac{4}{3} \tau
^{D}{}_{B'}Q_{(A}\epsilon _{B)D}
\end{align}
Hence, if the the twistor candidate wave equation is imposed, namely
$Q_A=0$, then
\begin{align}
H_{A'AB}|_{\mathcal{S}}=0\quad \& \quad B_{ABC}|_{\mathcal{S}}=0
\implies \nabla_\tau H_{A'AB}|_{\mathcal{S}}=0.
\end{align}
In other words, imposing $\nabla_\tau H_{A'AB}|_{\mathcal{S}}=0$ is
redundant if $H_{A'AB}|_{\mathcal{S}}=0$ and $
B_{ABC}|_{\mathcal{S}}=0$ are satisfied.  Using the definition
\eqref{Def_B_twistor}, the condition $B_{ABC}|_{\mathcal{S}}=0$ simply
reads,
\begin{align}
  \phi_{ABCD}\kappa^D=0 \qquad \text{on} \qquad \mathcal{S}.
\end{align}
Finally, for the condition $\nabla_{\tau}B_{ABC}|_{\mathcal{S}}=0$ one
has, applying $\nabla_\tau$ to equation \eqref{Def_B_twistor} that
\begin{align}
\nabla_\tau B_{ABC} = \phi _{ABCD}\nabla_\tau \kappa ^{D} + \kappa
^{D} \nabla_\tau \phi _{ABCD} .
\end{align}
At this point one can exploit the evolution equation for the rescaled
Weyl spinor \eqref{RescaledWeyl_evo_const} to subsitute for
$\nabla_\tau \phi_{ABCD}$ and using condition \eqref {H_twistor_vanishes_ID}
to substitute $\nabla_\tau \kappa_A$ when evaluating at
$\mathcal{S}$. Namely,
%the condition
%$\nabla_{\tau}B_{ABC}|_{\mathcal{S}}=0$
%reads
\begin{align}\label{normalderB_twistor_exp}
\nabla_{\tau}B_{ABC}|_{\mathcal{S}}= -2\kappa ^{D} \mathcal{D} _{DF}\phi _{ABC}{}^{F} + \tfrac{2}{3}  \phi
_{ABCD} \mathcal{D} ^{D}{}_{F}\kappa ^{F} = 0 \qquad \text{on} \qquad \mathcal{S}.
\end{align}
In fact, the latter expression can be rewritten in terms of $\mathcal{H}_{ABC}|_{\mathcal{S}}$
and $B_{ABC}|_{\mathcal{S}}$. This can be done as follows: swapping indices $D$ and $A$ in equation
\eqref{normalderB_twistor_exp}, and exploiting the constraint equation for the rescaled Weyl spinor in expression
\eqref{RescaledWeyl_evo_const} renders
\begin{align}\label{normalderB_twistor_exp2}
\nabla_{\tau}B_{ABC}|_{\mathcal{S}}= -2 \kappa ^{D} \mathcal{D} _{AF}\phi _{DBC}{}^{F} + \tfrac{2}{3} \phi _{ABCD}
\mathcal{D} ^{D}{}_{F}\kappa ^{F} = 0 \qquad \text{on} \qquad
\mathcal{S}.
\end{align}
Applying a $\mathcal{D}_{FQ}$ to the definition \eqref{Def_B_twistor}
and using the Leibnitz rule, one can replace the first term in the
last equation to obtain
\begin{align}\label{normalderB_twistor_exp3}
\nabla_{\tau}B_{ABC}|_{\mathcal{S}}= -2 \mathcal{D} _{AD}B_{BC}{}^{D}
-2 \phi _{BCDF} \mathcal{D} _{A}{}^{F}\kappa ^{D} +\tfrac{2}{3} \phi
_{ABCF} \mathcal{D} _{D}{}^{F}\kappa ^{D} = 0 \quad \text{on} \quad
\mathcal{S}.
\end{align}
From the irreducible decomposition of $\mathcal{D} _{AB}\kappa _{C}$
and using the expression for $\mathcal{H}_{ABC}$ equation
\eqref{spacespinordecompHtotwistorders} one has
\begin{align}\label{decompSenKappa}
\mathcal{D} _{AB}\kappa _{C} = \tfrac{1}{2} \mathcal{H} _{ABC} +
\tfrac{1}{3} \epsilon _{BC} \mathcal{D} _{AD}\kappa ^{D} +
\tfrac{1}{3} \epsilon _{AC} \mathcal{D} _{BD}\kappa ^{D}.
\end{align}
Substituting equation \eqref{decompSenKappa} into equation
\eqref{normalderB_twistor_exp3} renders
\begin{align}
\nabla_{\tau}B_{ABC}|_{\mathcal{S}}=- \phi _{BCDF} \mathcal{H}
_{A}{}^{DF} -2 \mathcal{D} _{AD}B_{BC}{}^{D} = 0 \quad \text{on} \quad
\mathcal{S}
\end{align}

\noindent Hence, overall, the only independent conditions to be
imposed are $H_{A'AB}|_{\mathcal{S}}=0$ and
$B_{ABC}|_{\mathcal{S}}=0$.

The given in this section can be summaried in the following



\begin{theorem}\label{Theorem_twitor}
Consider an initial data set for the vacuum conformal Einstein
field equations, as encoded in the CFE zero-quantities
\eqref{Def_ConfFactor_CFE_zeroquant}-\eqref{Def_cons_CFE_zeroquant},
on a spacelike hypersurface $\mathcal{S}$ and let
$\mathcal{U}\subset\mathcal{S}$ denote an open set.
The development
of the initial data set will have a twistor
(valence-1 Killing spinor) in the domain of
dependence of $\mathcal{U}$ if and only if
\begin{flalign}
   \mathcal{D} _{(AB}\kappa _{C)}=0 , %\label{CS-KID1_twistor}
  \qquad \phi_{ABCD}\kappa^D=0,  \label{CS-KID_twistor} 
\end{flalign}
are satisfied on $\mathcal{U}$. The twistor is obtained
evolving according to the wave equation:
\begin{align} \label{Wave_eq_twistor_candidate_theo}
\square \kappa _{A} = -2 \Lambda \kappa _{A}.
\end{align}
with initial data satisfying conditions in equation
\eqref{CS-KID_twistor} and
\begin{equation}
  \nabla_\tau \kappa _{A} = \tfrac{2}{3} \mathcal{D} _{AB}.
\end{equation}
\end{theorem}

\begin{proof}
The analysis of the last subsection shows that  conditions
\begin{eqnarray*}
 H_{A'AB}=0, \quad \nabla_\tau  H_{A'AB}=0, \quad B_{ABC}=0, \quad \nabla_\tau  B_{ABC}=0
\end{eqnarray*}
on $\mathcal{U}\subset \mathcal{S}$ are equivalent to the conditions \eqref{CS-KID_twistor}. 
Hence, using Proposition \ref{Prop:Propagation_twistor} one concludes that if equations \eqref{CS-KID_twistor}
hold on $\mathcal{U}$, then the domain of dependence of $\mathcal{U}$ is endowed with a twistor.  
\end{proof}



}

\medskip

GOT HERE!!! 11.11.2021.

  
\section{Conformal Killing spinor initial data}


\section*{Conclusions}

In this article a \emph{conformal} version of the Killing spinor
initial data equations given in \cite{GarVal08c} are derived. By
conformal it is understood that $(\mathcal{M},\bmg)$ is conformally
related to an Einstein spacetime
$(\tilde{\mathcal{M}},\tilde{\bmg})$. Consequently, we call these
conditions the \emph{conformal Killing spinor initial data equations}.
The existence of a non-trivial solution of this system of equations is
a necessary and sufficient condition for the existence of a Killing
spinor on the development. The conditions are intrinsic to a spacelike
hypersurface $\mathcal{S}\subset\mathcal{M}$. In the
case where the conformal rescaling is trivial, $\Xi = 1$, the
conditions reduce to those given in \cite{BaeVal10b}.  These
conditions contain one differential condition and two algebraic
conditions.  The differential condition corresponds to the so-called
\emph{spatial Killing spinor equation}.
%%%%%%%%%%%%%%%%%%%%%
%% The first algebraic condition
%% corresponds to the restriction of the Buchdahl constraint on the
%% initial hypersurface and the second imposes restrictions on the Cotton
%% spinor of the initial data set. Moreover, it was shown that, in a spin
%% dyad adapted to the Killing spinor, these conditions can be used along
%% with the conformal Einstein field equations to show that certain
%% components (at least half of them) of the Cotton spinor $Y_{ABCA'}$
%% have to vanish on the initial hypersurface $\mathcal{S}$.
Notice that
the conformal approach followed in this article ---i.e., use of the
 conformal Einstein field equations--- opens the
possibility to allow $\mathcal{S}$ to be determined by $\Xi = 0$ so
that it to corresponds to the conformal boundary $\mathscr{I}$. The
analysis given in this article already shows that in a potential
characterisation of the Kerr-de Sitter spacetime, via the existence of
Killing spinors at the (spacelike) conformal boundary.
%%%%%%%%%%%%%%%%%%%%%%%%%
%% % the Cotton spinor will play
%% %a replant role. This is not unexpected since the conformal boundary of
%% %the Kerr-de Sitter spacetime is conformally flat ---see
%% %\cite{AshBonKes15a, Olz13}.  Therefore, the Cotton tensor associated
%% %with asymptotic initial data corresponding to the Kerr-de Sitter
%% %spacetime vanishes.

Nonetheless, future applications are not
restricted to the analysis of de-Sitter like spacetimes. To see this,
notice that, the most delicate part of the analysis consisted on
finding a system of homogeneous wave equations for $H_{A'ABC}$ and
$B_{ABCD}$ and $F_{ABCD}$. This system of wave equations in turn, leads to
conditions \eqref{}-\eqref{}
which are irrespective of the causal nature of
$\mathcal{S}$. Consequently, one could investigate the analogous
conditions to those derived in Section \ref{}
considering a timelike or null hypersurface $\mathcal{S}$ instead. In
the latter case one could consider the conformal boundary of an
asymptotically flat spacetime. In the case of a timelike hypersurface
$\mathcal{S}$, the analogous conditions could be useful for the
 analysis of anti-de Sitter like spacetimes.


\subsection*{Acknowledgements}

We have profited from discussions with Juan A. Valiente Kroon.
E. Gasper\'in acknowledges support from Consejo Nacional de
Ciencia y Tecnolog\'ia (Mexico) ---CONACyT studentship
494039/218141--- in the early stages of this work and from Fundaç\~ao
para a Ci\^encia e a Tecnologia (Portugal) ---FCT-2020.03845.CEECIND---
during its completion.

% 
\bibliographystyle{/home/gasperin/Academic/References/reporthack}
\bibliography{/home/gasperin/Academic/References/GRbibJune2021a}
%

%% \begin{thebibliography}{10}

%% \bibitem{AshBonKes15a}
%% A.~{Ashtekar}, B.~{Bonga}, \& A.~{Kesavan},
%% \newblock {\em {Asymptotics with a positive cosmological constant: I. Basic
%%   framework}},
%% \newblock Classical and Quantum Gravity {\bf 32}(2), 025004 (Jan. 2015).

%% \bibitem{BaeVal10a}
%% T.~B\"{a}ckdahl \& J.~A. {Valiente Kroon},
%% \newblock {\em Geometric invariant measuring the deviation from Kerr data},
%% \newblock Phys. Rev. Lett. {\bf 104}, 231102 (2010).

%% \bibitem{BaeVal10b}
%% T.~B\"{a}ckdahl \& J.~A. {Valiente Kroon},
%% \newblock {\em On the construction of a geometric invariant measuring the
%%   deviation from Kerr data},
%% \newblock Ann. Henri Poincar\'e {\bf 11}, 1225 (2010).

%% \bibitem{BaeVal11b}
%% T.~B\"{a}ckdahl \& J.~A. {Valiente Kroon},
%% \newblock {\em The "non-Kerrness" of domains of outer communication of black
%%   holes and exteriors of stars},
%% \newblock Proc. Roy. Soc. Lond. A {\bf 467}, 1701 (2011).

%% \bibitem{BaeVal12}
%% T.~B\"{a}ckdahl \& J.~A. {Valiente Kroon},
%% \newblock {\em Constructing ``non-Kerrness'' on compact domains},
%% \newblock J. Math. Phys. {\bf 53}, 04503 (2012).

%% \bibitem{BeiChr97b}
%% R.~Beig \& P.~T. Chru\'{s}ciel,
%% \newblock {\em Killing initial data},
%% \newblock Class. Quantum Grav. {\bf 14}, A83 (1997).

%% \bibitem{ColVal16}
%% M.~J. {Cole} \& J.~A. {Valiente Kroon},
%% \newblock {\em {A geometric invariant characterising initial data for the
%%   Kerr-Newman spacetime}},
%% \newblock ArXiv e-prints  (Sept. 2016).

%% \bibitem{Fri81b}
%% H.~Friedrich,
%% \newblock {\em The asymptotic characteristic initial value problem for
%%   {Einstein}'s vacuum field equations as an initial value problem for a
%%   first-order quasilinear symmetric hyperbolic system},
%% \newblock Proc. Roy. Soc. Lond. A {\bf 378}, 401 (1981).

%% \bibitem{Fri81a}
%% H.~Friedrich,
%% \newblock {\em On the regular and the asymptotic characteristic initial value
%%   problem for {Einstein}'s vacuum field equations},
%% \newblock Proc. Roy. Soc. Lond. A {\bf 375}, 169 (1981).

%% \bibitem{Fri82}
%% H.~Friedrich,
%% \newblock {\em On the existence of analytic null asymptotically flat solutions
%%   of {Einstein}'s vacuum field equations},
%% \newblock Proc. Roy. Soc. Lond. A {\bf 381}, 361 (1982).

%% \bibitem{Fri83}
%% H.~Friedrich,
%% \newblock {\em Cauchy problems for the conformal vacuum field equations in
%%   General Relativity},
%% \newblock Comm. Math. Phys. {\bf 91}, 445 (1983).

%% \bibitem{Fri86c}
%% H.~Friedrich,
%% \newblock {\em Existence and structure of past asymptotically simple solutions
%%   of Einstein's field equations with positive cosmological constant},
%% \newblock J. Geom. Phys. {\bf 3}, 101 (1986).

%% \bibitem{Fri86b}
%% H.~Friedrich,
%% \newblock {\em On the existence of n-geodesically complete or future complete
%%   solutions of {E}instein's field equations with smooth asymptotic structure},
%% \newblock Comm. Math. Phys. {\bf 107}, 587 (1986).

%% \bibitem{Fri91}
%% H.~Friedrich,
%% \newblock {\em On the global existence and the asymptotic behaviour of
%%   solutions to the Einstein-Maxwell-Yang-Mills equations},
%% \newblock J. Diff. Geom. {\bf 34}, 275 (1991).

%% \bibitem{GasVal17}
%% E.~Gasper{\'i}n \& J.~A. Valiente~Kroon,
%% \newblock {\em Perturbations of the Asymptotic Region of the Schwarzschild--de
%%   Sitter Spacetime},
%% \newblock Annales Henri Poincar{\'e} , 1--73 (2017).

%% \bibitem{GarVal08c}
%% A.~G.-P. {G{\'o}mez-Lobo} \& J.~A. {Valiente Kroon},
%% \newblock {\em {Killing spinor initial data sets}},
%% \newblock Journal of Geometry and Physics {\bf 58}, 1186--1202 (Sept. 2008).

%% \bibitem{Mar99}
%% M.~Mars,
%% \newblock {\em A spacetime characterization of the Kerr metric},
%% \newblock Class. Quantum Grav. {\bf 16}, 2507 (1999).

%% \bibitem{Mar00}
%% M.~Mars,
%% \newblock {\em Uniqueness properties of the Kerr metric},
%% \newblock Class. Quantum Grav. {\bf 17}, 3353 (2000).

%% \bibitem{MarPaeSen16}
%% M.~{Mars}, T.-T. {Paetz}, J.~M.~M. {Senovilla}, \& W.~{Simon},
%% \newblock {\em {Characterization of (asymptotically) Kerr-de Sitter-like
%%   spacetimes at null infinity}},
%% \newblock Classical and Quantum Gravity {\bf 33}(15), 155001 (Aug. 2016).

%% \bibitem{McLBer93}
%% R.~G. McLenaghan \& N.~V. den Bergh,
%% \newblock {\em Spacetimes admitting Killing 2-spinors},
%% \newblock Classical and Quantum Gravity {\bf 10}(10), 2179 (1993).

%% \bibitem{Olz13}
%% C.~\"{O}lz,
%% \newblock {\em The global structure of Kerr-de Sitter metrics},
%% \newblock Master thesis, University of Vienna, 2013.

%% \bibitem{Pae13}
%% T.-T. Paetz,
%% \newblock {\em Conformally covariant systems of wave equations and their
%%   equivalence to Einstein's field equations},
%% \newblock In {\tt arXiv:1306.6204}, 2013.

%% \bibitem{Pae14}
%% T.-T. {Paetz},
%% \newblock {\em {Killing Initial Data on spacelike conformal boundaries}},
%% \newblock ArXiv e-prints  (Mar. 2014).

%% \bibitem{Pae14a}
%% T.-T. Paetz,
%% \newblock {\em KIDs prefer special cones},
%% \newblock Classical and Quantum Gravity {\bf 31}(8), 085007 (2014).

%% \bibitem{PenRin84}
%% R.~Penrose \& W.~Rindler,
%% \newblock {\em Spinors and space-time. {V}olume 1. {T}wo-spinor calculus and
%%   relativistic fields},
%% \newblock Cambridge University Press, 1984.

%% \bibitem{PenRin86}
%% R.~Penrose \& W.~Rindler,
%% \newblock {\em Spinors and space-time. {V}olume 2. {S}pinor and twistor methods
%%   in space-time geometry},
%% \newblock Cambridge University Press, 1986.

%% \bibitem{Rob75b}
%% D.~C. Robinson,
%% \newblock {\em Uniqueness of the Kerr black hole},
%% \newblock Phys. Rev. Lett. {\bf 34}, 905 (1975).

%% \bibitem{Sim84}
%% W.~Simon,
%% \newblock {\em Characterizations of the Kerr metric},
%% \newblock Gen. Rel. Grav. {\bf 16}, 465 (1984).

%% \bibitem{Som80}
%% P.~Sommers,
%% \newblock {\em Space spinors},
%% \newblock J. Math. Phys. {\bf 21}, 2567 (1980).

%% \bibitem{Ste91}
%% J.~Stewart,
%% \newblock {\em Advanced general relativity},
%% \newblock Cambridge University Press, 1991.

%% \bibitem{Tay96c}
%% M.~E. Taylor,
%% \newblock {\em Partial differential equations {III}: nonlinear equations},
%% \newblock Springer Verlag, 1996.

%% \bibitem{CFEbook}
%% J.~A. {Valiente Kroon},
%% \newblock {\em Conformal methods in General Relativity},
%% \newblock Cambridge University Press ---in preparation.

%% \bibitem{Wei90a}
%% G.~Weinstein,
%% \newblock {\em On rotating black holes in equilibrium in general relativity},
%% \newblock Communications on Pure and Applied Mathematics {\bf 43}(7), 903--948
%%   (1990).

%% \end{thebibliography}






\end{document}
