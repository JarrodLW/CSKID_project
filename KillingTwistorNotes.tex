%%%%%%%
%%%%%%% Conformal Killing Spinor Initial Data 
%%%%%%% 
%%%%%%% arXiv version.
%%%%%%%
%%%%%%%
%%%%%%% Started on: 10.9.2016
%%%%%%% Current version: 09.11.2021
%%%%%%%

\documentclass[10pt,a4paper]{article}
\usepackage{amssymb}
\usepackage{amsmath}
\usepackage{amsfonts}
\usepackage{amsthm}
\usepackage{latexsym}
\usepackage{mathrsfs}
\usepackage{stmaryrd}
\usepackage[dvips]{epsfig}
\usepackage{setspace}
\usepackage{float}
\usepackage{bm}
%\usepackage{showkeys}
%\usepackage[pdftex]{graphicx}
\usepackage{tikz}
\usepackage{enumerate}
\usepackage{subfigure}
\usepackage{nomencl}
%\usepackage{makeidx} 
%\makeindex
\usepackage{authblk}
\renewcommand\Affilfont{\itshape\small}
\usepackage{textcomp}
\usepackage{scrextend}% add KOMA-Script features to other classes
\usepackage[toc,page]{appendix}
\usepackage{comment}
\usepackage{xcolor}


\theoremstyle{plain}
\newtheorem{proposition}{Proposition}
\newtheorem{lemma}{Lemma}
\newtheorem{theorem}{Theorem}
\newtheorem{assumption}{Assumption}
\newtheorem*{conjecture}{Conjecture}
\newtheorem*{subconjecture}{Subconjecture}
\newtheorem{corollary}{Corollary}
\newtheorem*{main}{Theorem}
\newtheorem*{definition}{Definition}
\newtheorem{remark}{Remark}

\setlength{\textwidth}{148mm}           % Width of text on page- max 148
\setlength{\textheight}{235mm}          % height of text on page-max 235
\setlength{\topmargin}{-10mm}            % Margin at top ofpage- max -5
\setlength{\oddsidemargin}{0mm}         % Odd page sidemargin max 15
\setlength{\evensidemargin}{0mm}

% Underlined lowcase latin letters
\def\es{{\bar{s}}}
\def\er{{\bar{r}}}

% Boldface mathmode lowcase latin letters
\def\bma{{\bm a}}
\def\bmb{{\bm b}}
\def\bmc{{\bm c}}
\def\bmd{{\bm d}}
\def\bme{{\bm e}}
\def\bmf{{\bm f}}
\def\bmg{{\bm g}}
\def\bmh{{\bm h}}
\def\bmi{{\bm i}}
\def\bmj{{\bm j}}
\def\bmk{{\bm k}}
\def\bml{{\bm l}}
\def\bmn{{\bm n}}
\def\bmm{{\bm m}}
\def\bmo{{\bm o}}
\def\bmq{{\bm q}}
\def\bms{{\bm s}}
\def\bmt{{\bm t}}
\def\bmu{{\bm u}}
\def\bmv{{\bm v}}
\def\bmw{{\bm w}}
\def\bmx{{\bm x}}
\def\bmy{{\bm y}}
\def\bmz{{\bm z}}

% Boldface mathmode numbers
\def\bmzero{{\bm 0}}
\def\bmone{{\bm 1}}
\def\bmtwo{{\bm 2}}
\def\bmthree{{\bm 3}}

% Boldface mathmode uppercase latin letters
\def\bmA{{\bm A}}
\def\bmB{{\bm B}}
\def\bmC{{\bm C}}
\def\bmD{{\bm D}}
\def\bmE{{\bm E}}
\def\bmF{{\bm F}}
\def\bmG{{\bm G}}
\def\bmH{{\bm H}}
\def\bmK{{\bm K}}
\def\bmL{{\bm L}}
\def\bmM{{\bm M}}
\def\bmN{{\bm N}}
\def\bmP{{\bm P}}
\def\bmQ{{\bm Q}}
\def\bmR{{\bm R}}
\def\bmS{{\bm S}}
\def\bmT{{\bm T}}
\def\bmX{{\bm X}}
\def\bmZ{{\bm Z}}


\def\Riem{{\bm R}{\bm i}{\bm e}{\bm m}}
\def\Ric{{\bm R}{\bm i}{\bm c}}
\def\Weyl{{\bm W}{\bm e}{\bm y}{\bm l}}
\def\RWeyl{{\bm R}{\bm W}{\bm e}{\bm y}{\bm l}}
\def\Sch{{\bm S}{\bmc}{\bm h}}
\def\Schouten{{\bm S}{\bmc}{\bm h}{\bm o}{\bm u}{\bm t}{\bm e}{\bm n}}
\def\Hessian{{\bm H}{\bm e}{\bm s}{\bm s}}

% Fracture letters
\def\fraka{{\frak a}}
\def\frakb{{\frak b}}
\def\frakc{{\frak c}}
\def\frakd{{\frak d}}
\def\frakf{{\frak f}}
\def\frakg{{\frak g}}
\def\fraki{{\frak i}}
\def\frakj{{\frak j}}
\def\frakk{{\frak k}}

% Mathbf letters
\def\mbfu{\mathbf{u}}

% Boldface mathmode lowcase greek letters
\def\bmalpha{{\bm \alpha}}
\def\bmbeta{{\bm \beta}}
\def\bmgamma{{\bm \gamma}}
\def\bmdelta{{\bm \delta}}
\def\bmepsilon{{\bm \epsilon}}
\def\bmeta{{\bm \eta}}
\def\bmzeta{{\bm\zeta}}
\def\bmxi{{\bm \xi}}
\def\bmchi{{\bm \chi}}
\def\bmiota{{\bm \iota}}
\def\bmomega{{\bm \omega}}
\def\bmlambda{{\bm \lambda}}
\def\bmmu{{\bm \mu}}
\def\bmnu{{\bm \nu}}
\def\bmphi{{\bm \phi}}
\def\bmvarphi{{\bm \varphi}}
\def\bmsigma{{\bm \sigma}}
\def\bmvarsigma{{\bm \varsigma}}
\def\bmtau{{\bm \tau}}
\def\bmupsilon{{\bm \upsilon}}

% Boldface mathmode uppercase greek letters
\def\bmGamma{{\bm \Gamma}}
\def\bmPhi{{\bm \Phi}}
\def\bmUpsilon{{\bm \Upsilon}}
\def\bmSigma{{\bm \Sigma}}

% Boldface operators
\def\bmpartial{{\bm \partial}}
\def\bmnabla{{\bm \nabla}}
\def\bmhbar{{\bm \hbar}}
\def\bmperp{{\bm \perp}}
\def\bmell{{\bm \ell}}

% Complex and real numbers
\font\SYM=msbm10
\newcommand{\Real}{\mbox{\SYM R}}
\newcommand{\Complex}{\mbox{\SYM C}}
\newcommand{\Natural}{\mbox{\SYM N}}
\newcommand{\Integer}{\mbox{\SYM Z}}
\newcommand{\Sphere}{\mbox{\SYM S}}

% ParallelPerp symbol
\newcommand{\parperp}{\mathbin{\text{\rotatebox[origin=c]{90}{$\models$}}}}
\newcommand{\perppar}{\mathbin{\text{\rotatebox[origin=c]{-90}{$\models$}}}}

%%%%% for piecewise functions right hand side brace %%%%%
\newenvironment{rcases}
  {\left.\begin{aligned}}
  {\end{aligned}\right\rbrace}

% Projector symbols
%\newcommand{\proj2perp}{\pi^{\small{\perp}}}
%\newcommand{\proj2par}{\pi^{\small{\parallel}}}
%\newcommand{\proj4par}{\pi^{\small{\parallel}}}
%\newcommand{\proj4parperp}{\pi^{\small{\parperp}}}
%\newcommand{\proj4perppar}{\pi^{\small{\perppar}}}
%\newcommand{\proj4perp}{\pi^{\small{\perp}}}

%Counter variable for the margin notes
\newcounter{mnotecount}%[section]

% This code generates the margin notes
\newcommand{\mnotex}[1]%{}
{\protect{\stepcounter{mnotecount}}$^{\mbox{\footnotesize $\bullet$\themnotecount}}$ 
\marginpar{%\color{red}%
\raggedright\tiny\em
$\!\!\!\!\!\!\,\bullet$\themnotecount: #1} }

\renewcommand\labelitemi{\tiny$\bullet$}

\newcommand{\notimplies}{%
  \mathrel{{\ooalign{\hidewidth$\not\phantom{=}$\hidewidth\cr$\implies$}}}}


%%%%%%%%%%%%%%%%%%%%%%% needed for long lists of equations
\allowdisplaybreaks
%%%%%%%%%%%%%%%%%%%%%%%%

\begin{document}


\title{\textbf{The conformal Killing spinor initial data equations}}

\author[1]{E. Gasper\'in \footnote{E-mail address:{\tt edgar.gasperin@tecnico.ulisboa.pt}}}
\author[2]{J.L. Williams \footnote{E-mail address:{\tt jlw31@bath.ac.uk}}}
\affil[1]{CENTRA, Departamento de F\'isica,
  Instituto Superior T\'ecnico IST, Universidade de Lisboa UL, Avenida
  Rovisco Pais 1, 1049 Lisboa, Portugal.}
\affil[2]{Department of Mathematical Sciences, University of Bath, Claverton Down, Bath BA2 7AY, United Kingdom.}



\maketitle
\begin{abstract}
 We obtain necessary and sufficient conditions for an initial data set for the
 conformal Einstein field equations to give rise to a spacetime
 development in possession of a Killing spinor. This
 constitutes the conformal analogue of the Killing spinor initial data
 equations derived in \cite{GarVal08c}. The fact that the conformal
 Einstein field equations are used in our derivation allows for the 
possibility that the initial hypersurface be (part of) the conformal boundary
 $\mathscr{I}$. For conciseness, these conditions are derived assuming
 that the initial hypersurface is spacelike. Consequently, these
 equations encode necessary and sufficient conditions for the
 existence of a Killing spinor in the development of asymptotic
 initial data on spacelike components of $\mathscr{I}$.
\end{abstract}


\section{Introduction}

The discussion of symmetries  in General
Relativity is ubiquitous. From the question of integrability of the geodesic
equations to the existence of explicit solutions to the Einstein field
equations and the black hole uniqueness problem, symmetries always
  play an important role.   Symmetry assumptions are usually incorporated into
 the Einstein field equations ---which in vacuum read
\begin{equation}
\tilde{R}_{ab}=\lambda \tilde{g}_{ab},
\label{EFEVacuum}
\end{equation} 
 through the use of Killing vectors.  From the spacetime point
 of view, the existence of Killing vectors allows one to perform
 \emph{symmetry reductions} of the Einstein field equations ---see for
 instance \cite{Wei90a}. This approach has been exploited in classical
 uniqueness results such as \cite{Rob75b}.  Closely related to the black
 hole uniqueness problem, characterisations and classifications of
  solutions to the Einstein field equations usually exploit the
 symmetries of the spacetime in one way or another, e.g., in the
 characterisations of the Kerr spacetime via the \emph{Mars-Simon
   tensor} ---see \cite{Mar99,Mar00,Sim84}.  On the other hand, from
 the point of view of the Cauchy problem, symmetry assumptions should
 be imposed only at the level of initial data. In this regard,
 symmetry assumptions can be phrased in terms of the \emph{Killing
   vector initial data}.  The Killing vector initial data equations
 constitute a set of conditions that an initial data set
 $(\tilde{\mathcal{S}},\tilde{\bmh},\tilde{\bmK})$ for the Einstein
 field equations has to satisfy to ensure that the development will
 contain a Killing vector ---see \cite{BeiChr97b}.  Nevertheless,
 despite the fact that the existence of Killing vector plays a
central  role in the discussion of the symmetries, the
 existence of Killing vectors is sometimes not enough to encode all
 the symmetries and conserved quantities that a spacetime can posses,
 e.g., the Carter constant in the Kerr spacetime. To unravel some of
 these \emph{hidden symmetries} one has analyse the existence of a
 more fundamental type of objects; \emph{Killing spinors}
 $\tilde{\kappa}_{AB}$ ---in vacuum spacetimes, the existence of a
 Killing spinor directly implies the existence of a Killing
 vector. The \emph{Killing spinor initial data equations} have been
 derived in the \emph{physical framework} ---governed by the Einstein field
   equations--- in \cite{GarVal08c}.  These equations have been
 successfully employed in the construction of a geometric invariant
 which detects whether or not an initial data set corresponds to
 initial data for the Kerr spacetime ---see
 \cite{BaeVal10a,BaeVal10b,BaeVal11b}.
This analysis has also been extended to include suitable classes of
matter ---see \cite{ColVal16} for an analogous characterisation of
initial data for the Kerr-Newman spacetime.  In these
characterisations, some asymptotic conditions on the initial data are
required. These conditions  usually take the form of decay assumptions
on   $\tilde{\bmh}$, $\tilde{\bmK}$ and $\tilde{\bm\kappa}$ on
$\tilde{\mathcal{S}}$, given in terms
of asymptotically Cartesian coordinates.  Nonetheless, in other
approaches, the asymptotic behaviour of the spacetime can be studied in
a geometric way through conformal compactifications. The latter is
sometimes referred as the Penrose proposal. In this approach one
starts with a \emph{physical spacetime}
$(\tilde{\mathcal{M}},\tilde{\bmg})$ where $\tilde{\mathcal{M}}$ is a
4-dimensional manifold and $\tilde{\bmg}$ is a Lorentzian metric which
is a solution to the Einstein field equations.  Then, one introduces a
\emph{unphysical spacetime} $(\mathcal{M},\bmg)$ into which
$(\tilde{\mathcal{M}},\tilde{\bmg})$ is conformally embedded.
Accordingly, there exists an embedding $\varphi: \tilde{\mathcal{M}}
\rightarrow \mathcal{M}$ such that
\begin{equation} \label{eqn:Chapter:Introduction:ConformalRescaling}
\varphi^{*}\bmg=\Xi^2\tilde{\bmg}.
\end{equation}
 By suitably choosing the \emph{conformal factor} $\Xi$ the metric
 $\bmg$ may be well defined at the points where $\Xi=0$. In such
 cases, the set of points for where the conformal factor vanishes is
 at infinity from the physical spacetime perspective.
\noindent The set
\[
 \mathscr{I} \equiv \big\{p \in \mathcal{M} \hspace{0.2cm}| \hspace{0.2cm} \Xi(p)=0
 , \hspace{0.2cm} \mathbf{d}\Xi(p) \neq0\big\}
\]
is called the conformal boundary.  However, it can be readily
verified that the Einstein field equations are not conformally
invariant. Moreover, a direct computation using the conformal
transformation formula for the Ricci tensor shows that the vacuum
Einstein field equations \eqref{EFEVacuum}, lead to an equation which
is formally singular at the conformal boundary.  An approach to deal
with this problem was given in \cite{Fri81a} where a regular set of
equations for the unphysical metric was derived. These equations are
known as the \emph{conformal Einstein field equations}.  The crucial
property of these equations is that they are regular at the points
where $\Xi=0$ and a solution thereof implies whenever $\Xi\neq 0$ a
solution to the Einstein field equations ---see \cite{Fri81a,Fri83}
and \cite{CFEbook} for an comprehensive discussion.  There are three
ways in which these equations can be presented, the metric, the frame
and spinorial formulations. These equations have been mainly used in
the stability analysis of spacetimes ---see for instance \cite{Fri86b,
  Fri86c} for the proof of the global and semiglobal non-linear
stability of the de Sitter and Minkowski spacetimes, respectively.

\medskip

A conformal version of the Killing vector initial data equations using
the metric formulation of the conformal Einstein field equations has
been obtained in \cite{Pae14a}.  In the latter reference, intrinsic
conditions on an initial hypersurface $\mathcal{S}\subset \mathcal{M}$
of the unphysical spacetime are found such that the development of the
data ---in the unphysical setting the evolution is governed by the
conformal Einstein field equations--- gives rise to a conformal
Killing vector of the unphysical spacetime $(\mathcal{M},\bmg)$ which, in
turn, corresponds to a Killing vector of the physical spacetime
$(\tilde{\mathcal{M}},\tilde{\bmg})$.  Notice that this approach, in
particular, allows $\mathcal{S}$ to be determined by $\Xi=0$ so that
it to corresponds to the conformal boundary $\mathscr{I}$.  The
unphysical Killing vector initial data equations have been derived for
the characteristic initial value problem on a cone in \cite{Pae14a}
and on a spacelike conformal boundary in \cite{Pae14}.


\medskip
For applications involving the the conformal Einstein field equations
---say in its spinorial formulation, one frequently has to fix the
gauge and write the equations in components.  Despite the fact that,
at first glance, the conformal Einstein field equations expressed in
components with respect to an arbitrary spin frame seem to be
overwhelmingly complicated, as shown in \cite{GasVal17}, symmetry
assumptions (spherical symmetry in the latter case) greatly reduce the
number of equations to be analysed.  In the case of Petrov type D
spacetimes, e.g.the Kerr-de Sitter spacetime, the symmetries of the
spacetime are closely related to the existence of Killing spinors.
Therefore, a natural question in this setting is whether a conformal
version of the Killing spinor initial data equations introduced in
\cite{GarVal08c} can be found. In other words, what are the extra
conditions that one has to impose on an initial data set for the
conformal Einstein field equations so that the arising development
contains a Killing spinor?  This question is answered in this article
by deriving such conditions which we call the \emph{conformal Killing
  spinor initial data equations}

Despite the fact that the
Killing spinor equation is conformally invariant, it is not a priori
clear whether the conditions 
of \cite{GarVal08c, BaeVal10b} may be translated directly into the unphysical setting.
Indeed, one expects this not to be the case, since the Einstein field equations
are not conformally invariant. Moreover, one consideration that is
exploited in the discussion of \cite{GarVal08c} is based on the fact
that, on an Einstein spacetime $(\tilde{\mathcal{M}},\tilde{\bmg})$, a
Killing spinor $\tilde{\kappa}_{AB}$ gives rise to a Killing vector
$\tilde{\xi}_{a}$ whose spinorial counterpart is given by
$\tilde{\xi}_{AA'}=\tilde{\nabla}_{A'}{}^{Q}\tilde{\kappa}_{QA}$. Nevertheless,
this property does not hold in general. In other words, if
$(\mathcal{M},\bmg)$, where $\bmg$ is not assumed to satisfy the
Einstein field equations, possess a Killing spinor $\kappa_{AB}$, then
the analogous concomitant $\xi_{AA'}=\nabla_{A'}{}^{Q}\kappa_{QA}$
does not correspond to a Killing vector ---not even a conformal
Killing vector.  This situation is not ameliorated if one assumes that
$(\mathcal{M},\bmg)$ satisfies the conformal Einstein field
equations. Nevertheless, as discussed in this article, in the latter
case one can show that using the conformal factor $\Xi$, the Killing
spinor $\kappa_{AB}$ and the \emph{auxiliary vector} $\xi_{a}$, one
can construct a conformal Killing vector $X_{a}$ associated to a
Killing vector $\tilde{X}_{a}$ of the physical spacetime
$(\tilde{\mathcal{M}},\tilde{\bmg})$.

%% The conditions of \cite{GarVal08c, BaeVal10b}
%% may be recovered from the results presented here by setting $\Xi = 1$. 

\medskip

{\color{blue}{ Although the conditions of \cite{GarVal08c} may be
    recovered from the results presented here by setting $\Xi = 1$ an
    important difference is that the set of variables that allow to
    obtain a closed system of homogeneous wave equations in the
    present case are different. The need for a
    different set of \emph{zero-quantities} to be propagated in the
    conformal case, can be traced back to the previous observation
    that in $(\mathcal{M},\bmg)$ the vector
    $\xi_{AA'}=\nabla_{A'}{}^{Q}\kappa_{QA}$ does not correspond to a
    (conformal) Killing vector. However,  a by product of the present analysis
    is that $\xi_{AA'}$ is a Weyl collineation ---see \cite{KatLevDav69} for definitions
    of curvature collineations.
    In the analysis of \cite{GarVal08c} the fact that
    $\tilde{\xi}_{AA'}=\tilde{\nabla}_{A'}{}^{Q}\tilde{\kappa}_{QA}$
    is a Killing vector is crucial since one propagates off the initial hypersurface,
    simultaneously,   $\tilde{\kappa}_{AA'}$ and the Killing vector $\tilde{\xi}^a$
    and introduces $\tilde{S}_{ab} \equiv \tilde{\nabla}_{(a}\tilde{\xi}_{b)}$ as a zero-quantity.
    Similarly, in the work of \cite{ValCol16} where the results of \cite{GarVal08c} are
    generalised to the case where $(\tilde{\mathcal{M}},\tilde{\bmg})$ satisfies the
    Einstein-Maxwell equations, the condition
    $\tilde{S}_{ab}=0$ is also verified by virtue of the so-called
    \emph{matter aligment condition}.  In the conformal setting analysed in this article,
     the analogous quantity $S_{ab}$ is not as geometrically motivated as in the previous cases
    and its usage as a variable in the system does not seem to lead to a closed system
    of explicitly regular homogeneous wave equations. Here by regular
    we refer to the absence of formally singular terms, such as $\Xi^{-1}$, in the
    equations. Instead, the variable that is central for the present analysis
    turns out to be the so-called \emph{Buchdahl constraint} (and derivatives
    thereof), which links directly the existence of Killing spinors
    with the Petrov type of $(\mathcal{M},\bmg)$. 

    Although the main
   objective of the present paper is deriving the valence-2 Killing spinor initial
   data in the conformal setting $(\mathcal{M},\bmg)$ ,
   we also derive the analogous conditions encoding the
    existence of a valence-1 Killing spinor.   The latter serves as a
    warm up exercise for the valence-2 case where one can
    already observe and understand the above discussed features and
    differences between the derivation of the conditions on
    $(\tilde{\mathcal{M}},\tilde{\bmg})$ and those on
    $(\mathcal{M},\bmg)$ in a simpler arena.
}}

   %% An interesting feature of our analysis is the fact that we make use of an
   %% alternative representation of the conformal Einstein field
   %% equations.  In
   %% principle, one could use the standard representation of the
   %% conformal Einstein field equations, however, some
   %% experimentation reveals that the latter approach leads to 
   %% Fuchsian systems of equations ---formally singular at the conformal
   %% boundary--- for quantities associated to the Killing spinor.


   For
   conciseness, the conformal Killing initial data equations are
   obtained on a spacelike hypersurface $\mathcal{S}$. Nonetheless, a
   similar computation can be performed on an hypersurface
   $\mathcal{S}$ with a different causal character.  The conditions
   found in this article have potential applications for the black
   hole uniqueness problem.  In particular, they can be used for an
   asymptotic characterisation of the Kerr-de Sitter spacetime analogous to
   \cite{MarPaeSenSim16} in terms of the existence of Killing spinors at
   the conformal boundary $\mathscr{I}$.

\medskip

The main results of this article are summarised  informally
 in the following:


\begin{main}\label{TheoremSummary}
If the conformal Killing spinor initial data equations
 \eqref{CS-KID1}-\eqref{CS-KID3} are satisfied
on an open set $\mathcal{U}\subset \mathcal{S}$, where
 $\mathcal{S}$ is a spacelike hypersurface on which initial data for 
the conformal Einstein field equations has been prescribed,
 then, the domain of dependence of $\mathcal{U}$  possesses a Killing spinor.
%% Moreover, assuming conditions \eqref{CS-KID1}, \eqref{CS-KID2} to hold, 
%% condition \eqref{CS-KID3} is equivalent to the vanishing of certain components
%%  of the Cotton spinor, with respect to a suitably-adapted spin dyad.
\end{main}

 A precise formulation is the content of Theorem
\ref{MainTheorem} and Proposition \ref{PropositionRestrictionOnCotton}.

\medskip

Involved computations throughout this article were facilitated through
the suite {\tt xAct} in {\tt Mathematica}.
Note that since the existence of a spinor structure is guaranteed for
globally-hyperbolic spacetimes ---see Proposition $4$ in
\cite{CFEbook}--- the use of spinors is not overly restrictive.

\subsection*{Overview of the article}
Section \ref{Sec:KillingSpinors} gives  an overview of Killing
spinors along with their conformal properties. In Section
\ref{Sec:CFEs} we describe the conformal Einstein
field equations, for later use; Section
\ref{Sec:KillinSpinorZeroQuantities} introduces the main objects of
interest in the propagation of Killing spinor data, namely the
\emph{Killing spinor zero-quantities}. In Section
\ref{Sec:PropagationEquations} we construct conformally-regular wave
equations for the zero-quantities, leading to necessary and sufficient
conditions for the existence of a Killing spinor ---see Proposition
\ref{Prop:Propagation}. In Section \ref{Sec:IntrinsicConditions} the
latter conditions and the space spinor formalism are used to obtain 
the conformal Killing spinor initial data equations
 on spacelike hypersurfaces---see
 Theorem \ref{MainTheorem}.
% % Finally, in Section
% % \ref{Sec:FurtherAnalysis} the latter equations are analysed with respect to
% % an adapted spin dyad and the implied restrictions on the Cotton
% % spinor are presented ---see Proposition
% % \ref{PropositionRestrictionOnCotton}.

 \subsection*{Notation and conventions}
 %\section{Spinor formalism in a nutshell}
Upper case Latin indices ~$_{ABC\cdots A'B'C'}$~ will be used as
abstract indices of the \emph{spacetime spinor} algebra, and the bold
numerals ~$_{\bm0\bm1\bm2\cdots}$~ denote components with respect to a
fixed spin dyad $ o^A\equiv
\epsilon_{\bm0}{}^A,\iota^A\equiv\epsilon_{\bm1}{}^A $ ---see Penrose
\& Rindler \cite{PenRin84} for further details.  Although spinor
notation will be preferred, for certain computations tensors will be
employed. Lower case Latin indices $_{a,b,c...}$ will be used as
abstract tensor indices.  For tensors, our curvature conventions are
fixed by
\[\nabla_{a}\nabla_{b}\kappa^c-\nabla_{b}\nabla_{a}\kappa^c=R_{ab}{}^{c}{}_{d}\kappa^{d}.\]
For spinors, the curvature conventions are fixed via the spinorial
Ricci identities which will be written in accordance with the above
convention for tensors.  To see this, recall that the commutator of
covariant derivatives $[ \nabla_{AA'},\nabla_{BB'}]$ can be expressed
in terms of the symmetric operator $\square_{AB}$ as
\[
[ \nabla_{AA'},\nabla_{BB'}]= \epsilon_{AB}\square_{A'B'} +
\epsilon_{A'B'}\square_{AB}
\]
where
\[
\square_{AB} \equiv \nabla_{Q'(A} \nabla_{B)}{}^{Q'}.
\]
 The action of the symmetric operator $\square_{AB}$ on valence-1
 spinors is encoded in the spinorial Ricci identities
\begin{subequations}
\begin{eqnarray}
&& \square_{AB}\xi_{C}=-\Psi_{ABCD} \xi^{D} +
  2\Lambda\xi_{(A}\epsilon_{B)C},
 \label{SpinorialRicciIdentities1} \\
&& \square_{A'B'}\xi_{C}=-\xi^{A}\Phi_{CA A' B'},
\label{SpinorialRicciIdentities2}
\end{eqnarray}
\end{subequations}
where $\Psi_{ABCD}$ and $\Phi_{AA'BB'}$ and $\Lambda$ are curvature spinors.
 The above identities can be extended to higher valence spinors in an
 analogous way ---see \cite{Ste91} for further discussion on these
 identities using different conventions to the ones used in this
 article. A related identity which will be systematically used in the
 following discussion is
\begin{equation}\label{DecomposeDoubleDerivativeContracted}
\nabla_{AQ'}\nabla_{B}{}^{Q'}=\square_{AB}+
\frac{1}{2}\epsilon_{AB}\square,
\end{equation}
where $\square_{AB}$ is the symmetric operator defined above and
$\square \equiv \nabla_{AA'}\nabla^{AA'}.$


\subsection*{Space spinor formalism} %\label{Subsec:DecomposedOperators}
\label{SpaceSpinorFormalism}
To have a self-contained discussion in this section the space spinor
formalism, originally introduced in \cite{Som80}, is briefly recalled
---see also \cite{GarVal08c,BaeVal10b,CFEbook}.  Let $\tau^{AA'}$
denote the spinorial counterpart of a timelike vector $\tau^{a}$,
normal to a spacelike hypersurface $\mathcal{S}$ and normalised so
that $\tau_{a}\tau^{a}=2$.  Then, it follows that
$\tau_{AA'}\tau^{AA'}=2$ and, consequently,
\[\tau_{AA'}\tau_B{}^{A'}=\epsilon_{AB}.\]  
The covariant derivative $\nabla_{AA'}$ is then decomposed into the
\emph{normal} and \emph{Sen} derivatives:
\begin{align*}
& \mathcal{P}\equiv \tau^{AA'}\nabla_{AA'},\\ & \mathcal{D}_{AB}\equiv
  \tau_{(A}{}^{A'}\nabla_{B)A'}.
\end{align*}
The \emph{Weingarten} spinor and the \emph{acceleration} of the
congruence are then defined by
\begin{align*}
& K_{ABCD} \equiv \tau_{D}{}^{C'} \mathcal{D}_{AB}\tau_{CC'},\\ &
  K_{AB} \equiv \tau_{B}{}^{C'} \mathcal{P}\tau_{AC'}.
\end{align*}
The above can be inverted to obtain the following formulae which will
prove useful in the sequel
\begin{align*}
& \mathcal{P} \tau_{CC'}=- K_{CD} \tau^{D}{}_{C'},\\ &
  \mathcal{D}_{AB}\tau_{CA'} = - K_{ABCD} \tau^{D}{}_{A'}.
\end{align*}
The distribution induced by $\tau_{AA'}$ is integrable if and only
$K^D{}_{(AB)D}=0$, in which case $K_{ABCD}$ describes the extrinsic
curvature of the resulting foliation.
 Nevertheless, this is not required for our subsequent discussion.
In other words, we will allow
 the possibility that the distribution is non-integrable
---i.e. the spinor $ K^D{}_{(AB)D}$ will not be assumed
to vanish. 

\medskip

Defining the spinors $\chi_{AB}\equiv K^D{}_{(AB)D}$,
$\chi_{ABCD}\equiv K_{(ABCD)}$ and $\chi\equiv K_{AB}{}^{AB}$, the
Weingarten spinor decomposes as follows
\begin{equation}
\label{ExtrinsicCurvatureSplit}
K_{ABCD} = \chi_{ABCD} - \tfrac{1}{2} \epsilon_{A(C}\chi_{D)B} -
\tfrac{1}{2} \epsilon_{B(C}\chi_{D)A} - \tfrac{1}{3} \chi
\epsilon_{A(C} \epsilon_{D)B}.
\end{equation}
For the following discussion we will also need the commutators form
with $\mathcal{P},~\mathcal{D}_{AB}$. To write these commutators in a
succinct way, first define
\[\widehat{\square}_{AB}\equiv \tau_A{}^{A'}\tau_B{}^{B'}\square_{A'B'}\]
from which, proceeding analogously as in \cite{BaeVal10b}, one obtains
\begin{align}
 \left[\mathcal{P},\mathcal{D}_{AB}
   \right]&=-\tfrac{1}{2}\chi_{AB}-\square_{AB}+\widehat{\square}_{AB}
+K_{(A}{}^D\mathcal{D}_{B)D}-K_{AB}{}^{FG}\mathcal{D}_{FG}, \label{CommutatorNormalSenDeriv}
\\ \left[\mathcal{D}_{AB},\mathcal{D}_{CD}\right]&=
\tfrac{1}{2}\left(\epsilon_{A(C}\square_{D)B}+\epsilon_{B(C}\square_{D)A}\right)
+\tfrac{1}{2}\left(\epsilon_{A(C}\widehat{\square}_{D)B}+\epsilon_{B(C}\widehat{\square}_{D)A}\right)
 \nonumber \\ &
 +\tfrac{1}{2}\left(K_{CDAB}\mathcal{P}-K_{ABCD}\mathcal{P}\right)
+K_{CDF(A}\mathcal{D}_{B)}{}^F-K_{ABF(C}\mathcal{D}_{D)}{}^F \label{CommutatorSenSenDeriv}
\end{align}
%% It will also prove convenient to decompose the tracefree Ricci spinor,
%% $\Phi_{AA'BB'}$ in space spinor form. To do so, introduce its space
%% spinor counterpart $\Phi_{ABCD}\equiv\tau_{B}{}^{B'} \tau_{D}{}^{D'}
%% \Phi_{ACB'D'}$.  The latter can be decomposed as
%% \begin{align}
%% \label{RicciSpaceSpinorSplit}
%% \Phi_{ABCD}& = \Theta_{ABCD} + \tfrac{1}{2}
%% \left(\epsilon_{C(B}\Phi_{D)A} + \epsilon_{A(B}\Phi_{D)C}\right) -
%% \tfrac{1}{3} \Phi \epsilon_{A(B}\epsilon_{D)C}
%% \end{align}
%% where
%% \[ \Phi \equiv \Phi_{A}{}^{A}{}_{B}{}^{B},\qquad \Phi_{AB} \equiv \Phi_{(AB)C}{}^{C},\qquad
%%  \Theta_{ABCD} \equiv \Phi_{(ABCD)}\]


\section{Killing spinors}\label{Sec:KillingSpinors}

To start the discussion it is convenient to introduce some notation
and definitions. Let $(\tilde{\mathcal{M}},\tilde{\bmg})$ be a
4-dimensional manifold equipped with a Lorentzian metric
$\tilde{\bmg}$ and denote by $\tilde{\nabla}$ its associated
Levi-Civita connection.  For the time being $\tilde{\bmg}$ is not
assumed to be a solution to the Einstein field equations
\eqref{EFEVacuum}.

{\color{blue}
A totally symmetric
$\tilde{\kappa}_{A_1...A_q}=\tilde{\kappa}_{(A_1...A_q)}$ valence$-q$
spinor is said to be a Killing spinor if the following equation is
satisfied
\begin{equation}\label{qValenceKillingspinor}
\tilde{\nabla}_{Q'(Q}\tilde{\kappa}_{A_1...A_q)}=0.
\end{equation}
An important property of the Killing spinor equation is that it is
conformally-invariant, in other words if $\bmg$ is conformally related
to $\tilde{\bmg}$, namely $\bmg=\Xi^2\tilde{\bmg}$ then
${\kappa}_{A_1...A_q}=\Xi^2 \tilde{\kappa}_{A_1...A_q}$ satisfies
\[{\nabla}_{Q'(Q}{\kappa}_{A_1...A_q)}=0,\]
where ${\nabla}$ is the Levi--Civita connection of ${\bmg}$.

\medskip
\noindent In this paper we will only focus only the case $q=1$ and $q=2$.
If $q=1$, the equation
%%%%
%then a valence$-1$ Killing spinor $\tilde{\kappa}_A$ satisfies the equation
%%%
\begin{equation}\label{TwistorEq}
  \tilde{\nabla}_{Q'(Q}\tilde{\kappa}_{A)}=0.
\end{equation}
%%%
% which
%%%
is usually referred as the \emph{twistor equation}. We will follow
this naming convention and refer to a valence$-1$ spinor
 satisfying equation \eqref{TwistorEq} as twistor.
Since only the cases $q=1$ and $q=2$ will be discussed in this paper, we will refer
to the case $q=1$ as the twistor case and the $q=2$ as the Killing
spinor case. Namely, we will say that a symmetric valence$-2$ spinor, $\tilde{\kappa}_{AB}=
\tilde{\kappa}_{(AB)}$, is a \textit{Killing spinor} if it satisfies
the equation
\begin{equation}
\tilde{\nabla}_{A'(A}\tilde{\kappa}_{BC)}=0.
\end{equation}

The Killing spinor equation and twistor equations are, in general, overdetermined;
in particular, they imply the so-called \textit{Buchdahl constraint}.
In the twistor case ($q=1$) this has the form
\[
\kappa^D\Psi_{ABCD}=0,
\]
while in the Killing spinor case ($q=2$) the Buchdahl constraint adquires the
form
\[
\tilde{\kappa}^Q{}_{(A}\Psi_{BCD)Q}=0,
\]
where $\Psi_{ABCD}$ denotes the conformally invariant Weyl spinor.
The latter condition restricts $\Psi_{ABCD}$ to be algebraically
special.  In the twistor case the spacetime is necessarity of Petrov
type N or O, hence restricting its applicability for characterisation of
black holes.  In the Killing spinor case the spacetime is only
restricted to be of Petrov type D, N or O.

\medskip

At first glance, the conformal invariance property of the
%(valence-$q$)
Killing spinor equation would seem to indicate that the
approach leading to the Killing spinor initial data conditions derived
in \cite{GarVal08c} would identically apply for $(\mathcal{M},\bmg)$
with $\tilde\bmg=\Xi^2\tilde{\bmg}$. This is not the case simply
because the Einstein field equations are not conformally invariant.
In other words, in the analysis of \cite{GarVal08c} the vacuum
Einstein field equations $\tilde{R}_{ab}=0$ were used, and, despite
that one can relate $R_{ab}$ with $\tilde{R}_{ab}$ this leads to
formally singular terms (terms containing $\Xi^{-1}$). Moreover, even if one
is willing to work with formally singular equations it is not apriori clear
that the choice of variables made in \cite{GarVal08c} will form a
closed homogeneous system in the conformal seeting.  To understand
this second point further, notice that for general manifold with
metric $(\tilde{\mathcal{M}},\tilde{\bmg})$ ---namely $\tilde{\bmg}$
not satifying any field equation--- the existence of a Killing spinor
$\tilde{\kappa}_{AB}$ is not related directly to the existence of a
Killing vector.
Nevertheless, if one assumes that $\tilde{\bmg}$
satisfies the vacuum Einstein field equations \eqref{EFEVacuum} then
the concomitant
\begin{equation*}
\tilde{\xi}_{AA'} \equiv \tilde{\nabla}^{B}{}_{A'}\tilde{\kappa}_{AB},
\end{equation*}
represents the spinorial counterpart of a complex Killing vector of
the spacetime $(\tilde{\mathcal{M}},\tilde{\bmg})$ ---see
\cite{GarVal08c} for further discussion. This point is subtle and even
in the physical (non-conformal) set up if one is to include matter
such as the Maxwell field and the analysis of \cite{GarVal08c} does
not straighfowardly apply since further conditions (the matter
alignment conditions) ---see \cite{ValCol16}--- need to be propagated.
}


\begin{remark}
  \emph{
  The notion of Killing spinors is related to that
  of Killing--Yano tensors. If a Killing spinor
$\tilde{\xi}_{AA'}$ is Hermitian, i.e.,
$\bar{\tilde{\xi}}_{AA'}=\tilde{\xi}_{AA'}$, then one can construct the
spinorial counterpart of a \emph{Killing--Yano tensor}
$\tilde{\Upsilon}_{ab}$ ---i.e. an antisymmetric $2-$tensor satisfying
$\tilde{\nabla}_{(a}\tilde{\Upsilon}_{b)c}=0$--- as follows
\[\tilde{\Upsilon}_{AA'BB'}=i(\tilde{\kappa}_{AB}\bar{\tilde{\epsilon}}_{A'B'}
-\bar{\tilde{\kappa}}_{A'B'}\tilde{\epsilon}_{AB}).\] Conversely,
given a Killing--Yano tensor, one can construct a Killing spinor
---see \cite{ColVal16,McLBer93,PenRin86}.}
\end{remark}


\medskip

In the sequel $(\tilde{\mathcal{M}},\tilde{\bmg})$ will be reserved to
denote the \emph{physical spacetime}, in other words, the symbol
$\tilde{ \quad}$ will be added to those fields associated with a
solution $\tilde{\bmg}$ to the vacuum Einstein field equations
\eqref{EFEVacuum}.  Similarly $(\mathcal{M},\bmg)$ will be used to
represent the \emph{unphysical spacetime} related to
$(\tilde{\mathcal{M}},\tilde{\bmg})$ via $\bmg=\Xi^2\tilde{\bmg}$.
---in a slight abuse of notation $\varphi(\tilde{\mathcal{M}})$ and
$\mathcal{M}$ will be identified so that the mapping $\varphi:
\tilde{\mathcal{M}}\rightarrow\mathcal{M}$ can be omitted.








\section{The  conformal Einstein field equations}
\label{Sec:CFEs}

%% As discussed in the introduction, for the derivation of the conformal
%% Killing initial data equations, an appropriate formulation of the
%% conformal Einstein field equations will be required.

 %In this section
 %we begin, for the sake of completeness, with a discussion of the standard
 %conformal field equations (CFEs) originally introduced in
 %\cite{Fri81a} by H. Friedrich ---see also \cite{CFEbook}.
 

%% Then, an
%% alternative formulation to these equations are presented.  The main
%% benefit of these equations, which we refer to as the \emph{alternative
%%   CFEs}, in our context is that the so-called \emph{rescaled Weyl
%%   tensor} is replaced by the \emph{Weyl tensor} through the
%% introduction of the \emph{Cotton tensor} as an additional unknown.
%% This latter approach was first proposed in \cite{Pae14a} by T. Paetz.


%%  The use of the alternative CFEs is vindicated by our final result
%% which indicates that the existence of a Killing spinor necessarily
%% places restrictions on the components of the Cotton spinor at
%% the level of the initial data ---see Theorem \ref{TheoremSummary},
%% Proposition \ref{PropositionRestrictionOnCotton}.

%%%%%%%%
%\subsection{The standard conformal Einstein field equations}
%%%%%%%


\medskip

This section contains an abriged discussion of the CFEs in first and second order form.
At the end of this section the main technical tool from the theory of partial differential
equations to be used for deriving the Killing spinor intial data equations is given.

\medskip
 

The conformal Einstein field equations are a conformal formulation of
the Einstein field equations. In other words, given a spacetime
$(\tilde{\mathcal{M}},\tilde{\bmg})$ satisfying the Einstein field
equations, the conformal Einstein field equations encode a system of
differential conditions for the curvature and concomitants of the
conformal factor associated with $(\mathcal{M},\bmg)$ where
$\bmg=\Xi^2\tilde{\bmg}$. The key property of these equations is that
they are regular even at the conformal boundary $\mathscr{I}$, where
$\Xi=0$.  This formulation of the conformal Einstein field equations
was first given in \cite{Fri81a} ---see also \cite{CFEbook} for a
comprehensive discussion.

\medskip

The metric version of the standard vacuum conformal 
Einstein field equations are encoded in the following zero-quantities
  ---see \cite{Fri81a,Fri81b,Fri82,Fri83}:
\begin{subequations}\label{CFE_tensor_zeroquants}
\begin{eqnarray}
&& Z_{ab} \equiv \nabla_{a}\nabla_{b}\Xi  +\Xi L_{ab} - s g_{ab}=0 ,
 \label{StandardCEFEsecondderivativeCF}\\
&& Z_{a} \equiv \nabla_{a}s +L_{ac} \nabla ^{c}\Xi=0 , \label{standardCEFEs}\\
&& \delta_{bac} \equiv \nabla_{b}L_{ac}-\nabla_{a}L_{bc} -
 d_{abcd}\nabla^d{}\Xi =0 , \label{standardCEFESchouten}\\
&& \lambda_{abc}\equiv \nabla_{e}d_{abc}{}^{e}=0 , \label{standardCEFErescaledWeyl}\\
&& Z \equiv \lambda - 6 \Xi s + 3 \nabla_{a}\Xi \nabla^{a}\Xi
\label{standardCFEconstraintFriedrichScalar}
\end{eqnarray}
\end{subequations}
where $\Xi$ is the conformal factor, $L_{ab}$ is the Schouten tensor,
defined in terms of the Ricci tensor $R_{ab}$ and the Ricci scalar $R$
via
\begin{equation}\label{SchoutenDefinition}
L_{ab}=\frac{1}{2}R_{ab}-\frac{1}{12}Rg_{ab},
\end{equation}
 $s$ is the so-called \emph{Friedrich scalar} defined as
\begin{equation}\label{s-definition}
s\equiv \tfrac{1}{4}\nabla_{a}\nabla^{a}\Xi + \tfrac{1}{24}R\Xi
\end{equation}
and $d^{a}{}_{bcd}$ denotes the \emph{rescaled Weyl tensor}, defined
as
\[d^{a}{}_{bcd}=\Xi^{-1}C^{a}{}_{bcd},\]
where $C^{a}{}_{bcd}$ denotes the Weyl tensor.  The geometric meaning
of these zero-quantities is the following: The equation $Z_{ab}=0$
encodes the conformal transformation law between ${R}_{ab}$ and
$\tilde{R}_{ab}$.  The equation $Z_{a}=0$ is obtained considering
$\nabla^{a}Z_{ab}$ and commuting covariant derivatives.  Equations
$\delta_{abc}=0$ and $\lambda_{abc}=0$ encode the contracted second
Bianchi identity. Finally, $Z=0$ is a constraint in the sense that if
it is verified at one point $p\in\mathcal{M}$ then $Z=0$ holds in
$\mathcal{M}$ by virtue of the previous equations.  A solution to the
metric conformal Einstein field equations consist of a collection of
fields
\[
\{g_{ab}, \; \Xi, \; \nabla_{a}\Xi,s\;,L_{ab},\; d_{abcd}\}
\]
satisfying

\begin{equation}\label{vanishing_CFEs_tensorial_zq}
  Z_{ab}=0, \quad Z_{a}=0, \quad \delta_{abc}=0, \quad \lambda_{abc}=0, \quad Z=0.
\end{equation}

{ \color{blue}

\begin{remark}
  \emph{ If one opts to use the Ricci tensor $R_{ab}$ instead of the Schouten tensor $L_{ab}$ then
    the Ricci scalar $R$ appears in the right-hand side of
    equations but no equation for it has been provided.
    In the CFEs the Ricci scalar encodes the \emph{conformal gauge source function}, hence
    there is no equation to fix that variable since it is a gauge quantity in the formulation.}
\end{remark}

%% \begin{remark}
%% \emph{ In the metric formulation of the standard conformal Einstein field
%%   equations one needs to supplement the system encoded in the zero
%%   quantities defined above with an equation for the unphysical metric
%%   $g_{ab}$. To do so, one considers equation
%%   \eqref{SchoutenDefinition} expressed in some local coordinates
%%   $(x^{\mu})$. Recalling that in local coordinates the components of
%%   the Ricci tensor can be written as second order derivatives of the
%%   metric, one obtains the required equation for the unphysical metric.
%% % % This observation applies also for the subsequent
%% % % discussion of the alternative conformal Einstein field equations.
%% }
%% \end{remark}

Since the structural properties of the CFEs are better expressed in
spinorial formalism and due to the nature of the applications in this
article, the spinorial version of the CFEs will be used. The spinorial
 translation of the above CFEs zero-quantities render
---see \cite{CFEbook} for further details.

\begin{subequations}
\begin{flalign}
    Z_{AA'BB'}  = & - \Xi \Phi _{ABA'B'}  - s \epsilon _{AB} \epsilon
  _{A'B'} + \Xi \Lambda \epsilon _{AB} \epsilon _{A'B'} +
  \nabla_{BB'}\nabla_{AA'}\Xi \\
  Z_{AA'}  =& \Lambda  \nabla_{AA'}\Xi  + \nabla_{AA'}s  - \Phi _{ABA'B'} \nabla^{BB'}\Xi\\
  %%  Z_{AA'BB'CC'}  =& \epsilon _{BC} \epsilon _{B'C'} \nabla_{AA'}\Lambda
  % % - \nabla_{AA'}\Phi _{BCB'C'}   - \epsilon _{AC} \epsilon _{A'C'} \nabla_{BB'}\Lambda
  %% \\ &   + \nabla_{BB'}\Phi _{ACA'C'}
  %% - \bar{\phi }_{A'B'C'D'} \epsilon _{AB} \nabla_{C}{}^{D'}\Xi
  %% - \phi _{ABCD} \epsilon _{A'B'} \nabla^{D}{}_{C'}\Xi\\
  \delta_{ABCC'} = &  \nabla_{A'(A}\Phi _{B)CC'}{}^{A'} - \epsilon _{C(A} \nabla_{B)C'}\Lambda   +  \phi _{ABCD} \nabla^{D}{}_{C'}\Xi \label{Def_delta_CFE_zeroquant} \\
  \Lambda _{CC'AB}  =& \nabla_{DC'}\phi _{ABC}{}^{D} \label{Def_Lambda_CFE_zeroquant}\\
    Z  =& \lambda   -6 \Xi  s + 3 \nabla_{AA'}\Xi  \nabla^{AA'}\Xi
\end{flalign}
\end{subequations}
\mnotex{I think the original $Z_{AA'BB'CC'}$ does not
  contain more information than what $\delta_{ABCC'}$ encodes.   }
Similar to the tensorial case, one can choose to use  the
Schouten (tensor) spinor or the Ricci (tensor) spinor as a variable.
Here the equations have been expressed
using the standard curvature spinors and notation of the NP formalism,
namely, the trace-free Ricci spinor $\Phi_{ABA'B'}$, the Ricci scalar
$\Lambda$ ---in fact $R= 24\Lambda$--- and the Weyl spinor
$\Psi_{ABCD}$. The rescaled Weyl spinor $\phi_{ABCD}$ is defined as
\begin{equation}\label{Def_rescaled_Weyl_spinor}
\phi_{ABCD} \equiv \Xi^{-1} \Psi_{ABCD}.
\end{equation}


%%%%%
%\subsection{The Conformal Einstein field equations in second order form}
%%%%

The CFEs as previously presented can be regarded as a set of covariant
conditions for geometric fields on $(\mathcal{M},\bmg)$ and they do
not have a particular PDE character.  However, there are, depending on
the gauge fixing procedure, different hyperbolic reduction strategies
to extract a set of evolution and constraint equations.
For the subsequent discussion only the evolution and constraint equations
implied by the $\Lambda_{CC'AB}=0$ equation will play a role.
A direct calculation using the space spinor formalism shows that
$\Lambda_{CC'AB}=0$ can be recasted as the following system of evolution equation and
constraint equations
\begin{align}\label{RescaledWeyl_evo_const}
  %\nabla_\tau \phi _{ABCD} = -2 \mathcal{D} _{DF}\phi _{ABC}{}^{F}
  & \nabla_\tau \phi _{ABCD} =  2 \mathcal{D} _{(A}{}^{F}\phi _{BCD)F}, % \label{RescaledWeyl_evo}\\
  %& \qquad
  \qquad \mathcal{D} _{CD}\phi _{AB}{}^{CD} = 0. 
  %\label{RescaledWeyl_const}
\end{align}
The evolution and constraint equations associated to the
other zero-quantities depend on the particular gauge fixing strategy
and will not play a relevant role for the discussion in the next sections.

\medskip

The CFEs are usually presented as the first order system
\eqref{vanishing_CFEs_tensorial_zq} with the definitions
\eqref{CFE_tensor_zeroquants}, however, for several applications it is
convenient to use a second order formulation of the equations. In
\cite{Pae13} the tensorial version of the CFEs was recasted as a set
of (tensorial) wave equations.
%%%%%
%However, as
%previously discussed several structural properties of the CFEs are
%better expressed in spinorial formalism, with this motivation,
%%%%%
Similarly, in \cite{GasVal15} a second order form of the spinorial
formulation of the CFEs was obtained.  This version of the CFEs is
particularly suited for the applications of this article, and, in
fact, only one of those equations ---that for the rescaled Weyl
spinor--- will be needed \mnotex{Double check this is true}.
%%%%%%%%%%%
%Since, only not all of the equations
%will be needed, this section presents only those which are used in the
%calculations of the subsequent sections of this paper.
%%%%%%%%%%%
The wave equation for the rescaled Weyl spinor can be succintly obtained from
considering $\nabla^{QC'}\Lambda _{CC'AB}$.  A direct calculation using the identity
\eqref{DecomposeDoubleDerivativeContracted} shows that if $\Lambda _{CC'AB}=0$ then,
\begin{eqnarray}
  \square \phi _{ABCF} = 12 \Lambda  \phi _{ABCF}  -6 \Xi  \phi _{(AB}{}^{DG}\phi _{CF)DG}
  \label{Wave_eq_CFE_Weyl}
\end{eqnarray}
A similar calculation can be carried out for the other equations in comprising the CFEs.
A full discussion of the \emph{spinorial CFE wave equations} and their equivalence
with the standard first order formulation CFEs can be found in \cite{GasVal15}.
%
%\medskip
%
One of the main tools used in \cite{GasVal15} to show the
equivalence between these two set of equations is the uniqueness
property to a certain class of wave equations. This same result from
the theory of partial differential equations will be used to obtain
the main result of this article and is presented in the following
%%%%%%%
%One of the main tools used in the latter reference
%to show the equivalence between these two set of equations
%is the following result from the theory of partial differential equations which
%will be also used to obtain the main result of this article.
%%%%%
%This theorem will also be used in for de
% %The main tool from the theory of partial differential equations
%that will be used in the next sections is that of the uniqueness property for
%solutions to a certain class of wave equations.
%
%This theorem will be also used in the proof of the main result of this article.
%%%%%

}

\mnotex{Technical pde theorem moved here}
\begin{theorem}
\label{TheoremHomogeneousWave}
 Let $\mathcal{M}$ be a smooth manifold equipped with a 
Lorentzian metric $\bmg$ and consider the wave equation
\[\square \underline{u}=h\left(\underline{u},~\partial\underline{u}\right)\]
where $\underline{u}\in\mathbb{C}^m$ is a complex vector-valued function
on $\mathcal{M}$, 
$h:\mathbb{C}^{2m}\rightarrow\mathbb{C}^m$ is a smooth homogeneous
function of its arguments and $\square=g^{ab}\nabla_{a}\nabla_{b}$.
Let $\mathcal{U}\subset\mathcal{S}$ be an open set and $\mathcal{S}\subset \mathcal{M}$ be a spacelike hypersurface with normal
$\tau^{a}$ respect to $\bmg$. Then the Cauchy problem
\begin{align*}
\square
\underline{u}&=h\left(\underline{u},~\partial\underline{u}\right),\\ \underline{u}\left|_{\mathcal{U}}\right.&=\underline{u}_0,
\quad
\mathcal{P}\underline{u}\left|_{\mathcal{U}}\right.=\underline{u}_1,
\end{align*} 
where $\underline{u}_{0}$ and $\underline{u}_{1}$  are 
smooth on $\mathcal{U}$ and $\mathcal{P}\equiv \tau^\mu\nabla_\mu$,
 has a unique solution $\underline{u}$ in the domain of dependence of $\mathcal{U}$.
\end{theorem}
We refer the reader to
\cite{CFEbook,Tay96c} for a proof ---see also Theorem 1 in \cite{GarVal08c}.  

\begin{remark}
\emph{ Recall that an equation of the
above form are said to be \textit{homogeneous in} $\underline{u}$
\textit{and its first derivatives} if
$h\left(\lambda\underline{u},~\lambda\partial\underline{u}\right)=\lambda
h\left(\underline{u},~\partial\underline{u}\right)$ for all
$\lambda\in\mathbb{C}$. }
\end{remark}



%% \subsection{The alternative conformal Einstein field equations}
%% \label{AltCFEs}
%% %
%% %
%% In the current formulations of the conformal Einstein field equations
%% the rescaled Weyl tensor $d^{a}{}_{bcd}$ plays a central role in the
%% discussion.  Nevertheless in \cite{Pae13} an alternative approach was 
%% outlined, whereby the central object of interest is the Weyl tensor 
%% itself. In doing so, one must also introduce the Cotton tensor as an 
%% unknown. In \cite{Pae13} a set of wave equations for these unknowns is 
%% constructed. Here we follow a similar approach, but rather than 
%% deriving second-order equations for the conformal fields, we will obtain 
%% a closed system of equations which are (apart from the equation for the
%% conformal factor, \eqref{CEFEsecondderivativeCFAlt}) of first-order. We 
%% will call the resulting equations, along with their
%% spinorial equivalent, the \emph{alternative CFEs}.   


%% \medskip

%% Considering $\nabla^{a}\delta_{abc}=0$, with $\delta_{abc}$ as given
%% in in expression \eqref{standardCEFESchouten}, one obtains the
%% following wave equation for the Schouten tensor
%% \begin{equation}\label{WaveEqSchoutenForAlternativeCEFE}
%% \square L_{bc} = 4 L_{b}{}^{a} L_{ca} - L_{ad} L^{ad} g_{bc} - 2
%% L^{ad} C_{bacd} + \tfrac{1}{6} \nabla_{c}\nabla_{b}R.
%% \end{equation}
%% Recalling the definition of the Cotton tensor in terms of the Schouten
%% tensor
%% \begin{equation}\label{DefCotton}
%% Y_{abc} = 2 (- \nabla_{a}L_{bc} + \nabla_{b}L_{ac})
%% \end{equation}
%%  and using equations \eqref{WaveEqSchoutenForAlternativeCEFE} and
%%  \eqref{DefCotton} a computation shows that
%% \begin{equation} \label{DivY}
%% \nabla_{a}Y_{b}{}^{a}{}_{c} = -2 L^{ad} C_{bacd}.
%% \end{equation}
%% Therefore, to close the system one needs to find an equation for the
%% Weyl tensor. To do so, one can use the second Bianchi identity
%% \[
%% \nabla_{[a}R_{bf]c}{}^{d} = 0
%% \]
%% and the decomposition of the Riemann tensor in terms of the Weyl and
%% Schouten tensors
%% \[
%% R_{abc}{}^{d} = \delta_{b}{}^{d} L_{ac} - \delta_{a}{}^{d} L_{bc} +
%% L_{b}{}^{d} g_{ac} - L_{a}{}^{d} g_{bc} + C_{abc}{}^{d},
%% \]
%%  to obtain
%% \begin{equation}\label{SecondBianchiTemporaryBetter}
%%  \nabla_{[a}C_{bf]c}{}^{d} - 2 g_{[a|c|}\nabla_{b}L_{f]}{}^{d} + 2
%%  g_{[a}{}^{d}\nabla_{b}L_{f]c}=0.
%% \end{equation}
%% Using equations \eqref{SecondBianchiTemporaryBetter} and
%% \eqref{DefCotton} one can rewrite equation
%% \eqref{SecondBianchiTemporaryBetter} as
%% \begin{equation}\label{SecondBianchiForAlt}
%%  \nabla_{[a}C_{bf]c}{}^{d} = \tfrac{1}{2}Y_{[ab|c|}g_{f]}{}^{d} -
%%  \tfrac{1}{2} Y_{[ab}{}^{d}g_{f]c}.
%% \end{equation}
%% Consequently, one can replace the zero-quantities associated with the
%% rescaled Weyl tensor using equations \eqref{SecondBianchiForAlt},
%% \eqref{DefCotton} and \eqref{DivY}, obtaining then an alternative
%% version of the conformal Einstein field equations. The equations
%% so obtained are encoded in the vanishing of the following zero-quantities.
%% \begin{subequations}
%% \begin{eqnarray}
%% && Z_{ab} \equiv \nabla_{a}\nabla_{b}\Xi +\Xi L_{ab} - s g_{ab}=0 ,
%%  \label{CEFEsecondderivativeCFAlt}\\
%% && Z_{a} \equiv \nabla_{a}s +L_{ac} \nabla ^{c}\Xi=0
%%  , \label{CEFEsAlt}\\ && Z \equiv \lambda - 6 \Xi s + 3 \nabla_{a}\Xi
%%  \nabla^{a}\Xi,
%% \label{standardCFEconstraintFriedrichScalarAlt}\\
%% && \Delta_{bac} \equiv
%% \nabla_{b}L_{ac}-\nabla_{a}L_{bc}-\frac{1}{2}Y_{abc}, \label{CEFESchouten}
%% \\ && \Pi_{bc} \equiv \nabla_{a}Y_{b}{}^{a}{}_{c}+2
%% L^{ad}C_{bacd}, \label{CEFECotton}\\ && \Lambda_{abcd}{}^{e}=
%% 2\nabla_{[a}C_{bc]d}{}^{e} - Y_{[ab|d|}g_{c]}{}^{e} +
%% Y_{[ab}{}^{e}g_{c]d}
%% \label{CEFELambda}
%% \end{eqnarray}
%% \end{subequations}
%% Notice that in this alternative representation of the conformal
%% Einstein field equations the rescaled Weyl tensor does not
%% appear. Instead, the Weyl tensor $C_{abcd}$ and the Cotton tensor
%% $Y_{abc}$ are now part of the unknowns. Observe that the definition of
%% the Cotton tensor in terms of derivatives of the Schouten tensor is
%% encoded in equation \eqref{CEFESchouten}.  A solution to the alternative
%% conformal Einstein field equations consists of a collection of fields
%% \begin{equation}\label{UnknownsCFE}
%% \{g_{ab}, \; \Xi,\;\nabla_{a}\Xi \;,s\;,L_{ab},\; C_{abcd},\; Y_{abc}\}
%% \end{equation}
%% satisfying
%% \[ Z_{ab}=0, \quad Z_{a}=0, \quad \Delta_{abc}=0, \quad \Pi_{bc}=0,\quad \Lambda_{abcde}=0.   \]
%% \begin{remark}\emph{
%% Note that, by construction, any solution to the (alternative) CFEs
%% with $\Xi= 1$ corresponds to a solution of the Einstein field
%% equations. Conversely, given a solution to the Einstein field
%% equations, there corresponds a family of conformally-related solutions
%% to the (alternative) CFEs.}
%% \end{remark}

%% In view of the subsequent analysis of the Killing spinor equation it
%% is convenient to formulate the above system in spinorial form. Similar
%% to the case of the standard conformal Einstein field equations, the
%% spinorial formulation allows one to identify in a clearer way the
%% structure of the equations.  To obtain the spinorial formulation of
%% the the zero-quantities
%% \eqref{CEFEsecondderivativeCFAlt}-\eqref{CEFELambda} recall that the
%% spinorial counterpart of the Weyl tensor can be decomposed as
%% \[
%% C_{AA'BB'CC'DD'}=\bar{\Psi}_{A'B'C'D'} \epsilon_{AB} \epsilon_{CD} +
%% \Psi_{ABCD} \bar{\epsilon}_{A'B'} \bar{\epsilon}_{C'D'},
%% \]
%% where $\Psi_{ABCD}$ is the Weyl spinor.  Similarly, the irreducible
%% decomposition of the Cotton spinor $Y_{AA'BB'CC'}$ given by
%% \begin{equation}
%% Y_{AA'BB'CC'} = \bar{Y}_{A'B'C'C} \epsilon_{AB} + Y_{ABCC'}
%% \bar{\epsilon}_{A'B'},
%% \end{equation}
%% where
%% \begin{equation}
%% Y_{ABCC'} = \tfrac{1}{2} Y_{(A}{}_{|Q'|B}{}^{Q'}{}_{C)C'}.
%% \end{equation}
%% Additionally, the Schouten spinor $L_{AA'BB'}$ can be expressed in
%% terms of the tracefree Ricci spinor $\Phi_{AA'BB'}$ and
%% $\Lambda=\frac{1}{24}R$;
%% \begin{equation}
%% L_{AA'BB'}=-\Phi_{ABA'B'}+ \Lambda \epsilon_{AB}\epsilon_{A'B'} .
%% \end{equation} 
%% With these decompositions at hand, the spinorial formulation of the
%% above equations can be expressed as
%% \begin{equation}\label{CFEZeroQuantitiesEqualToZero}
%% Z_{AA'BB'}=0, \quad Z_{AA'}=0, \quad \Delta_{ABCC'}=0, \quad
%% \Pi_{BB'CC'}, \quad \Lambda_{C'BCF}=0,
%% \end{equation}
%% where
%% \begin{subequations}
%% \begin{flalign}
%% & Z_{AA'BB'} \equiv \nabla_{BB'}\nabla_{AA'}\Xi - \Xi
%%   \Phi_{ABA'B'}\epsilon_{A'B'} + \Xi \Lambda
%%   \epsilon_{AB}\epsilon_{A'B'} - s \epsilon_{AB}\epsilon_{A'B'},
%% \label{ZeroQuantitySecondDerivativeConformalFactor}  \\
%% & Z_{AA'} \equiv \nabla_{AA'}s - \Phi_{AA'}{}^{BB'} \nabla_{BB'}\Xi +
%% \Lambda\nabla_{AA'}\Xi, \label{ZeroQuantityDerivativeFriedrichScalar}
%% \\ &\Delta_{ABCC'} \equiv - Y_{ABCC'} + \nabla_{AA'}\Phi_{BCC'}{}^{A'} +
%% \epsilon_{BC} \nabla_{AC'}\Lambda + \nabla_{BA'}\Phi_{ACC'}{}^{A'} +
%% \epsilon_{AC} \nabla_{BC'}\Lambda,
%% \label{ZeroQuantitySchouten}\\
%% & \Pi_{AA'BB'} \equiv -2 \Phi^{CD}{}_{A'B'} \Psi_{ABCD} + -2
%% \Phi_{AB}{}^{C'D'} \overline{\Psi}_{A'B'C'D'} +
%% \nabla_{CC'}Y_{AA'}{}^{CC'}{}_{BB'} ,\label{ZeroQuantityCotton}
%% \\ &\Lambda_{C'BCF}
%% \equiv - \tfrac{1}{2} Y_{BCFC'} + \nabla_{AC'}\Psi_{BCF}{}^{A}.
%% \label{ZeroQuantityWeyl}
%% \end{flalign}
%% \end{subequations}
%% Notice that the zero-quantity $\Pi_{AA'BB'}$ can alternatively be
%% written in terms of the reduced Cotton spinor $ Y_{ABCC'}$ as
%% \begin{equation*}
%% \Pi_{AA'BB'} = -2 \Phi^{CD}{}_{A'B'} \Psi_{ABCD} - 2
%% \Phi_{AB}{}^{C'D'} \bar{\Psi}_{A'B'C'D'} +
%% \nabla_{AC'}\bar{Y}_{A'}{}^{C'}{}_{B'B} +
%% \nabla_{CA'}Y_{A}{}^{C}{}_{BB'}.
%% \end{equation*}
%% Furthermore, observe that a trace of the latter equation implies
%% \begin{equation*}
%%  \Pi^{A}{}_{A'AB'}= -\nabla_{AC'}\bar{Y}_{A'B'}{}^{C'A} .
%% \end{equation*}
%% Its worth noticing that in the formulation of the conformal Einstein
%% field equations, the Ricci scalar $\Lambda$ is not part of the
%% unknowns as it represents, the so-called, the \emph{conformal gauge
%%   source function} ---see \cite{Fri83, Fri91, CFEbook} for further
%% discussion.  The geometric meaning of the zero-quantities
%% \eqref{ZeroQuantitySecondDerivativeConformalFactor}-\eqref{ZeroQuantityWeyl}
%% is analogous to their tensorial counterparts.  In particular, the
%% Bianchi identities may be recovered from the alternative conformal
%% Einstein field equations by taking suitable contractions of the
%% zero-quantities $\Delta_{ABCC'}$ and $\Lambda_{ABB'C}$:
%% \begin{subequations}
%% \begin{eqnarray}
%% && \nabla_{AA'}\Phi_{BC'}{}^{AA'}  + 3 \nabla_{BC'}\Lambda =-\Delta_{B}{}^{A}{}_{AC'}
%% \label{BianchiIdentityRicciZeroQuantities} \\
%% && \nabla_{AC'}\Psi_{BCF}{}^{A} + \nabla_{(B}{}^{Q'}\Phi_{CF)C'Q'} = - \tfrac{1}{2} \Delta_{(BCF)C'} + \Lambda_{C'(BCF)}
%% \label{BianchiIdentityWeylZeroQuantities}
%% \end{eqnarray}
%% \end{subequations}


%% By \emph{initial data for the alternative CFEs} we mean the restriction to an
%% hypersurface $\mathcal{S}\subset \mathcal{M}$ of a collection of
%% fields \eqref{UnknownsCFE}, satisfying the constraint equations
%% implied by (\ref{CFEZeroQuantitiesEqualToZero}).
%% It will not be necessary for our purposes to study the constraints
%% equations in detail; indeed, the only
%% constraint that will be of interest ---see Section
%% \ref{Sec:IntrinsicConditions}--- is the following
%% \[\mathcal{D}^{PQ}\Psi_{ABPQ}-\tfrac{1}{2}Y_{AB}{}^{Q}{}_{Q}=0\]
%% which follows from $\Lambda_{A'ABC}=0$.  Since the (alternative)
%% conformal Einstein field equations imply a solution to the Einstein
%% field equations \eqref{EFEVacuum} whenever $\Xi\neq 0$, we will refer
%% to the development such an initial data set simply as a
%% \emph{spacetime development}.


{\color{blue}
  %\section{Conformally-Einstein vacuum twistor initial data}
  \section{Conformal twistor initial data}

\subsection{Twistor zero-quantities}
\label{Sec:TwistorZeroQuantities}

For the following discussion is convenient to make the following
\emph{zero-quantities}
\begin{subequations}
  \begin{eqnarray}
   && H_{A'AB} \equiv 2
    \nabla_{A'(A}\kappa_{B)},\label{Def_H_twistor}\\ && B_{ABC}
    \equiv \phi_{ABCD}\kappa^D.\label{Def_B_twistor}
    \end{eqnarray}
\end{subequations}
The spinors $H_{A'AB}$ and $B_{ABC}$ will be denoted in index free
notation as $\bmH$ and $\bmB$ and will be called the twistor
zero-quantity and the Buchdahl zero-quantity respectively.  The
Buchdahl zero-quantity arises as an integrability condition of the
twistor equation.  To see this, notice that, taking the following
derivative of $\bmH$ and substituting the definition
\eqref{Def_H_twistor} one obtains
  \begin{equation}\label{curl_H_twistor}
  \nabla_{AA'}H^{A'}{}_{BC}= 2 \nabla_{AA'}\nabla_{(B}{}^{A'}\kappa
  _{C)} = \tfrac{1}{2} \epsilon _{AB} \square \kappa _{C}  +
  \tfrac{1}{2}  \epsilon _{AC} \square \kappa _{B} +
  \square_{BA}\kappa _{C} + \square_{CA}\kappa _{B}.
  \end{equation}
  Symmetrsing and using equation \eqref{SpinorialRicciIdentities1} renders
  \[
  \nabla_{(A|A'|}H^{A'}{}_{BC)}= - 2\Psi_{ABCD}\kappa^D.
  \]
  The vanishing of the right-hand side of latter equation encodes the
  Buchdahl constraint, namely the fact that if $(\mathcal{M},\bmg)$
  admits a twistor then it is necesarilly of Petrov type N or O. To write this in the variables appearing in the  conformal Einstein field equations, using the definition of the
  rescaled Weyl spinor yields
  \begin{equation}\label{Curl_H_sym_toB_twistor}
  \nabla_{(A}{}^{A'}H_{|A'|BC)} = 2\Xi B_{ABC},
  \end{equation}
  which motivates the name for the zero-quantity $\bmB$.
  Thus, with this notation, it is clear that if the unphysical spacetime
  $(\mathcal{M},\bmg)$ admits a twistor (valence-1 Killing spinor) then following
  zero-quantities vanish
  \begin{equation}
H_{A'AB}=0, \qquad B_{ABC}=0.
  \end{equation}
  \subsection{Twistor auxiliary quantities and the twistor candidate equation}
Other useful definition to keep track of the subsequent
calculations is the following \emph{auxiliary quantities}
\begin{subequations}
  \begin{eqnarray}
      && Q_{A}  \equiv \nabla^{QA'}H_{A'QA} \label{def_Q_twistor} \\
      && \xi_{A'} \equiv \nabla^B{}_{A'}\kappa_B \label{def_xi_twistor}
  \end{eqnarray}
\end{subequations}
  The \emph{auxiliary spinor} $\xi_{A'}$ is merely a convenient
  placeholder for making irreducible decompositions of derivatives of
  $\kappa_A$ such as
  \begin{align}\label{decomp_Der_kappa}
    \nabla_{AA'}\kappa _{B} & = \tfrac{1}{2} \epsilon _{AB}
    \nabla_{CA'}\kappa ^{C} + \nabla_{(A|A'|}\kappa _{B)}
 \\ & = \tfrac{1}{2} H_{A'AB} - \tfrac{1}{2} \xi _{A'} \epsilon_{AB}.
  \end{align}
  and in principle one can carry out all the calculations without this
  definition. It is nevertheless illustrative to introduce this
  shorthand since the analogous quantity in the Killing spinor case
  ($q=2$) will have some geometrical significance.

  \medskip
  
  On the other hand, the \emph{auxiliary quantity}
  $Q_A$ will be central for the following discussion since it
  encodes a wave equation for $\kappa_A$. To see this, observe that
  tracing the identity \eqref{curl_H_twistor} and substituting the
  definition \eqref{def_Q_twistor} gives,
\begin{equation}\label{Q_to_box_twistor_candidate}
Q_{A} = 3 \Lambda \kappa _{A} + \tfrac{3}{2} \square \kappa _{A}.
\end{equation}
Solving for $\square \kappa _{A}$ one has
\[
\square \kappa _{A} = \tfrac{2}{3} Q_{A} -2 \Lambda \kappa _{A}.
\]
If the equation $Q_{A}=0$ is imposed, then
the latter expression can be read as a wave equation for $\kappa_A$.
This motivates the following definition: a valence-1 spinor $\eta_A$ satisfying
\begin{align} \label{Wave_eq_twistor_candidate}
\square \eta _{A} = -2 \Lambda  \eta _{A}
\end{align}
will be called a \emph{twistor candidate}. To understand the
motivation for this definition and its name, notice that in general,
any twistor $\kappa_A$ trivially satisfies the twistor candiadate
equation but not every twistor candidate $\eta_A$ will solve the
twistor equation. In other words,
\[
\bmH=0 \implies \bmQ =0, \qquad \text{but in general} \qquad \bmQ =0 \notimplies \bmH=0.
\]
However,  the initial data $(\nabla_\tau
\eta_A, \eta_A)|_{\mathcal{S}}$ for the wave equation
\eqref{Wave_eq_twistor_candidate} has not been fixed yet. The aim of the following
calculations is to determine the conditions on the initial data for the
twistor candidate such that if propagated off $\mathcal{S}$, using
equation \eqref{Wave_eq_twistor_candidate}, then the corresponding twistor
candidate $\eta_A$ is, in fact, a twistor. Namely,
\begin{equation}
\bmQ =0 \;\;\&\;\; \text{twistor initial data} \;\;\implies \bmH=0.
\end{equation}
The strategy to obtain such conditions on the intial data
$(\nabla_\tau \eta_A, \eta_A)|_{\mathcal{S}}$  is to derive a closed
system of homogeneous wave equations for the zero-quantities $\bmH$
and $\bmB$ to show that, if trivial initial data for such
equations is given, then, using Theorem \ref{TheoremHomogeneousWave},
$\bmH=0$ and $\bmB=0$ in the domain of dependence of the data.

\subsection{Wave equations for the zero-quantities}

A wave equation for the zero-quantity $\bmH$ can be constructed as
follows.  From the irreducible decomposition of
$\nabla_D{}^{A'}H_{A'AB}$,
\[
\nabla_{D}{}^{A'}H_{A'AB} = \tfrac{1}{3} \epsilon _{BD}
\nabla_{CA'}H^{A'}{}_{A}{}^{C} + \tfrac{1}{3} \epsilon _{AD}
\nabla_{CA'}H^{A'}{}_{B}{}^{C} + \nabla_{(A}{}^{A'}H_{|A'|BD)},
\]
and the definitions \eqref{def_Q_twistor} and equation
\eqref{Curl_H_sym_toB_twistor} one has that
\begin{align}\label{derH_twistor_toBandQ}
\nabla_{D}{}^{A'}H_{A'AB} = 2 B_{ABD} \Xi + \tfrac{1}{3} Q_{B}
\epsilon _{AD} + \tfrac{1}{3} Q_{A} \epsilon _{BD}
\end{align}
Applying $\nabla_{D}{}^{B'}$ to the last expression, and using the
identity \eqref{DecomposeDoubleDerivativeContracted} along with the
spinorial Ricci identities
\eqref{SpinorialRicciIdentities1}-\eqref{SpinorialRicciIdentities2},
renders
\begin{equation}\label{wave_H_twistor}
  \square H_{B'AB} = 6 \Lambda H_{B'AB} + 4 \Xi
  \nabla_{DB'}B_{AB}{}^{D} -4 B_{ABD} \nabla^{D}{}_{B'}\Xi -4
  \Phi_{(A}{}^{D}{}_{|B'}{}^{A'}H_{A'|B)D} + \tfrac{4}{3}
  \nabla_{(A|B'|}Q_{B)}
\end{equation}

\noindent To derive a wave equation for $\bmB$, one applies the D'Alembertian operator
$\square$ to the definition in equation \eqref{Def_B_twistor} to obtain
\begin{align}\label{pre_wave_B_twistor}
\square B_{ABC} = \kappa ^{D} \square \phi _{ABCD} + \phi _{ABCD}
\square \kappa ^{D} + 2 \nabla_{FA'}\phi _{ABCD} \nabla^{FA'}\kappa
^{D}
\end{align}
Substituting the definition \eqref{Def_B_twistor}, the identity
\eqref{Q_to_box_twistor_candidate}, and the wave equation satisfied by
the rescaled Weyl spinor \eqref{Wave_eq_CFE_Weyl} into the last expression gives
\begin{equation}\label{wave_B_twistor}
\square B_{ABC} = 10 B_{ABC} \Lambda + H^{A'DF} \nabla_{FA'}\phi _{ABCD}  -6 \Xi B_{(A}{}^{DF}\phi
_{BC)DF} + \tfrac{2}{3} \phi _{ABCD} Q^{D}
\end{equation}
Observe that if $Q_{A}=0$, namely if the twistor candidate wave equation is imposed then,
$\bmH$ and $\bmB$ satisfy the following set of wave equations
\begin{subequations}
\begin{eqnarray}
  && \square H_{B'AB} = 6 \Lambda H_{B'AB} + 4 \Xi
  \nabla_{DB'}B_{AB}{}^{D}  -4 B_{ABD} \nabla^{D}{}_{B'}\Xi   -4 \Phi_{(A}{}^{D}{}_{|B'}{}^{A'}H_{A'|B)D}
   \label{Hom_wave_HandB1} \\
 && \square B_{ABC} = 10 B_{ABC} \Lambda + H^{A'DF} \nabla_{FA'}\phi _{ABCD}  -6 \Xi B_{(A}{}^{DF}\phi
_{BC)DF}  \label{Hom_wave_HandB2}
\end{eqnarray}
\end{subequations}
Notice that the only place where the CFEs ---in their wave equation form---
have been used was in equation \eqref{pre_wave_B_twistor}
to substitute the term $\square \phi _{ABCD}$.

The relevant observation about equations
\eqref{Hom_wave_HandB1}-\eqref{Hom_wave_HandB2} is that they
constitute a closed system of \emph{regular and homogeneous}
wave equations for $\bmH$ and $\bmB$.
Hence prescribing trivial intial data
\[
H_{A'AB}=0, \qquad \nabla_\tau H_{A'AB}=0, \qquad B_{ABC}=0, \qquad \nabla_\tau B_{ABC}=0 \qquad \text{on} \qquad \mathcal{S}
\]
and using Theorem \ref{TheoremHomogeneousWave}, which stablishes the
uniquess of solutions to wave equations of the type
of \eqref{Hom_wave_HandB1}-\eqref{Hom_wave_HandB2}, on has that 
\begin{equation}\label{ID_trivial_H_B_twistor}
H_{A'AB}=0, \qquad B_{ABC}=0, \qquad \text{on} \qquad \mathcal{D}^{+}(\mathcal{S}) .
\end{equation}
In turn, substituting the definitions for the zero-quantities $\bmH$
and $\bmB$ into the conditions \eqref{ID_trivial_H_B_twistor} render a
prescription to fix the intial data $(\nabla_\tau \eta_A,
\eta_A)|_\mathcal{S}$ for  twistor
candidate wave equation \eqref{Wave_eq_twistor_candidate} that ensures that
the twistor candidate $\eta_A$ corresponds to an actual twistor $\kappa_A$.

This discussion is summarised in the following
\begin{proposition}\label{Prop:Propagation_twistor}
  Given initial data for the conformal field equations on $\mathcal{U}\subseteq\mathcal{S}$
  where $\mathcal{S}$ is a spacelike
hypersurface $\mathcal{S}$ with normal vector $\tau^{AA'}$, and
associated normal derivative $\nabla_\tau \equiv
\tau^{AA'}\nabla_{AA'}$, the corresponding spacetime development
admits a twistor (valence-1 Killing spinor) in $\mathcal{D}^{+}(\mathcal{U})$ ---the future domain of dependence of $\mathcal{U}$---  if and only if
\begin{subequations}
\begin{eqnarray}
  && H_{A'AB}=0,\label{eq:VanishingOfH_twistor}\\ && \nabla_\tau
  H_{A'AB}=0,\label{eq:VanishingOfNormalDerivH_twistor}\\ &&
  B_{ABC}=0,\label{eq:VanishingOfB_twistor}\\ &&\nabla_\tau
  B_{ABC}=0 \label{eq:VanishingOfNormalDerivB_twistor}
\end{eqnarray}
\end{subequations}
 hold on $\mathcal{U}$.
\end{proposition}
\begin{proof}
The \emph{only if} direction is immediate. Suppose, on the other hand,
that
\eqref{eq:VanishingOfH_twistor}-\eqref{eq:VanishingOfNormalDerivB_twistor}
hold on some $\mathcal{U}\subset\mathcal{S}$ ---that is to say, there
exist a spinor field $\kappa_{A}$ for which
\eqref{eq:VanishingOfH_twistor}-\eqref{eq:VanishingOfNormalDerivB_twistor}
are satisfied on $\mathcal{U}$. The latter is then used as initial
data for the twistor candidate wave equation
\begin{align} \label{Wave_eq_twistor_candidate_prop}
\square \kappa _{A} = -2 \Lambda \kappa _{A}
\end{align}
As the zero-quantities $H_{A'AB},~B_{ABC}$ satisfy the homogeneous
wave equations \eqref{Hom_wave_HandB1}-\eqref{Hom_wave_HandB2} then
the uniqueness result for homogeneous wave equations, given in
Theorem \ref{TheoremHomogeneousWave},
ensures that
\[ H_{A'ABC}=0,\qquad B_{ABC}=0,\]
in $\mathcal{D}^{+}(\mathcal{U})$. In other words,
$\kappa_{A}$ solves the twistor equation on $\mathcal{D}^{+}(\mathcal{U})$.
\end{proof}


\subsection{Comparisson between the Killing initial data conditions
  in the physical and unphysical pictures}

The main advantage of the conformal (unphysical) approach to the
Einstein field equations is that the conformal boundary $\mathscr{I}$
determined by $\Xi=0$ is a submanifold of $(\mathcal{M},\bmg)$. This
allows, in particular, to consider the $\Xi=0$ hypersurface as a
legitimate hypersurface to prescribe data which can be evolved using
regular ---without $\Xi^{-1}$-terms--- evolution equations.  This set up is
particularly atractive to study spacetimes with $\lambda>0$ in which
---given the appropriate conditions--- the conformal boundary $\mathscr{I}$
is a spacelike hypersurface and hence one can pose \emph{an asymptotic
initial value problem}: an initial value problem where the initial
hypersurface is $\mathscr{I}$.
On the other hand, the conformal (valence-1 Killing spinor) twistor conditions
of proposition \eqref{Prop:Propagation_twistor} allows to identify
asymptotic initial data  whose development will
contain a twistor. Although the twistor case is too restrictive to
characterise black hole spacetimes
%%%%
% (since the Buchdahl constraint forces the manifold $(\bmg, \mathcal{M})$
% to be of Petrov type N orO),
%%%
it is still illustrative to compare the derivation of the physical
twistor initial data conditions on
$(\tilde{\mathcal{M}},\tilde{\bmg})$ and that leading to proposition
\eqref{Prop:Propagation_twistor}.

For the twistor case, one important difference between discussion in
\cite{GasVal15} using the vacuum Einstein field equations in  $(\mathcal{M},\bmg)$
is that the system closes with $\tilde{H}_{A'AB}$ alone and there is no need
to introduce the analogous physical Buchdahl zero-quantity $\tilde{B}_{ABC}$.
Therefore it is interesting to check if in the conformal case discussed
in the previous sections one can also close the system with $H_{A'AB}$ alone.

Applying the D'Alembertian $\square$ to equation
\eqref{Def_H_twistor}, using the definition of the auxiliary quantity
$Q_A$ in equation \eqref{def_Q_twistor}, a direct calculation exploiting the
identities \eqref{SpinorialRicciIdentities1}-\eqref{SpinorialRicciIdentities2}
and \eqref{DecomposeDoubleDerivativeContracted}, gives

\begin{align}\label{Box_H_alone}
\square H_{A'AB} = & -2 \Psi _{ABCD} H_{A'}{}^{CD} + 6 \Lambda
H_{A'AB} -4 \Phi _{(A}{}^{C}{}_{|A'}{}^{B'}H_{B'|B)C} \nonumber \\ &
-2 \kappa _{(A}\nabla_{B)A'}\Lambda -2
\kappa^{C}\nabla_{(A}{}^{B'}\Phi _{B)CA'B'} + 2 \kappa ^{C}
\nabla_{DA'}\Psi _{ABC}{}^{D} + \tfrac{4}{3} \nabla_{(A|A'|}Q_{B)}.
\end{align}
If one were discussing the physical case ---adding a tilde to every
term in \eqref{Box_H_alone}--- in which the fields are defined on
$(\tilde{\mathcal{M}},\tilde{\bmg})$ which satisfies the vacuum
Einstein field equations
\begin{align}\label{vaccumEFE}
  \tilde{\Lambda}=0, \qquad \tilde{\Phi}_{AA'BB'}=0,
\end{align}
then, using the Bianchi identity $\tilde{\nabla}^{A}{}_{B'}\Psi
_{ABCD}=\tilde{\nabla}^{A'}{}_{(B}\tilde{\Phi}_{CD)A'B'}$ and
equations \eqref{vaccumEFE},
% it follows
%from the physical version of
%equation \eqref{Box_H_alone}
the physical version of equation \eqref{Box_H_alone} reduces to
\begin{align}\label{Box_H_alone}
  \tilde{\square} \tilde{H}_{A'AB} = & -2 \Psi _{ABCD} \tilde{H}_{A'}{}^{CD}
  + \tfrac{4}{3} \tilde{\nabla}_{(A|A'|}\tilde{Q}_{B)}.
\end{align}
Hence, imposing $\tilde{Q}_A=0$, the system
closes with $\tilde{H}_{A'AB}$ alone.


\medskip

In the other hand, if one tries to follow the same strategy in the unphysical set up
---with $(\mathcal{M},\bmg)$ satisfying the CFEs--- one ends up with a
formally singular equation. To see this, observe that starting from
the identity \eqref{Box_H_alone} and using equation \eqref{Def_rescaled_Weyl_spinor}
along the CFEs zero-quantities
\[ \delta_{ABCC'}=0, \qquad \Lambda_{CC'AB}=0,\]
as defined in equations \eqref{Def_delta_CFE_zeroquant}-\eqref{Def_Lambda_CFE_zeroquant},
a calculation gives

\begin{align}\label{WaveH_twistor_singular}
  \square H_{A'AB} = & - \frac{2 \nabla^{C}{}_{A'}\Xi
    \nabla_{(A}{}^{B'}H_{|B'|BC)}}{\Xi } + 6 \Lambda H_{A'AB} -2 \Xi
  \phi _{ABCD}H_{A'}{}^{CD} -4
  \Phi_{(A}{}^{C}{}_{|A'}{}^{B'}H_{B'|B)C} \nonumber \\ & +
  \tfrac{4}{3} \nabla_{(A|A'|}Q_{B)}.
\end{align}

%\noindent
Hence, setting $Q_{A}=0$, renders an homogeneous but
\emph{singular equation} ---due to the $\Xi^{-1}$ coefficient---
for $H_{A'AB}$, for which the theory of behind Theorem
\ref{TheoremHomogeneousWave} does not apply. Arguably, one could try
to use the theory of Fuchsian systems to see if the analogous of
Theorem \ref{TheoremHomogeneousWave} applies for the singular equation
\eqref{WaveH_twistor_singular}.  However, one of the advantages of the
conformal approach of the CFEs respect to other approaches to include
$\mathscr{I}$ is that one deals with formally regular equations.
Therefore, from this perspective,
%%%%%%
%, in the spirit of the CFEs,
%%%%%
it is preferable to work with explicitly regular equations and hence,
it is necessary to introduce $B_{ABC}$ as a further zero-quantity to
be propagated.  A analogous observation holds for the conformal valence-2
Killing spinor initial data discussion of the following sections,
where to close the system in a regular way, one needs to introduce not
only the Buchdahl zero-quantity but also its derivative.

%% 
%% Making formally identical definition as that of \eqref{Def_H_twistor}
%% to define the physical twistor zero-quantity $\tilde{H}_{A'AB}$ and applying $\square$
%% one obtains

%% \[
%% \square H_{A'AB} =-2 \square_{(A}{}^{C}\nabla_{|CA'|}\kappa _{B)}  -2 \square_{A'}{}^{B'}\nabla_{(A|B'|}\kappa _{B)} + 2 \nabla^{C}{}_{A'}\square_{(A|C|}\kappa _{B)} + 2 \nabla_{(A}{}^{B'}\square_{|A'B'|}\kappa _{B)}
%% + 2 \nabla_{(A|A'|}\square \kappa_{B)}
%% \]
%% and the physical auxiliary quantity $\tilde{Q}_A$



\subsection{Intrinsic conformal twistor initial data conditions}

In this section,  the conformal twistor initial data conditions
of proposition \ref{Prop:Propagation_twistor} are written in terms
of intrinsic quantities at $\mathcal{S}$.
To understand the need of the calculation to be carried out in this section
observe that although the conditions of proposition \ref{Prop:Propagation_twistor} are given
on $\mathcal{S}$ it contains not only derivatives tangential to $\mathcal{S}$ but
also normal to it. Hence, to obtain genuine intrinsic conditions on $\mathcal{S}$
one needs to remove these normal derivatives.

%\subsubsection{The $H|_{\mathcal{S}}=0$}

Using the following definitions,
\begin{align}
  %H_{CAB} \eqref \tau _{C}{}^{A'} H_{A'AB}, \qquad
  %\mathcal{H} _{ABC} \equiv H_{(ABC)},
  %\qquad \mathcal{H} _{A} \equiv H^{D}{}_{AD}.
  \mathcal{H} _{ABC}  \equiv \tau _{(A}{}^{A'}H_{|A'|BC)}, \qquad
  \mathcal{H}_{A}  \equiv  \tau^{QA'} H_{A'AQ},
\end{align}
the space spinor split of $H_{A'AB}$ reads,
\begin{align}
  H_{A'AB} = - \tfrac{1}{2} \tau ^{C}{}_{A'} \mathcal{H} _{ABC}  -
  \tfrac{1}{6} \tau ^{C}{}_{A'} \mathcal{H} _{B} \epsilon _{AC}  -
  \tfrac{1}{6} \tau ^{C}{}_{A'} \mathcal{H} _{A} \epsilon _{BC}
\end{align}
Hence, the space spinors $\mathcal{H} _{ABC}$ and $\mathcal{H}_{A}$
contain all the information of $H_{A'AB}$. In other words,
\[
H_{A'ABC}=0 \quad                   %|_{\mathcal{S}}=0
\iff \quad \mathcal{H} _{A}=0      %|_{\mathcal{S}} =0
\quad
\& \quad \mathcal{H}_{ABC}=0  %|_{\mathcal{S}}=0
\]
Substituting the definition \eqref{Def_H_twistor} one obtains
\begin{align}\label{spacespinordecompHtotwistorders}
\mathcal{H} _{A} = \tfrac{3}{2} \nabla_\tau \kappa_{A} - \mathcal{D} _{AB}\kappa^{B}, \qquad \mathcal{H} _{ABC} = 2 \mathcal{D} _{(AB}\kappa _{C)},
\end{align}
Then $H_{A'AB}|_{\mathcal{S}}=0$  imposes then following conditions on the
the initial data
$(\kappa_A,\nabla_\tau\kappa_A)|_{\mathcal{S}}$
for the twistor candidate wave equation
\eqref{Wave_eq_twistor_candidate_prop}:
\begin{align}\label{H_twistor_vanishes_ID}
 \nabla_\tau \kappa _{A} = \tfrac{2}{3} \mathcal{D} _{AB}\kappa ^{B}, \qquad
 \quad \mathcal{D} _{(AB}\kappa _{C)}=0 \qquad \text{on} \qquad \mathcal{S}.
\end{align}
Another set of constraints arise from the conditions
$\nabla_\tau H_{A'BC}|_{\mathcal{S}}=0$ and $B_{ABC}|_{\mathcal{S}}=0$. These two
conditions can be analysed in tandem since $\bmB$ is related to the derivative of
$\bmH$. Using the space spinor split of $\nabla$ it follows from the
identity \eqref{derH_twistor_toBandQ} that
\begin{align}
  \tau _{D}{}^{A'}\nabla_\tau H_{A'AB}   -2 \tau ^{CA'} \mathcal{D} _{DC}H_{A'AB}
  = 4 B_{ABD} \Xi  + \tfrac{4}{3} Q_{(A}\epsilon _{B)D}\quad 
\end{align}
Hence, transvecting with $\tau^{D}{}_{B'}$ and rearranging gives
\begin{align}
\nabla_\tau H_{B'AB} = -4 B_{ABD} \Xi  \tau ^{D}{}_{B'}  -2 \tau ^{CA'} \tau ^{D}{}_{B'} \mathcal{D} _{DC}H_{A'AB}  - \tfrac{4}{3} \tau ^{D}{}_{B'}Q_{(A}\epsilon _{B)D}
\end{align}
Hence, if the  the twistor candidate wave equation is imposed, namely $Q_A=0$ then
\begin{align}
H_{A'AB}|_{\mathcal{S}}=0\quad \& \quad B_{ABC}|_{\mathcal{S}}=0 \implies \nabla_\tau H_{A'AB}|_{\mathcal{S}}=0.
\end{align}
In other words, imposing $\nabla_\tau H_{A'AB}|_{\mathcal{S}}=0$ is redundant if
$H_{A'AB}|_{\mathcal{S}}=0$ and $ B_{ABC}|_{\mathcal{S}}=0$. Using the definition \eqref{Def_B_twistor},
the condition $B_{ABC}|_{\mathcal{S}}=0$ simply reads,
\begin{align}
  \phi_{ABCD}\kappa^D=0 \qquad \text{on} \qquad \mathcal{S}.
\end{align}
Finally, for the condition $\nabla_{\tau}B_{ABC}|_{\mathcal{S}}=0$ one has, applying $\nabla_\tau$ to
equation \eqref{Def_B_twistor}
\begin{align}
\nabla_\tau B_{ABC} = \phi _{ABCD}\nabla_\tau \kappa ^{D}  + \kappa ^{D} \nabla_\tau \phi _{ABCD} 
\end{align}
At this point one can exploit the evolution equation for the rescaled
Weyl spinor \eqref{RescaledWeyl_evo_const} to subsitute for
$\nabla_\tau \phi_{ABCD}$ and condition \eqref {H_twistor_vanishes_ID}
to substitute $\nabla_\tau \kappa_A$ when evaluating at
$\mathcal{S}$. Thus
%the condition
%$\nabla_{\tau}B_{ABC}|_{\mathcal{S}}=0$
%reads
\begin{align}\label{normalderB_twistor_exp}
\nabla_{\tau}B_{ABC}|_{\mathcal{S}}= -2\kappa ^{D} \mathcal{D} _{DF}\phi _{ABC}{}^{F} + \tfrac{2}{3}  \phi
_{ABCD} \mathcal{D} ^{D}{}_{F}\kappa ^{F} = 0 \qquad \text{on} \qquad \mathcal{S}.
\end{align}
In fact, the latter expression can be rewritten in terms of $\mathcal{H}_{ABC}|_{\mathcal{S}}$
and $B_{ABC}|_{\mathcal{S}}$ as follows. Swapping indices $D$ and $A$ in equation
\eqref{normalderB_twistor_exp}, and exploiting the constraint equation for the rescaled Weyl spinor in expression
\eqref{RescaledWeyl_evo_const} renders
\begin{align}\label{normalderB_twistor_exp2}
\nabla_{\tau}B_{ABC}|_{\mathcal{S}}= -2 \kappa ^{D} \mathcal{D} _{AF}\phi _{DBC}{}^{F} + \tfrac{2}{3} \phi _{ABCD}
\mathcal{D} ^{D}{}_{F}\kappa ^{F} = 0 \qquad \text{on} \qquad
\mathcal{S}
\end{align}
Applying a $\mathcal{D}_{FQ}$ to the definition \eqref{Def_B_twistor}
and using the Leibnitz rule, one can replace the first term in the
last equation to obtain
\begin{align}\label{normalderB_twistor_exp3}
\nabla_{\tau}B_{ABC}|_{\mathcal{S}}= -2 \mathcal{D} _{AD}B_{BC}{}^{D} -2 \phi _{BCDF} \mathcal{D}
_{A}{}^{F}\kappa ^{D} +\tfrac{2}{3} \phi _{ABCF} \mathcal{D} _{D}{}^{F}\kappa ^{D}
= 0 \quad \text{on} \quad \mathcal{S}
\end{align}
From the irreducible decomposition of $\mathcal{D} _{AB}\kappa _{C}$
and using the expression for $\mathcal{H}_{ABC}$ equation
\eqref{spacespinordecompHtotwistorders} one has
\begin{align}\label{decompSenKappa}
\mathcal{D} _{AB}\kappa _{C} = \tfrac{1}{2} \mathcal{H} _{ABC} + \tfrac{1}{3} \epsilon _{BC} \mathcal{D} _{AD}\kappa ^{D} + \tfrac{1}{3} \epsilon _{AC} \mathcal{D} _{BD}\kappa ^{D}.
\end{align}
Substituting equation \eqref{decompSenKappa}
into equation \eqref{normalderB_twistor_exp3} one concludes one gets
\begin{align}
\nabla_{\tau}B_{ABC}|_{\mathcal{S}}=- \phi _{BCDF} \mathcal{H} _{A}{}^{DF} -2 \mathcal{D} _{AD}B_{BC}{}^{D} = 0 \quad \text{on} \quad \mathcal{S}
\end{align}

\noindent Hence, overall, the only independent conditions to be imposed
are $H_{A'AB}|_{\mathcal{S}}=0$ and $B_{ABC}|_{\mathcal{S}}=0$.







%\subsubsection{The $(\nabla_\tau H)|_{\mathcal{S}}=0$ and
% $B_{ABC}|_{\mathcal{S}}=0$ conditions}




  
}

\medskip

GOT HERE!!! 10.11.2021.

  


\section{Killing spinor zero-quantities}
\label{Sec:KillinSpinorZeroQuantities}


For the subsequent discussion it is convenient to introduce the
following spinors
\begin{subequations}
\begin{eqnarray}
&& H_{A'ABC} \equiv 3
  \nabla_{A'(A}\kappa_{BC)}, \label{DefZeroQuantityH}\\ && S_{AA'BB'}
  \equiv
  \nabla_{QA'}H_{B'}{}^{Q}{}_{AB}, \label{DefZeroQuantityS}\\ &&
  B_{ABCD} \equiv -\frac{1}{6}
  \nabla_{Q'(A}H^{Q'}{}_{BCD)}. \label{DefZeroQuantityB}
\end{eqnarray}
\end{subequations}
Observe that if $(\mathcal{M},\bmg)$ admits a Killing spinor
$\kappa_{AB}$ then, by definition, one has
\[
H_{A'ABC}=0, \qquad S_{AA'BB'}=0, \qquad B_{ABCD}=0.
\]
To see the geometric significance of the above defined zero-quantities
it is convenient to introduce the Hermitian spinor
\begin{equation}\label{DefAuxiliaryVector}
 \xi_{AA'} \equiv \nabla^{B}{}_{A'}\kappa_{AB}.
\end{equation}
Observe that the zero-quantity $S_{AA'BB'}$ can be written in terms of
$\xi_{AA'}$ as
\begin{equation}\label{DefinitionZeroQuantitySIntersmOfAuxiliaryV}
S_{CC'DD'}=-6\kappa_{(D}{}^{A}\Phi_{C)AC'D'} - \nabla_{CC'}\xi_{DD'} -
\nabla_{DD'}\xi_{CC'}.
\end{equation}
Notice that, if $\Phi_{ABA'B'}$ vanishes then $S_{AA'BB'}$ reduces to
the Killing vector equation and $\xi_{AA'}$ corresponds to the
spinorial counterpart of a Killing vector.  To clarify this point
further observe that, as a consequence of the conformal properties of
the Killing spinor equation, if $\kappa_{AB}$ is a Killing spinor in
the unphysical spacetime
then \[\tilde{\kappa}_{AB}=\frac{1}{\Xi^2}\kappa_{AB}\] is a Killing
spinor of the physical spacetime whenever $\Xi$ is not vanishing.  As
discussed before, the physical Killing spinor $\tilde{\kappa}_{AB}$
gives rise to a Killing vector $\tilde{\xi}_{a}$ whose spinorial
counterpart is
\begin{equation}
 \tilde{\xi}_{AA'} \equiv \tilde{\nabla}^{B}{}_{A'}\tilde{\kappa}_{AB}.
\end{equation}
Furthermore, if $\tilde{\xi}_{a}$ is a Killing 
vector in the physical spacetime
  then, 
\begin{equation}
X_{a} \equiv \Xi^2\tilde{\xi}_{a}
\end{equation}
corresponds to a conformal Killing vector for the unphysical
spacetime, namely, it can be verified that
\begin{equation}\label{ConfKillingEquation}
\nabla_{a}X_{b}+ \nabla_{b}X_{a}=\frac{1}{2}\nabla^{a}X_{a}g_{ab}.
\end{equation}
This conformal Killing vector additionally satisfy
\begin{equation}
\label{UnphysicalKillingCondition}
X^{a}\nabla_{a}\Xi=\frac{1}{4}\nabla_{a}X^{a}.
\end{equation}
Equations
\eqref{ConfKillingEquation}-\eqref{UnphysicalKillingCondition} are the
so-called \emph{unphysical Killing equations} ---see \cite{Pae14} for
a discussion on the unphysical Killing equations.  A direct
computation shows that given a Killing spinor in the unphysical
spacetime $\kappa_{AB}$, the concomitant
\begin{equation}
X_{AA'} \equiv \Xi \xi_{AA'}-3 \kappa_{AQ}\nabla_{A'}{}^{Q}\Xi
\end{equation}
corresponds to the spinorial counterpart of a conformal Killing vector
satisfying equations \eqref{ConfKillingEquation} and
\eqref{UnphysicalKillingCondition}.  However, in general, the
spinor $\xi_{AA'}$ has no straightforward interpretation and will be
regarded as an auxiliary variable for the subsequent discussion.
Additionally, observe that the introduction of $\xi_{AA'}$ allows one to
write the irreducible decomposition of the gradient of the Killing
spinor as
\begin{equation}\label{DecompositionDKappaAlt}
\nabla_{AA'}\kappa_{BC}=
\frac{1}{3}H_{A'ABC}-\frac{1}{3}\xi_{CA'}\epsilon_{AB}
-\frac{1}{3}\xi_{BA'}\epsilon_{AC}.
\end{equation}
On the other hand, a direct computation using the definition of
$H_{A'ABC}$ shows that the zero-quantity $B_{ABCD}$ encodes the
Buchdahl constraint, namely
\begin{equation}\label{BuchdahlInTermsOfWeylAndKillingSpinor}
B_{ABCD}= \kappa_{(D}{}^{F}\Psi_{ABC)F}.
\end{equation} 
To complete the discussion observe that if the Killing spinor equation
$H_{A'ABC}=0$ is satisfied, the Killing spinor $\kappa_{AB}$ and the
auxiliary spinor $\xi_{AA'}$ satisfy the following wave equations:
\begin{eqnarray} 
&& \square \kappa_{BC}=-4\Lambda\kappa_{BC} + \kappa^{AD}\Psi_{BCAD} ,
\label{WaveEquationKillingSpinor}\\
&& \square \xi_{AA'}=- \tfrac{4}{3} \Delta^{BC}{}_{BA'} \kappa_{AC} -
2 \Lambda_{A'A}{}^{BC} \kappa_{BC} - 6 \xi_{AA'} \Lambda - 2 \xi^{BB'}
\Phi_{ABA'B'}\nonumber \\ &&\qquad\qquad - \tfrac{3}{2} \kappa^{BC}
Y_{ABCA'} + \Psi_{ABCD} H_{A'}{}^{BCD} - 12 \kappa_{AB}
\nabla^{B}{}_{A'}\Lambda - \tfrac{1}{2} \kappa^{BC} \Delta_{(ABC)A'}
\label{WaveEquationAuxiliaryVariable}
\end{eqnarray}
The wave equation \eqref{WaveEquationKillingSpinor} is derived
considering the integrability condition $\nabla^{AA'}H_{A'ABC}=0$,
substituting equation \eqref{DefZeroQuantityH} and exploiting the
spinorial Ricci identities
\eqref{SpinorialRicciIdentities1}-\eqref{SpinorialRicciIdentities2}.
To obtain equation \eqref{WaveEquationAuxiliaryVariable} observe that
from equation \eqref{DefAuxiliaryVector} one has
\begin{equation}
\square
\xi_{AA'}=\nabla_{CC'}\nabla^{CC'}\nabla^{B}{}_{A'}\kappa_{AB}.
\end{equation}
Commuting covariant derivatives in the last expression renders
\[
\square \xi_{AA'}= \square_{A'B'}\nabla^{BB'}\kappa_{AB} +
\square^{C}{}_{B}\nabla^{B}{}_{A'}\kappa_{AC}-\nabla_{BA'}\square^{CD}\kappa_{AC}
+
\nabla^{B}{}_{A'}\square\kappa_{AB}-\nabla^{B}{}_{B'}\square_{A'}{}^{B'}\kappa_{AB}.
\]
Then, a lengthy computation using the decomposition of the auxiliary
vector as given in equation \eqref{DecompositionDKappaAlt}, the
Bianchi identities
\eqref{BianchiIdentityRicciZeroQuantities}-\eqref{BianchiIdentityWeylZeroQuantities}
and the spinorial Ricci identities
\eqref{SpinorialRicciIdentities1}-\eqref{SpinorialRicciIdentities2}
render equation \eqref{WaveEquationAuxiliaryVariable}.  The above
discussion is summarised in the following

\begin{lemma}\label{LemmaRelationKillingSpinorConformalKillingVector}
Let $(\mathcal{M},\bmg)$ represent a solution to the conformal
Einstein field equations admitting a Killing spinor $\kappa_{AB}$;
namely suppose that
\[
H_{A'ABC}=0, \quad Z_{AA'BB'}=0, \quad Z_{AA'}=0,\quad
\Delta_{ABCC'}=0, \quad \Pi_{AA'BB'}=0, \quad \Lambda_{AB'BC}=0.
\]
Let $\xi_{AA'}$ denote the auxiliary vector defined as in equation
\eqref{DefAuxiliaryVector}, then
\[
 \nabla_{CC'}\xi_{DD'} + \nabla_{DD'}\xi_{CC'}
 +6\kappa_{(D}{}^{A}\Phi_{C)AC'D'} =0, \qquad
 \kappa_{(D}{}^{F}\Psi_{ABC)F}=0.
\]
Moreover
\[
X_{AA'} =\Xi\xi_{AA'}-3\kappa_{AQ}\nabla_{A'}{}^{Q}\Xi
\]
is the spinorial counterpart of a conformal Killing vector $X_{a}$
satisfying the unphysical Killing vector equations
\eqref{ConfKillingEquation}--\eqref{UnphysicalKillingCondition}. In
addition, the Killing spinor $\kappa_{AB}$ and the auxiliary vector
$\xi_{AA'}$ satisfy the following wave equations
\begin{eqnarray} 
&& \square \kappa_{BC}=-4\Lambda\kappa_{BC} +
  \kappa^{AD}\Psi_{BCAD}, \label{WaveEqKillingSpinor} \\ && \square
  \xi_{AA'}= - 6 \xi_{AA'} \Lambda - 2 \xi^{BB'} \Phi_{ABA'B'} -
  \tfrac{3}{2} \kappa^{BC} Y_{ABCA'} - 12 \kappa_{AB}
  \nabla^{B}{}_{A'}\Lambda. \label{WaveEqAuxiliaryVector}
\end{eqnarray}
\end{lemma}

In the following, we aim to identify the initial data for
$\kappa_{AB}$ which, when propagated according to
\eqref{WaveEqKillingSpinor} gives rise to a Killing spinor on the
spacetime development. It is important to note that, since
$\kappa_{AB}$ solves equation \eqref{WaveEqKillingSpinor} by
construction, equation \eqref{WaveEqKillingSpinor} can be assumed to
hold throughout $\mathcal{M}$.


\section{Propagation equations}
\label{Sec:PropagationEquations}
In this section we construct, given a solution to equations
\eqref{WaveEqKillingSpinor}-\eqref{WaveEqAuxiliaryVector} 
 on $\mathcal{M}$, a set of wave equations for the
zero-quantities $H_{A'ABC}$ and $S_{AA'BB'}$ which are homogeneous in
these zero-quantities and their first derivatives.

\subsection{The general strategy}
\label{TheGeneralStrategy}



As  discussed in detail in Sections
\ref{FirstWaveEquationForS}-\ref{SecondWaveEquationS}, deriving the
required wave equation for $S_{A'ABC}$ is more involved than the one
for $H_{A'ABC}$.  In order to obtain an homogeneous wave equation for
$S_{AA'BB'}$, we first derive separately two inhomogeneous equations,
which when combined yield the desired homogeneous wave equation. The
first inhomogeneous equation, derived in Section
\ref{FirstWaveEquationForS}, makes use of the definition of
$S_{AA'BB'}$ in terms of the auxiliary vector
(\ref{DefinitionZeroQuantitySIntersmOfAuxiliaryV}). The second
inhomogeneous equation, derived in Section \ref{SecondWaveEquationS},
is obtained through the use of equations \eqref{DefZeroQuantityS}-\eqref{DefZeroQuantityB}
and \eqref{BuchdahlInTermsOfWeylAndKillingSpinor}. 
%to derive the
%inhomogeneous wave equation for $S_{A'ABC}$.
The procedure is summarised in the schematic of Figure \ref{fig:Schematic}.


\begin{figure}[h!]

\begin{equation*}
\begin{rcases}
  \mathbf{S}=\mathbf\Phi\bm\times\bm\kappa +
  \bm\nabla\bm\times\bm\xi
  -\hspace{-1.5mm}-\hspace{-1.5mm}-\hspace{-1.5mm}-\hspace{-1.5mm}-
\hspace{-1.5mm}-\hspace{-1.5mm}-\hspace{-1.5mm}-\hspace{-1.5mm}-
\hspace{-1.5mm}-\hspace{-1.5mm}-\hspace{-1.5mm}-\hspace{-1.5mm}-
\hspace{-1.5mm}-\hspace{-1.5mm}-\hspace{-1.5mm}-\hspace{-1.5mm}-
\hspace{-1.5mm}-\hspace{-1.5mm}-\hspace{-1.5mm}-\hspace{-1.5mm}-
\hspace{-1.5mm}-\hspace{-1.5mm}-\hspace{-1.5mm}-\hspace{-1.5mm}-
\hspace{-1.5mm}-\hspace{-1.5mm}-\hspace{-1.5mm}-\hspace{-2mm}
  \longrightarrow & \square \mathbf{S}= \mathbf{I}_{sym} +
  \mathbf{h}_{1} \\ \\ \\
\begin{rcases}
   &\bm\nabla\bm\times \mathbf{H}= \mathbf{B}+ \mathbf{S}
  \longrightarrow
  (\bm\nabla\bm\times\bm\nabla\bm\times\mathbf{B})_{sym} =\square
  \mathbf{S}_{sym} + \mathbf{h}_{2\;sym} \\ \\ \\ &\mathbf{B}=\bm\Psi
  \bm\times \bm\kappa
  -\hspace{-1.5mm}-\hspace{-1.5mm}-\hspace{-1.5mm}-\hspace{-1.5mm}-\hspace{-1.5mm}\longrightarrow
  (\bm\nabla\bm\times\bm\nabla \bm\times \mathbf{B})_{sym}=
  \mathbf{E}_{sym} + \mathbf{h}_{3\;sym}
\end{rcases}
&\square \mathbf{S}_{sym} =\mathbf{E}_{sym}+ \mathbf{h}_{sym}
\end{rcases}
\square\mathbf{S}=\mathbf{h}
\end{equation*}
\caption{ Schematic description of the derivation of the wave
  equation for $S_{AA'BB'}$, given in Sections
  \ref{FirstWaveEquationForS}-\ref{SecondWaveEquationS}.  In this
  diagram spinors are represented simply by their kernel letters, e.g., 
  $T_{AB...F,C'D'...H'}$ is denoted by $\mathbf{T}$.
 In addition, the symbol $\bm\times$ has been used
  to denote, in a schematic way, contractions between
  spinors.  The subscript \emph{sym} has been added to indicate that a
  given expression is symmetric. The quantities $\mathbf{h}_{1}$,
  $\mathbf{h}_{2}$ and $\mathbf{h}_{3}$ denote  homogeneous
  expressions in $\mathbf{H}$, $\bm\nabla
  \bm\times\mathbf{H}$, $\mathbf{S}$ and $\bm\nabla \bm\times
  \mathbf{S}$. The quantities $\mathbf{I}_{sym}$ and
  $\mathbf{E}_{sym}$ encode the inhomogeneous terms appearing in the
  corresponding equations.}
\label{fig:Schematic}
\end{figure}


Once the desired wave equations are obtained the following
result for homogeneous wave equations will be used:

\medskip

Observe that  it follows from the
uniqueness property of Theorem \ref{TheoremHomogeneousWave} that the
zero-quantities are \emph{propagated}; that is to say that if the
initial conditions
\begin{eqnarray*}
&& H_{A'ABC}=0,\\ && \mathcal{P} H_{A'ABC}=0,\\ &&
  S_{AA'BB'}=0,\\ &&\mathcal{P} S_{AA'BB'}=0,
\end{eqnarray*} 
hold on $\mathcal{U}\subset\mathcal{S}$, then $H_{A'ABC}$ and
$S_{AA'BB'}$ vanish identically on the domain of dependence of
$\mathcal{U}$. The above initial conditions may then be translated
into necessary and sufficient conditions for the Killing spinor
candidate, $\kappa_{AB}$, restricted to a Cauchy hypersurface
$\mathcal{S}$; this is done in Section
\ref{Sec:IntrinsicConditions}.


\subsection{Wave equation for $H_{A'ABC}$}
To construct the wave equation for $H_{A'ABC}$ one starts from the
identity
\[
\square H_{A'ABC}= \nabla_{DD'}\nabla_{A}{}^{D'}H_{A'BC}{}^{D} -
\nabla_{AD'}\nabla_{D}{}^{D'}H_{A'}{}^{D}{}_{BC}.
\]
Substituting the definition of $S_{AA'BB'}$, as given in equation
\eqref{DefZeroQuantityS}, in the second term of the last expression
and using equations
\eqref{SpinorialRicciIdentities1}-\eqref{SpinorialRicciIdentities2}
one obtains
\begin{equation}\label{WaveEquationForH}
\square H_{A'ABC} = 10 \Lambda H_{A'ABC} - 4 \Psi_{ADF(B}
H_{|A'|C)}{}^{DF} - 2 \Phi_{A}{}^{D}{}_{A'}{}^{D'} H_{D'BCD} - 2
\nabla_{AD'}S_{B}{}^{D'}{}_{CA'}.
\end{equation}
Exploiting the symmetries of $H_{A'ABC}$ one obtains
\begin{equation}\label{WaveEquationForHreduced}
\square H_{A'ABC} = 10 \Lambda H_{A'ABC} + 2
\nabla_{(A}{}^{D'}S_{B|D'|C)A'} - 2
\Phi_{(A}{}^{D}{}_{|A'}{}^{D'}H_{D'|BC)D} - 4
\Psi_{(AB}{}^{DF}H_{|A'|C)DF}.
\end{equation}
Observe that equations \eqref{WaveEquationForHreduced} and
\eqref{WaveEquationForH} contain the same information as the traces of
equation \eqref{WaveEquationForH} represent identities which follow
from the definition of $S_{AA'BB'}$ in terms of $H_{A'ABC}$ as given
in equation \eqref{DefZeroQuantityS}. These identities will be useful
for the subsequent discussion.
\begin{subequations}
\begin{eqnarray}
&& \nabla_{AD'}S^{AD'}{}_{CA'} = - \Psi_{CADB} H_{A'}{}^{ADB} +
  \Phi^{AD}{}_{A'}{}^{D'} H_{D'CAD}, \label{ExtraEquations1}\\ &&
  \nabla_{AD'}S_{B}{}^{D'A}{}_{A'} = - \Psi_{BADC} H_{A'}{}^{ADC} +
  \Phi^{AD}{}_{A'}{}^{D'} H_{D'BAD}, \label{ExtraEquations2}\\ &&
  \nabla_{AD'}S^{BD'}{}_{BA'} = 0. \label{ExtraEquations3}
\end{eqnarray}
\end{subequations}
Notice that equation \eqref{WaveEquationForHreduced} is a wave
equation homogeneous in $H_{A'ABCD}$, $S_{AA'BB'}$ and their first
derivatives.
 
\subsection{Inhomogeneous wave equation for $S_{AA'BB'}$}
\label{FirstWaveEquationForS}

The purpose of this section is to derive the first of two inhomogeneous
 wave equations for $S_{AA'BB'}$, which, when combined, will yield the
  desired homogeneous equation. To proceed, some ancillary spinorial 
  decompositions will be required.
   

\subsubsection{Ancillary decompositions}
\label{AnciliaryResults}
This section collects some ancillary decompositions which
will prove useful for the
derivation of the wave equation for $S_{AA'BB'}$. To start the
discussion, observe that from expression \eqref{DefZeroQuantityS} it
follows that
\begin{equation}\label{DivergenceXiIntermsofS}
\nabla_{CC'}\xi^{CC'}=-\frac{1}{2}S_{CC'}{}^{CC'}, \qquad
\nabla_{(A|(A'}\xi_{B')|B)}=-\frac{1}{2}S_{(AB)(A'B')}-3\kappa_{(A}{}^{C}\Phi_{B)CA'B'}.
\end{equation}
Using the above expressions the gradient of the auxiliary vector
$\xi_{AA'}$ can be decomposed as
\begin{multline}\label{SymmetrisedDerivativesXi}
\nabla_{AA'}\xi_{BB'}= -3\kappa_{(B}{}^{C}\Phi_{A)CA'C'}
-\frac{1}{8}S^{CC'}{}_{CC'}\epsilon_{AB}\epsilon_{A'B'}-\frac{1}{2}S_{(AB)(A'B')}
\\ -
\frac{1}{2}\epsilon_{A'B'}\nabla_{(A}{}^{C'}\xi_{B)C'}-\frac{1}{2}\epsilon_{AB}\nabla^{C}{}_{(A'}\xi_{B')C}.
\end{multline}
Using the definitions of $H_{A'ABC}$ and $\xi_{AA'}$ encoded in
equations \eqref{DefZeroQuantityH} and \eqref{DefAuxiliaryVector}
respectively, one can reexpress the gradient of the Killing spinor as
\begin{equation}\label{eq:decompositionDKappaAlt}
\nabla_{AA'}\kappa_{BC} = \tfrac{1}{3} H_{A'ABC} - \tfrac{1}{3}
\xi_{CA'} \epsilon_{AB} - \tfrac{1}{3} \xi_{BA'} \epsilon_{AC}.
\end{equation}
The irreducible decomposition of the gradient of the trace-free Ricci
spinor the second Bianchi identity as expressed in
\eqref{BianchiIdentityRicciZeroQuantities} and the equation encoded in
the zero-quantity \eqref{ZeroQuantitySchouten} implies
\begin{multline}
\nabla_{AA'}\Phi_{BCB'C'} = \tfrac{1}{6} \bar{Y}_{A'B'C'C}
\ \epsilon_{AB} + \tfrac{1}{6} \bar{Y}_{A'B'C'B} \epsilon_{AC} +
\tfrac{1}{6} Y_{ABCC'} \bar{\epsilon}_{A'B'} + \tfrac{1}{6} Y_{ABCB'}
\bar{\epsilon}_{A'C'} \\ \qquad \qquad \qquad \qquad - \tfrac{2}{3}
\ \bar{\epsilon}_{A'C'} \epsilon_{A(C}\nabla_{B)B'}\Lambda -
\tfrac{2}{3} \ \bar{\epsilon}_{A'B'}
\epsilon_{A(C}\nabla_{B)C'}\Lambda + \nabla_{(A'|(A}\Phi_{BC)|B'C')}.
\end{multline}

\noindent Another identity that will be used involving second
derivatives of the tracefree Ricci spinor is derived as follows:
Applying $\nabla^{A}{}_{B'}$ to the zero-quantity encoded in
\eqref{ZeroQuantitySchouten}, and after a lengthy computation
using equations
\eqref{BianchiIdentityRicciZeroQuantities}-\eqref{BianchiIdentityWeylZeroQuantities}
and
\eqref{ZeroQuantitySecondDerivativeConformalFactor}-\eqref{ZeroQuantityWeyl}, one obtains 
the identity 
\begin{align}
\square \Phi_{BCC'D'} &=
8 \Lambda  \Phi_{BCB'C'} - 2 \
\Phi_{B}{}^{A}{}_{C'}{}^{A'} \Phi_{CAB'A'} - 2 \
\Phi_{B}{}^{A}{}_{B'}{}^{A'} \Phi_{CAC'A'} - 2 \Phi_{BC}{}^{A'D'} \
\Psi_{B'C'A'D'} \nonumber\\ 
& +\tfrac{3}{2}\epsilon_{BC}\bar{\epsilon}_{B'C'}\square  \Lambda- 2 \
\nabla_{(B|(B'}\nabla_{C')|C)}\Lambda
 + \nabla_{AB'}Y_{B}{}^{A'A}{}_{A'CC'} + 2 \nabla_{AB'}\nabla_{BA'}\Phi_{C}{}^{A}{}_{C'}{}^{A'} .
  \label{WaveEquationRicciSpinor}
\end{align}

\noindent Additionally, the following decompositions will be used in
the subsequent discussion
\begin{eqnarray}
&&\nabla_{AA'}\nabla_{BB'}\Lambda = \tfrac{1}{4} \epsilon_{AB}
  \bar{\epsilon }_{A'B'} \square \Lambda +
  \nabla_{(A'|(A}\nabla_{B)|B')}\Lambda , \\ &&\xi_{DD'}
  \nabla_{CC'}\Lambda = \tfrac{1}{4} \xi^{AA'} \epsilon_{CD}
  \bar{\epsilon}_{C'D'} \nabla_{AA'}\Lambda + \xi_{(C|(C'}
  \nabla_{D')|D)}\Lambda \nonumber \\ && \qquad \qquad \qquad
  \qquad\qquad \qquad + \tfrac{1}{2} \bar{\epsilon}_{C'D'}
  \xi_{(C}{}^{A'}\nabla_{D)A'}\Lambda + \tfrac{1}{2} \epsilon_{CD}
  \xi^{A}{}_{(C'}\nabla_{|A|D')}\Lambda .
\end{eqnarray}



\subsubsection{Wave equation for $S_{AA'BB'}$}


 Applying $\nabla_{PP'}\nabla^{PP'}$ to the expression for
 $S_{AA'BB'}$ in terms of first derivatives of the auxiliary vector as
 encoded in equation
 \eqref{DefinitionZeroQuantitySIntersmOfAuxiliaryV} one obtains
\begin{equation}
\square
S_{CC'DD'}=-6\nabla_{PP'}\nabla^{PP'}(\kappa_{(C}{}^{Q}\Phi_{D)QC'D'})
-\nabla_{PP'}\nabla^{PP'}\nabla_{CC'}\xi_{DD'} -
\nabla_{PP'}\nabla^{PP'} \nabla_{DD'}\xi_{CC'}.
\end{equation}
Commuting covariant derivatives renders
\begin{align*}
\square S_{CC'DD'}&= -6 \Phi_{C'D'A(C} \square \kappa_{D)}{}^{A}-6
\kappa_{(C}{}^{A}\square \Phi_{D)AC'D'} -\nabla_{CC'}\square
\xi_{DD'}-\nabla_{DD'}\square \xi_{CC'}\\ &- \square_
   {CA}\nabla^{A}{}_{C'}\xi_{DD'} - \square_
   {C'A'}\nabla_{C}{}^{A'}\xi_{DD'} - \square_ {DA}\nabla^{A}{}_
   {D'}\xi_{CC'} - \square_ {D'A'}\nabla_{D}{}^{A'}\xi_{CC'}\\ & +
   \nabla_{AC'}\square_{C}{}^{A}\xi_{DD'} +
   \nabla_{AD'}\square_{D}{}^{A}\xi_{CC'} +
   \nabla_{CA'}\square_{C'}{}^{A'}\xi_{DD'}+\nabla_{DA'}\square_{D'}{}^{A'}\xi_{CC'}\\ &
   -6 \nabla_{BA'}\Phi_{DAC'D'} \nabla^{BA'}\kappa_{C}{}^{A} -6
   \nabla_{BA'}\Phi_{CAC'D'} \nabla^{BA'}\kappa_{D}{}^{A} .
\end{align*}
At this point one can substitute the wave equation for the Killing
spinor, the auxiliary spinor and the tracefree Ricci spinor as given
in equations \eqref{WaveEqKillingSpinor},
\eqref{WaveEqAuxiliaryVector} and \eqref{WaveEquationRicciSpinor}
respectively. Additionally, observe that the the spinorial Ricci
identities
\eqref{SpinorialRicciIdentities1}-\eqref{SpinorialRicciIdentities2}
can be employed to replace the operator $\square_{AB}$ by curvature
terms.  Substituting equations \eqref{WaveEqKillingSpinor},
\eqref{WaveEqAuxiliaryVector}, \eqref{WaveEquationRicciSpinor},
\eqref{BuchdahlInTermsOfWeylAndKillingSpinor}, the conformal
Einstein field equations as encoded in
\eqref{CFEZeroQuantitiesEqualToZero} and the auxiliary results
collected in Section \ref{AnciliaryResults}, one obtains the
 following wave
equation for $S_{CC'DD'}$
\begin{align}\label{WaveEqSversion1}
 \square S_{CC'DD'}&= 12 B_{CDAB} \Phi^{AB}{}_{C'D'} - 3
 \xi^{A}{}_{D'} Y_{CDAC'} - 3 \xi^{A}{}_{C'} Y_{CDAD'} \nonumber\\ & +
 3 \kappa_{D}{}^{A} \nabla_{BC'}Y_{CA}{}^{B}{}_{D'} + 3
 \kappa_{C}{}^{A} \nabla_{BC'}Y_{DA}{}^{B}{}_{D'} + \tfrac{3}{2}
 \kappa^{AB} \nabla_{DD'}Y_{CABC'}+ \tfrac{3}{2} \kappa^{AB}
 \nabla_{CC'}Y_{DABD'} \nonumber\\ & + \tfrac{1}{6}
 Y_{D}{}^{AB}{}_{D'} H_{C'CAB}+ \tfrac{1}{6} Y_{C}{}^{AB}{}_{C'}
 H_{D'DAB} - \tfrac{1}{3} Y_{C}{}^{AB}{}_{D'} H_{C'DAB} - \tfrac{1}{3}
 Y_{D}{}^{AB}{}_{C'} H_{D'CAB} \nonumber\\ & + \tfrac{8}{3} H_{D'CDA}
 \nabla^{A}{}_{C'}\Lambda + \tfrac{8}{3} H_{C'CDA}
 \nabla^{A}{}_{D'}\Lambda - \tfrac{2}{3} \bar{Y}_{C'D'}{}^{A'A}
 H_{A'CDA} - 3 \Lambda S^{AA'}{}_{AA'} \epsilon_{CD}
 \bar{\epsilon}_{C'D'} \nonumber\\ & -
 \nabla_{CC'}\left(H_{D'}{}^{ABF} \Psi_{DABF}\right) -
 \nabla_{DD'}\left(H_{C'}{}^{ABF} \Psi_{CABF}\right) \nonumber\\ &- 2
 \Psi_{CDAB} S^{(A}{}_{(C'}{}^{B)}{}_{D')} - 2 \bar{\Psi}_{C'D'A'B'}
 S_{(C}{}^{(A'}{}_{D)}{}^{B')} + 4 \Lambda S_{(C|(C'|D)D')}
 \nonumber\\ & - 2 \Phi_{D}{}^{A}{}_{D'}{}^{A'} (S_{(C|(C'|A)A')}- 2
 \Phi_{D}{}^{A}{}_{C'}{}^{A'} S_{(C|(D'|A)A)'}- 2
 \Phi_{C}{}^{A}{}_{D'}{}^{A'} S_{(D|C'|A)A')}\nonumber\\ & - 2
 \Phi_{C}{}^{A}{}_{C'}{}^{A'} S_{(D|D'|A)A')} - 2 H^{A'}{}_{D}{}^{AB}
 \nabla_{C'|(C}\Phi_{AB)|D'A')} \nonumber\\ & - 2 H^{A'}{}_{C}{}^{AB}
 \nabla_{(C'|(D}\Phi_{AB)|D'A')}.
\end{align}
Observe that the latter is an homogeneous expression in $H_{A'ABC}$,
$S_{AA'BB'}$ and its first derivatives except for the term
\begin{multline*}
I_{CC'DD'} \equiv 12 B_{CDAB} \Phi^{AB}{}_{C'D'} - 3 \xi^{A}{}_{D'}
Y_{CDAC'} - 3 \xi^{A}{}_{C'} Y_{CDAD'} + 3 \kappa_{D}{}^{A}
\nabla_{BC'}Y_{CA}{}^{B}{}_{D'} \\ + 3 \kappa_{C}{}^{A}
\nabla_{BC'}Y_{DA}{}^{B}{}_{D'} + \tfrac{3}{2} \kappa^{AB}
\nabla_{CC'}Y_{DABD'} + \tfrac{3}{2} \kappa^{AB} \nabla_{DD'}Y_{CABC'}.
\end{multline*}
As an additional remark observe that taking a trace of equation
\eqref{WaveEqSversion1} one obtains
\begin{multline}
\square S_{CC'}{}^{C}{}_{D'}= \tfrac{1}{2} \bar{Y}_{D'}{}^{A'B'A}
\bar{H}_{AC'A'B'} - \tfrac{1}{2} \bar{Y}_{C'}{}^{A'B'A}
\bar{H}_{AD'A'B'} - 6 \Lambda \bar{S}^{A'A}{}_{A'A}
\bar{\epsilon}_{C'D'} \\ - \nabla_{AC'}\left(\bar{H}^{AA'B'F'}
\bar{\Psi}_{D'A'B'F'}\right) + \nabla_{AD'}\left(\bar{H}^{AA'B'F'}
\bar{\Psi}_{C'A'B'F'}\right).
\end{multline}
Observe that the right-hand side of the last equation is an
homogeneous expression in $H_{A'ABC}$, $S_{AA'BB'}$ and its first
derivatives. Consequently, exploiting the irreducible decomposition of
$S_{AA'BB'}$ to write
\begin{align}\label{decompositionWaveEqS}
\square S_{CC'DD'} = \square S_{(CD)(C'D')} &- \tfrac{1}{2}
\bar{\epsilon}_{C'D'}\square S_{(C}{}^{A'}{}_{D)A'} \nonumber\\ &-
\tfrac{1}{2} \epsilon_{CD} \square S^{A}{}_{(C'|A|D')}+ \tfrac{1}{4}
\epsilon_{CD} \bar{\epsilon}_{C'D'}\square S^{AA'}{}_{AA'},
\end{align}
 one can re-express equation \eqref{WaveEqSversion1} as
\begin{equation}\label{WaveEqSFirstSplit}
\square S_{CC'DD'}=I_{(CD)(C'D')} + F_{CC'DD'},
\end{equation}
where $ F_{CC'DD'}$ is an homogeneous expression depending on
$H_{A'ABC}$, $S_{AA'BB'}$ and its first derivatives.  Notice that the
inhomogeneous term $I_{(CD)(C'D')} $ contains the Buchdahl constraint
$B_{ABCD}$. Consequently, one needs to analyse more closely this
quantity.  An immediate but important observation for the latter
discussion is that the main obstruction to obtaining an homogeneous wave
equation is contained in the symmetric part of $S_{CDC'D'}$:
\begin{equation}\label{SymmetricWaveEqS1}
\square S_{(CD)(C'D')}=I_{(CD)(C'D')} + F_{(CD)(C'D)'}.
\end{equation}

\subsection{Derivatives of the Buchdahl zero-quantity}
\label{DerivativesOfTheBuchdahlZeroQuantity}

The presence of the Buchdahl zero-quantity in the
 inhomogeneous term of equation \eqref{SymmetricWaveEqS1} suggests
that it is necessary to find auxiliary identities associated to the
Buchdahl constraint. The aim of this section is to derive such
identities by expressing $\nabla^{C}{}_{C'}\nabla^{A}{}_{A'}B_{ABCD}$
in two different ways by exploiting equations
\eqref{DefZeroQuantityS}-\eqref{DefZeroQuantityB} and
\eqref{BuchdahlInTermsOfWeylAndKillingSpinor}.

\subsubsection{First approach to express second derivatives of the Buchdahl constraint}

The irreducible decomposition of $\nabla_{AA'}H^{A'}{}_{BCD}$ and
definitions \eqref{DefZeroQuantityS}-\eqref{DefZeroQuantityB} give
\begin{equation}
\nabla_{AA'}H^{A'}{}_{BCD} = -6 B_{ABCD} - \tfrac{3}{4} \epsilon_{A(B}
S_{C}{}^{A'}{}_{D)A'}.
\end{equation}
Applying $\nabla^{A}{}_{Q'}$ to the last expression one obtains
\begin{equation}\label{IntermediateStepForCurlOfBuchdahl}
\nabla^{A}{}_{Q'}\nabla_{AA'}H^{A'}{}_{BCD} = 6
\nabla_{AQ'}B_{BCD}{}^{A} - \tfrac{3}{4}
\nabla_{Q'(B}S_{C}{}^{A'}{}_{D)A'}.
\end{equation}
Exploiting the spinorial Ricci identities
\eqref{SpinorialRicciIdentities1}-\eqref{SpinorialRicciIdentities2} in equation
\eqref{IntermediateStepForCurlOfBuchdahl} and rearranging one derives
the following expression
\begin{align*}
\nabla_{AQ'}B_{BCD}{}^{A}&=\tfrac{1}{12}\square H_{Q'BCD} -
\tfrac{1}{2} \Lambda H_{Q'BCD} \\ &+\tfrac{1}{6}
\Phi_{D}{}^{A}{}_{Q'}{}^{A'} H_{A'BCA} + \tfrac{1}{6}
\Phi_{C}{}^{A}{}_{Q'}{}^{A'} H_{A'BDA} + \tfrac{1}{6}
\Phi_{B}{}^{A}{}_{Q'}{}^{A'} H_{A'CDA} \\ &+ \tfrac{1}{24}
\nabla_{BQ'}S_{C}{}^{A'}{}_{DA'} + \tfrac{1}{24}
\nabla_{CQ'}S_{B}{}^{A'}{}_{DA'} + \tfrac{1}{24}
\nabla_{DQ'}S_{B}{}^{A'}{}_{CA'}.
\end{align*}
At this point we can substitute for the D'Alembertian of the zero
quantity $H_{A'ABC}$ using the wave equation
(\ref{WaveEquationForHreduced}). Having done so, one can apply a
further derivative to the above equation to obtain
\begin{equation}
\nabla^{C}{}_{C'}\nabla^{A}{}_{A'}B_{ABCD}=\tfrac{1}{36}\square
S_{(B|C'|D)A'} -\tfrac{1}{48}\square
S_{B}{}^{B'}{}_{DB'}\bar{\epsilon}_{A'C'} + \Sigma_{C'A'BD} +
P_{C'A'BD}
\end{equation}
where $ P_{C'A'BD}$ is an homogeneous expression in on $H_{A'ABC}$,
$S_{AA'BB'}$ and its first derivatives, given in appendix
\ref{Appendix}, and where $\Sigma_{C'A'BD}$ is given by
\begin{align*}
\Sigma_{C'A'BD} &\equiv \tfrac{1}{24}
\nabla_{AC'}\nabla_{BA'}S^{AB'}{}_{DB'} - \tfrac{1}{36}
\nabla_{AC'}\nabla_{BB'}S^{AB'}{}_{DA'} - \tfrac{1}{36}
\nabla_{AC'}\nabla_{BB'}S_{D}{}^{B'A}{}_{A'} \\ &+ \tfrac{1}{24}
\nabla_{AC'}\nabla_{DA'}S_{B}{}^{B'A}{}_{B'} - \tfrac{1}{36}
\nabla_{AC'}\nabla_{DB'}S^{AB'}{}_{BA'} - \tfrac{1}{36}
\nabla_{AC'}\nabla_{DB'}S_{B}{}^{B'A}{}_{A'}.
\end{align*}
Commuting covariant derivatives and using the identities
\eqref{ExtraEquations1}-\eqref{ExtraEquations3} one can rewrite
the above expression as
follows
\begin{equation}
\Sigma_{C'A'BD} = \tfrac{1}{18}\square S_{(B|C'|D)A'} +
Q_{C'A'BD}
\end{equation}
where $Q_{C'A'BD}$ is an homogeneous expression in on $H_{A'ABC}$,
$S_{AA'BB'}$ and its first derivatives, given in Appendix
\ref{Appendix}. Consequently, one has
\begin{equation} \label{SecondDerivativesOfBuchdahlToEquate1}
\nabla^{C}{}_{C'}\nabla^{A}{}_{A'}B_{ABCD}=\tfrac{1}{12}\square
S_{(B|C'|D)A'} -\tfrac{1}{48}\square
S_{B}{}^{B'}{}_{DB'}\bar{\epsilon}_{A'C'} + P_{C'A'BD}+Q_{C'A'BD}
\end{equation}


\subsubsection{Second approach to express second derivatives of the Buchdahl constraint}

An alternative way to obtain an expression for
$\nabla^{C}{}_{C'}\nabla^{A}{}_{A'}B_{ABCD}$ is to start from equation
equation \eqref{BuchdahlInTermsOfWeylAndKillingSpinor} to obtain
\begin{align*}
& \nabla^{A}{}_{A'}B_{ABCD} = \Psi_{F(BCD} \nabla^{A}{}_{\vert
    A'\vert}\kappa_{A)}{}^{F} + \kappa_{(D}{}^{F} \nabla^{A}{}_{\vert
    A'\vert}\Psi_{ABC)F} .
\end{align*}
Exploiting the decomposition \eqref{eq:decompositionDKappaAlt} and the
conformal Einstein field equations as encoded in
\eqref{CFEZeroQuantitiesEqualToZero} gives
\begin{align}\label{CurlBuchdahlStraighforward}
&\nabla^{A}{}_{A'}B_{ABCD} = \tfrac{1}{2} \xi^{A}{}_{A'} \Psi_{BCDA} -
  \tfrac{3}{8} \kappa_{(B}{}^{A} Y_{CD)AA'} + \tfrac{1}{4}
  \Psi_{AF(BC} H_{D)A'}{}^{AF} - \tfrac{1}{4} \kappa^{AF}
  \nabla_{FA'}\Psi_{BCDA}.
\end{align} 
Applying $\nabla^{C}{}_{C'}$ to equation
\eqref{CurlBuchdahlStraighforward}, a long calculation exploiting the
irreducible decomposition of $\nabla_{AA'}\kappa_{BC}$ and
$\nabla_{AA'}\xi_{BB'}$, commuting covariant derivatives and using the
conformal Einstein field equations as given in
\eqref{CFEZeroQuantitiesEqualToZero} renders


\begin{align}\label{SecondDerivativesOfBuchdahlToEquate2}
 \nabla^{C}{}_{C'}\nabla^{A}{}_{A'}B_{ABCD} &= \tfrac{1}{4}
 \left(\Phi_{D}{}^{F}{}_{A'C'} \Psi_{BACF} - 4 \Phi_{A}{}^{F}{}_{A'C'}
 \Psi_{BDCF} + \Phi_{B}{}^{F}{}_{A'C'} \Psi_{DACF}\right)\kappa^{AC}
 \nonumber \\ & +\tfrac{1}{4}\bar{\epsilon}_{A'C'}\left( \Psi_{ACFG}
 \Psi_{BD}{}^{FG} + 2 \Psi_{BA}{}^{FG} \Psi_{DCFG} - 6\Lambda
 \Psi_{BDAC}\right)\kappa^{AC} \nonumber \\ & + \tfrac{1}{8}
 \kappa^{AC} \nabla_{CA'}Y_{BDAC'} + \tfrac{1}{8} \kappa_{D}{}^{A}
 \nabla_{CC'}Y_{BA}{}^{C}{}_{A'} - \tfrac{1}{4} \xi^{A}{}_{C'}
 Y_{BDAA'} \nonumber \\ & + \tfrac{1}{8} \kappa^{AC}
 \nabla_{CC'}Y_{BDAA'}+ \tfrac{1}{8} \kappa_{B}{}^{A}
 \nabla_{CC'}Y_{DA}{}^{C}{}_{A'} - \tfrac{1}{4} \xi^{A}{}_{A'}
 Y_{BDAC'}\nonumber\\ & + \tfrac{1}{4} \Psi_{BDAC}
 \bar{\epsilon}_{A'C'} \nabla^{(A|B'|}\xi^{C)}{}_{B'} + U_{A'BC'D},
\end{align}
where $ U_{A'BC'D}$ is an homogeneous expression in on $H_{A'ABC}$,
$S_{AA'BB'}$ and its first derivatives, given  in Appendix \ref{Appendix}.

\subsection{Wave equation for $S_{(AB)(A'B')}$}
\label{SecondWaveEquationS}

In Section \ref{DerivativesOfTheBuchdahlZeroQuantity} two different expressions
for $ \nabla^{C}{}_{C'}\nabla^{A}{}_{A'}B_{ABCD}$ were computed.
 Observe that the right-hand side of equation
\eqref{SecondDerivativesOfBuchdahlToEquate1} contains $\square
S_{(B|C'|D)A'}$ while \eqref{SecondDerivativesOfBuchdahlToEquate2}
does not. Consequently, one can use equations
\eqref{SecondDerivativesOfBuchdahlToEquate1} and
\eqref{SecondDerivativesOfBuchdahlToEquate2} to obtain a wave equation
for $S_{(B|C'|D)A'}$.  A direct computation using equations
\eqref{SecondDerivativesOfBuchdahlToEquate1},
\eqref{SecondDerivativesOfBuchdahlToEquate2} and
\eqref{BuchdahlInTermsOfWeylAndKillingSpinor} renders
\begin{equation}\label{SymmetricWaveEqS2}
\square S_{(AB)(A'B')}= \mathcal{I}_{ABA'B'} + W_{ABA'B'},
\end{equation}
where $W_{ABA'B'} =W_{(AB)(A'B')}$ and $\mathcal{I}_{ABA'B'}=
\mathcal{I}_{(AB)(A'B')}$ are homogeneous expressions in $H_{A'ABC}$,
$S_{AA'BB'}$ and their first derivatives, given by
\begin{align*}
W_{ABA'B'}\equiv &
12\left(U_{(AB)(A'B')}-P_{(AB)(A'B')}-Q_{(AB)(A'B')}\right),\\ \mathcal{I}_{ABA'B'}
\equiv &-12 B_{ABCD} \Phi^{CD}{}_{A'B'} + 3 \xi^{C}{}_{B'} Y_{ABCA'} +
3 \xi^{C}{}_{A'} Y_{ABCB'} \\ &- \tfrac{3}{4} \kappa_{B}{}^{C}
\nabla_{DA'}Y_{AC}{}^{D}{}_{B'} - \tfrac{3}{4} \kappa_{A}{}^{C}
\nabla_{DA'}Y_{BC}{}^{D}{}_{B'} - \tfrac{3}{2} \kappa^{CD}
\nabla_{DB'}Y_{ABCA'} \\ &- \tfrac{3}{4} \kappa_{B}{}^{C}
\nabla_{DB'}Y_{AC}{}^{D}{}_{A'} - \tfrac{3}{4} \kappa_{A}{}^{C}
\nabla_{DB'}Y_{BC}{}^{D}{}_{A'}- \tfrac{3}{2} \kappa^{CD}
\nabla_{DA'}Y_{ABCB'}.
\end{align*}
Through use of the following identity
\begin{equation*}\label{SwappindicesKillingAndCotton}
\kappa^{CD} \nabla_{AA'}Y_{BCDB'} = \kappa^{CD} \nabla_{DA'}Y_{ABCB'}
- \kappa_{A}{}^{C} \nabla_{DA'}Y_{BC}{}^{D}{}_{B'}
\end{equation*}
it may be shown that in fact
\[\mathcal{I}_{ABA'B'}=-\tfrac{1}{2}I_{(AB)(A'B')}.\]


\subsection{Homogeneous wave equation for $S_{AA'BB'}$}

As discussed before, the main obstruction to obtaining an homogeneous wave
equation is contained to the symmetric part of $S_{ABA'B'}$. More
specifically, the obstruction is contained in the terms denoted by $I_{AA'BB'}$ and
$\mathcal{I}_{AA'BB'}$ in equations \eqref{SymmetricWaveEqS1} and
\eqref{SymmetricWaveEqS2} respectively. However,
 one can take linear combinations of
equations \eqref{SymmetricWaveEqS1} and \eqref{SymmetricWaveEqS2} to
remove the inhomogeneous terms.  In particular one has
\begin{equation*}
3\square S_{(AB)(A'B')}= I_{(AB)(A'B')} + 2\mathcal{I}_{ABA'B'}+
F_{(AB)(A'B')} + 2W_{ABA'B'}.
\end{equation*}
After a direct computation using the explicit form of $I_{ABA'B'}$ and
$\mathcal{I}_{ABA'B'}$ one concludes that
\begin{equation} \label{homogenoeousWaveEquationSymmetrisedPart}
\square S_{(AB)(A'B')}=\frac{1}{3} F_{(AB)(A'B')} +
\frac{2}{3}W_{ABA'B'}.
\end{equation}
 Finally, using equation
 \eqref{homogenoeousWaveEquationSymmetrisedPart},
 \eqref{decompositionWaveEqS} and \eqref{WaveEqSFirstSplit} one
 obtains
\begin{align}
\label{FinalWaveEqS}
\square S_{AA'BB'} & = \frac{1}{3} F_{(AB)(A'B')} - \tfrac{1}{2}
\bar{\epsilon}_{A'B'} F_{(A}{}^{Q'}{}_{B)Q'} - \tfrac{1}{2}
\epsilon_{AB} F^{Q}{}_{(A'|Q|B')}\nonumber \\ & \qquad \qquad \qquad
\qquad \qquad \qquad \qquad \qquad + \tfrac{1}{4} \epsilon_{AB}
\bar{\epsilon}_{A'B'}F^{QQ'}{}_{QQ'} + \frac{2}{3}W_{ABA'B'}.
\end{align}
Notice that latter encodes an homogeneous expressions for $S_{AA'BB'}$
as $F_{(AB)(A'B')}$ and $W_{ABA'B'}$ represent homogeneous expressions
on $H_{A'ABC}$, $S_{AA'BB'}$ and its first derivatives.
\\

We are now in a position to state the following proposition:

\begin{proposition}\label{Prop:Propagation}
Given initial data for the alternative conformal field equations on a
spacelike hypersurface $\mathcal{S}$ with normal vector $\tau^{AA'}$,
and associated normal derivative
 $\mathcal{P}\equiv \tau^{AA'}\nabla_{AA'}$,
the corresponding spacetime development admits a Killing spinor in the
domain of dependence of $\mathcal{U}\subset\mathcal{S}$ if and only if
\begin{subequations}
\begin{eqnarray}
&& H_{A'ABC}=0,\label{eq:VanishingOfH}\\ && \mathcal{P}
  H_{A'ABC}=0,\label{eq:VanishingOfNormalDerivH}\\ &&
  S_{AA'BB'}=0,\label{eq:VanishingOfS}\\ &&\mathcal{P}
  S_{AA'BB'}=0 \label{eq:VanishingOfNormalDerivS}
\end{eqnarray}
\end{subequations}
 hold on $\mathcal{U}$.
\end{proposition}
\begin{proof}
The \emph{only if} direction is immediate. Suppose, on the other hand,
that (\ref{eq:VanishingOfH})--(\ref{eq:VanishingOfNormalDerivS}) hold
on some $\mathcal{U}\subset\mathcal{S}$ ---that is to say, there exist
spinor fields $\kappa_{AB},~\xi_{AA'}$ for which
(\ref{eq:VanishingOfH})--(\ref{eq:VanishingOfNormalDerivS}) are
satisfied on $\mathcal{U}$. The latter is then used as initial data
for the wave equations
\begin{eqnarray*} 
&& \square \kappa_{BC}=-4\Lambda\kappa_{BC} + \kappa^{AD}\Psi_{BCAD},
  \\ && \square \xi_{AA'}= - 6 \xi_{AA'} \Lambda - 2 \xi^{BB'}
  \Phi_{ABA'B'} - \tfrac{3}{2} \kappa^{BC} Y_{ABCA'} - 12 \kappa_{AB}
  \nabla^{B}{}_{A'}\Lambda.
\end{eqnarray*}
As the zero-quantities $H_{A'ABC},~S_{AA'BB'}$ satisfy the homogeneous
wave equations \eqref{WaveEquationForHreduced}, \eqref{FinalWaveEqS}
then the uniqueness result for homogeneous wave equations discussed in
Section \ref{TheGeneralStrategy} ensures that
\[ H_{A'ABC}=0,\qquad S_{AA'BB'}=0,\]
in the domain of dependence of $\mathcal{U}$. In other words,
$\kappa_{AB}$ solves the Killing spinor equation on the domain of
dependence of $\mathcal{U}$.
\end{proof}


\begin{remark}
\emph{ At first glance one might assume that the standard
  formulation of the conformal Einstein field equations is the
  appropriate setting for deriving the conditions obtained in this
  article. Nevertheless some experimentation reveals that instead of a
  conformally regular system of wave equations for $H_{A'ABC}$ and
  $S_{AA'BB'}$ one is confronted with an homogeneous Fuchsian system
  ---formally singular at $\Xi=0$. Although one could potentially
  still analyse this system and obtain an analogous result to
  Proposition \ref{Prop:Propagation}, one would require a uniqueness
  result for solutions to Fuchsian systems of wave
  equations. Moreover, following the original spirit of the derivation
  of the conformal Einstein field equations in \cite{Fri81a} one is
  interested in finding conformally regular equations instead of
  analysing Fuchsian systems. Fortunately, as shown in Section
  \ref{Sec:PropagationEquations} the alternative formulation of the
  conformal Einstein field equations given in Section \ref{AltCFEs}
  leads to a regular set of wave equations for $H_{A'ABC}$ and
  $S_{AA'BB'}$.  }
\end{remark}

\section{The intrinsic conditions}
\label{Sec:IntrinsicConditions}
In this section the conditions
\eqref{eq:VanishingOfNormalDerivH}-\eqref{eq:VanishingOfNormalDerivS}
are written in terms of intrinsic quantities on $\mathcal{S}$.  To do
so, the space spinor formalism will be exploited. The discussion given
in this section is similar to that of \cite{BaeVal10b}. Notice that,
nevertheless, in the discussion given in \cite{BaeVal10b} the Einstein
field equations are used to simplify expressions associated with the
curvature spinors. In the present analysis the curvature spinors are
subject to the alternative conformal Einstein field equations as
encoded in the zero-quantities
\eqref{ZeroQuantitySecondDerivativeConformalFactor}-\eqref{ZeroQuantityWeyl}.



\subsection{Space spinor decompositions and ancillary identities}
To obtain intrinsic conditions on $\mathcal{S}$ from equations
\eqref{eq:VanishingOfH}-\eqref{eq:VanishingOfNormalDerivS} we start
defining the space spinorial counterpart of the zero quantities
$H_{A'ABC},~S_{AA'BB'}$:
\begin{subequations}
\begin{align} 
 & H_{ABCD} \equiv \tau_{A}{}^{A'}
  H_{A'BCD},\label{DefSpaceSpinorH}\\ & S_{ABCD} \equiv
  \tau_{A}{}^{B'} \tau_{C}{}^{D'} S_{BB'DD'}. \label{DefSpaceSpinorS}
\end{align}
\end{subequations}
 Next, we define the following spinors
 \begin{align*}
& \xi_{AB} \equiv \mathcal{D}_{(A}{}^{D}\kappa_{B)D},\\ & \xi \equiv
   \mathcal{D}^{AB}\kappa_{AB},\\ & \xi_{ABCD} \equiv
   \mathcal{D}_{(AB}\kappa_{CD)},
 \end{align*}
in terms of which we have the following irreducible decomposition
\[H_{CABD} = 3 \xi_{ABCD} + \tfrac{1}{2} \left(\mathcal{P}\kappa_{BD} 
+ \xi_{BD}\right) \epsilon_{AC} + \tfrac{1}{2}\left(
\mathcal{P}\kappa_{AD} + \xi_{AD}\right) \epsilon_{BC} - \tfrac{1}{2}
\left(\mathcal{P}\kappa_{AB}+ \xi_{AB}\right) \epsilon_{CD}.\]
 Additionally, the following
 decompositions will prove useful
\begin{align}
\mathcal{D}_{AB}\kappa_{CD} &= \xi_{ABCD} - \tfrac{1}{2}
\epsilon_{A(C}\xi_{D)B} - \tfrac{1}{2} \epsilon_{B(C}\xi_{D)A} -
\tfrac{1}{3} \xi \epsilon_{A(C}
\epsilon_{D)B}, \label{GradSenDCDeKillingSpinor}
\\ \mathcal{D}_{AB}\xi_{CD} &=\tfrac{1}{6} \epsilon_{AD} \epsilon_{BC}
\mathcal{D}_{FG}\xi^{FG} + \tfrac{1}{6} \epsilon_{AC} \epsilon_{BD}
\mathcal{D}_{FG}\xi^{FG} - \tfrac{1}{4} \epsilon_{BD}
\mathcal{D}_{(A}{}^{F}\xi_{C)F} \nonumber \\ &\qquad - \tfrac{1}{4}
\epsilon_{BC} \mathcal{D}_{(A}{}^{F}\xi_{D)F} - \tfrac{1}{4}
\epsilon_{AD} \mathcal{D}_{(B}{}^{F}\xi_{C)F} - \tfrac{1}{4}
\epsilon_{AC} \mathcal{D}_{(B}{}^{F}\xi_{D)F} +
\mathcal{D}_{(AB}\xi_{CD)}. \label{GradSenDCDeSpaceSpinorAuxiliaryVector}
\end{align}
Using the definition of $\xi_{AB}$, and by commuting derivatives, one
obtains the following identities:
\begin{subequations}
\begin{eqnarray}
&& \mathcal{D}_{AB}\xi^{AB} = -\tfrac{1}{3}\chi\xi+\tfrac{1}{2}
  \chi^{AB} \xi_{AB} -\tfrac{1}{2}\chi^{AB}\mathcal{P} \kappa_{AB}-
  \kappa^{AB} \Phi_{AB} + \tfrac{1}{2} \xi^{ABCD}
  \chi_{ABCD}, \label{SenDivergenceOfSpaceSpinorAuxiliaryVector}\\ &&
  \mathcal{D}_{A(B}\xi_{D)}{}^A =- \tfrac{2}{3}
  \mathcal{D}_{BD}\xi+\mathcal{D}_{AC}\xi_{BD}{}^{AC}+\chi_{(B}{}^A\mathcal{P}\kappa_{D)A}
  + \tfrac{1}{3} \xi \chi_{BD} \nonumber\\ && \qquad \qquad+
  \tfrac{2}{3} \chi \xi_{BD}- \tfrac{1}{2} \chi_{(B}{}^{A}\xi_{D)A}+
  \tfrac{1}{2} \xi^{AC} \chi_{BDAC} - \tfrac{1}{2} \chi^{AC}
  \xi_{BDAC} +\xi_{(B}{}^{ACF}\chi_{D)ACF}\nonumber\\ && \qquad \qquad
  -4 \kappa_{BD} \Lambda - \tfrac{2}{3} \kappa_{BD} \Phi -
  \kappa_{(B}{}^{A}\Phi_{D)A}+ \kappa^{AC} \Theta_{BDAC} + \kappa^{AC}
  \Psi_{BDAC}, \label{SenCurlOfSpaceSpinorAuxiliaryVector}\\ &&
  \mathcal{D}_{(AB}\xi_{CD)}= 2
  \mathcal{D}_{(A}{}^{F}\xi_{BCD)F}+\chi_{(AB}\mathcal{P}\kappa_{CD)}
  \nonumber\\ && \qquad \qquad+\tfrac{2}{3} \chi \xi_{ABCD} -
  \tfrac{1}{3} \xi \chi_{ABCD} + \tfrac{1}{2} \chi_{(AB}\xi_{CD)} -
  \chi_{(A}{}^{F}\xi_{BCD)F} - \xi_{(A}{}^{F}\chi_{BCD)F}
  \nonumber\\ && \qquad \qquad - \xi_{(AB}{}^{FG}\chi_{CD)FG} - 2
  \kappa_{(A}{}^{F}\Theta_{BCD)F} - \kappa_{(AB}\Phi_{CD)} - 2
  \kappa_{(A}{}^{F}\Psi_{BCD)F}, \label{SenGradSymmetrisedOfSpaceSpinorAuxiliaryVector}
\end{eqnarray}
\end{subequations}
and similarly,
\begin{subequations}
\begin{eqnarray}
&& \mathcal{P}\xi
  =-\tfrac{1}{2}K^{AB}\mathcal{P}\kappa_{AB}+\mathcal{D}^{AB}\mathcal{P}\kappa_{AB}-
  \tfrac{1}{3} \chi \xi + K^{AB} \xi_{AB} \nonumber\\ && \qquad \qquad
  \qquad \qquad \qquad \qquad \qquad \qquad \qquad + \chi^{AB}
  \xi_{AB} + 2 \kappa^{AB} \Phi_{AB} - \xi^{ABCD}
  \chi_{ABCD},\label{NormalDerivativeXiDerivedFromCommutator1}\\ &&
  \mathcal{P}\xi_{AB} =4 \kappa_{AB} \Lambda - \tfrac{2}{3}
  \kappa_{AB} \Phi - \kappa^{CD} \Psi_{ABCD} - \tfrac{1}{3} K_{AB} \xi
  - \tfrac{1}{3} \xi \chi_{AB} - \tfrac{1}{3} \chi \xi_{AB} +
  \tfrac{1}{2} K^{CD} \xi_{ABCD} \nonumber \\ && \qquad\qquad +
  \tfrac{1}{2} \chi^{CD} \xi_{ABCD} + \kappa^{CD} \Theta_{ABCD} +
  \tfrac{1}{2} \xi^{CD} \chi_{ABCD} + \tfrac{1}{2}
  K_{(A}{}^{C}\xi_{B)C} - \kappa_{(A}{}^{C}\Phi_{B)C} \nonumber
  \\ &&\qquad\qquad + \tfrac{1}{2} \chi_{(A}{}^{C}\xi_{B)C} +
  \xi_{(A}{}^{CDF}\chi_{B)CDF}+
  \mathcal{D}_{(A}{}^C\mathcal{P}\kappa_{B)C}
  -\tfrac{1}{2}K_{(A}{}^C\mathcal{P}\kappa_{(B)C},
\label{NormalDerivativeXiDerivedFromCommutator2}
\\ &&\mathcal{P}\xi_{ABCD}=\mathcal{D}_{(AB}\mathcal{P}\kappa_{CD)}
-\tfrac{1}{2}K_{(AB}\mathcal{P}\kappa_{CD)} - \tfrac{1}{3} \chi
\xi_{ABCD} - \tfrac{1}{3} \xi \chi_{ABCD} - \tfrac{1}{2}
K_{(AB}\xi_{CD)} \nonumber \\ &&\qquad\qquad+ K_{(A}{}^{F}\xi_{BCD)F}
+ 2 \kappa_{(A}{}^{F}\Psi_{BCD)F} - 2 \kappa_{(A}{}^{F}\Theta_{BCD)F}
- \kappa_{(AB}\Phi_{CD)} \nonumber \\ &&\qquad\qquad - \tfrac{1}{2}
\chi_{(AB}\xi_{CD)} + \chi_{(A}{}^{F}\xi_{BCD)F} -
\xi_{(A}{}^{F}\chi_{BCD)F} -
\xi_{(AB}{}^{FG}\chi_{CD)FG}. \label{NormalDerivativeXiDerivedFromCommutator3}
\end{eqnarray}
\end{subequations}

\begin{remark}
\emph{
If the tracefree Ricci spinor $\Phi_{AA'BB'}$ is made to vanish, then the above identities reduce to 
those given in \cite{BaeVal10b}. This corresponds to setting $\Xi= 1$ in the alternative CFEs.}
\end{remark}

\subsection{The conditions $H_{A'ABC}=\mathcal{P} H_{A'ABC}=0$}
\label{TheHandPHCondition}

Given the irreducible decomposition of the zero quantity $H_{A'ABC}$, provided above, the condition $H_{A'ABC}=0$ is equivalent to
\begin{subequations}
\begin{eqnarray}
&& \xi_{ABCD}=0,\label{ProjectedKillingEquation}\\ &&
  \mathcal{P}\kappa_{AB}=-\xi_{AB}, \label{NormalDerivKillingSpinor}
\end{eqnarray}
\end{subequations}
and the condition $\mathcal{P} H_{A'ABC}=0$ is equivalent to
\[\mathcal{P} H_{ABCD}=\chi_A{}^F H_{FBCD},\]
which, in turn is equivalent to
\begin{subequations}
\begin{eqnarray}
&&
  \mathcal{P}\xi_{ABCD}=0, \label{NormalDerivProjectedKillingEquation}\\ &&
  \mathcal{P}^2\kappa_{AB}=-\mathcal{P}\xi_{AB}. \label{DoubleNormalDerivKappaCondition}
\end{eqnarray}
\end{subequations}
The wave equation for $\kappa_{AB}$, given in
equation (\ref{WaveEqKillingSpinor}), can be rewritten in terms of the quantities
$\xi,~\xi_{AB},~\xi_{ABCD}$
 as follows
\begin{align}
\mathcal{P}^2\kappa_{BC} = & - 2 \mathcal{D}_{AD}\xi_{BC}{}^{AD} -
\tfrac{2}{3} \mathcal{D}_{BC}\xi + 2
\mathcal{D}_{(B}{}^{A}\xi_{C)A}-\chi\mathcal{P}\kappa_{BC}-8
\kappa_{BC} \Lambda + 2 \kappa^{AD} \Psi_{BCAD}\nonumber\\ & +
\tfrac{1}{3} K_{BC} \xi + \tfrac{2}{3} \xi \chi_{BC} + K^{AD}
\xi_{BCAD} + 2 \chi^{AD} \xi_{BCAD} - K_{(B}{}^{A}\xi_{C)A} - 2
\chi_{(B}{}^{A}\xi_{C)A}.
\label{WaveEqForKillingSpinorSolvedForNormalNormalDerivative}
\end{align}
It is important to note that the Killing spinor satisfies equation
(\ref{WaveEqKillingSpinor}) by construction, and therefore \eqref{WaveEqForKillingSpinorSolvedForNormalNormalDerivative}
can be assumed to hold throughout the domain of dependence of
$\mathcal{U}$; in particular, we are free to take further
$\mathcal{P}-$derivatives of the equation.  Through repeated use of the
identities
(\ref{SenDivergenceOfSpaceSpinorAuxiliaryVector})-(\ref{SenGradSymmetrisedOfSpaceSpinorAuxiliaryVector}),
(\ref{NormalDerivativeXiDerivedFromCommutator1})-(\ref{NormalDerivativeXiDerivedFromCommutator3}),
along with
(\ref{ProjectedKillingEquation})--(\ref{NormalDerivKillingSpinor}),
the above wave equation can be seen to imply
(\ref{DoubleNormalDerivKappaCondition}), which is therefore trivially
satisfied. For future reference note that, using equation
\eqref{NormalDerivativeXiDerivedFromCommutator2}, the wave equation for
$\kappa_{AB}$ can alternatively be expressed as
\begin{align}
\mathcal{P}^{2}\kappa_{AB}+\mathcal{P}\xi_{AB}
=&-4\kappa_{AB}\Lambda-\chi\mathcal{P}\kappa_{AB} - \tfrac{2}{3}
\kappa_{AB} \Phi + \kappa^{CD} \Psi_{ABCD} + \tfrac{1}{3} \xi
\chi_{AB} - \tfrac{1}{3} \chi \xi_{AB} + \tfrac{3}{2} K^{CD}
\xi_{ABCD} \nonumber \\ & + \tfrac{5}{2} \chi^{CD} \xi_{ABCD} +
\kappa^{CD} \Theta_{ABCD} + \tfrac{1}{2} \xi^{CD} \chi_{ABCD} -
\tfrac{2}{3} \mathcal{D}_{AB}\xi - 2 \mathcal{D}_{CD}\xi_{AB}{}^{CD}
\nonumber \\ & + 2 \mathcal{D}_{(A}{}^{C}\xi_{B)C} -
2\mathcal{D}_{(A}{}^{C}\mathcal{P}\kappa_{B)C}-\tfrac{1}{2}K_{A}{}^{C}\mathcal{P}\kappa_{B)C}
- \kappa_{(A}{}^{C}\Phi_{B)C} - \tfrac{3}{2} \chi_{(A}{}^{C}\xi_{B)C}
\nonumber \\ & +
\xi_{(A}{}^{CDF}\chi_{B)CDF}. \label{WaveEquationSpaceSpinorUseful}
\end{align}
Finally, using equations (\ref{NormalDerivativeXiDerivedFromCommutator3}), and
 (\ref{SenGradSymmetrisedOfSpaceSpinorAuxiliaryVector}) along
with
(\ref{ProjectedKillingEquation})--(\ref{NormalDerivKillingSpinor}),
one obtains 
\[\mathcal{P}\xi_{ABCD}=\kappa_{(A}{}^F\Psi_{BCD)F}.\]
Therefore, equation \eqref{NormalDerivProjectedKillingEquation} is equivalent
to imposing the Buchdahl constraint, $\kappa_{(A}{}^F\Psi_{BCD)F}=0$, on $\mathcal{U}$.


\subsection{The conditions $S_{AA'BB'}=\mathcal{P} S_{AA'BB'}=0$}


Using the definition of $S_{ABCD}$ given in expression
\eqref{DefSpaceSpinorS} and the expression for $S_{AA'BB'}$ as given
in equation \eqref{DefinitionZeroQuantitySIntersmOfAuxiliaryV}, a
direct computation using the space spinor formalism introduced in
Section \ref{SpaceSpinorFormalism} renders
\begin{align}
\label{eqnSprojectedComplete}
 S_{ABCD} &= K_{C(D|AF|}\mathcal{P}\kappa_{B)}{}^{F}-6
 \kappa_{(D}{}^{F} \Phi_{B)FAC} - \tfrac{1}{2} K_ {ABCD} \xi -
 \tfrac{1}{2} K_ {CDAB} \xi - K_{CDAF} \xi_{B}{}^{F} \nonumber \\ &
 -K_{ABCF} \xi_{D}{}^{F} +\tfrac{1}{4}
 K_{C}{}^{F}\mathcal{P}\kappa_{DF}\epsilon_{AB} - \tfrac{1}{4}
 \mathcal{P}^2 \kappa_{CD} \epsilon_{AB} +
 \tfrac{1}{2}\mathcal{P}\xi_{CD} \epsilon_{AB} + \tfrac{1}{4} K_ {CD}
 \xi \epsilon_{AB} \nonumber \\ & - \tfrac{1}{2} K_{C}{}^{F} \xi_{DF}
 \epsilon_{AB} + \tfrac{1}{4} K_{A}{}^{F}
 \mathcal{P}\kappa_{BF}\epsilon_{CD} - \tfrac{1}{4} \mathcal{P}^2
 \kappa_{AB} \epsilon_{CD} + \tfrac{1}{2}\mathcal{P}\xi_{AB}
 \epsilon_{CD} + \tfrac{1}{4} K_{AB} \xi \epsilon_{CD} \nonumber \\ &
 - \tfrac{1}{2} K_{A}{}^{F} \xi_{BF} \epsilon_{CD}
 -\tfrac{1}{2}\mathcal{P}\xi \epsilon_{AB}\epsilon_{CD} +
 \tfrac{1}{2}\mathcal{D}_{AB}\mathcal{P}\kappa_{CD}+
 \tfrac{1}{2}\mathcal{D}_{CD}\mathcal{P}\kappa_{AB}
 +\tfrac{1}{2}\epsilon_{CD}\mathcal{D}_{AB}\xi \nonumber \\ & +
 \tfrac{1}{2}\epsilon_{AB}\mathcal{D}_{CD}\xi
 -\mathcal{D}_{AB}\xi_{CD}-\mathcal{D}_{CD}\xi_{AB}.
\end{align}
Using the decompositions \eqref{ExtrinsicCurvatureSplit},
\eqref{RicciSpaceSpinorSplit} for $K_{ABCD}$ and $\Phi_{ABCD}$,
equation \eqref{eqnSprojectedComplete} can decomposed in irreducible
components. The non-vanishing components (or combinations thereof) of
this decomposition are:
\begin{subequations}
\begin{align}
S_{(ABCD)} &= - \xi \chi_{ABCD} -2 \mathcal{D}_{(AB}\xi_{CD)} +
\mathcal{D}_{(AB}\mathcal{P}\kappa_{CD)}-6
\kappa_{(A}{}^{F}\Theta_{BCD)F} \nonumber \\ & \quad -3
\kappa_{(AB}\Phi_{CD)} - \chi_{(AB}\xi_{CD)} + \tfrac{1}{2}
\chi_{(AB}\mathcal{P} \kappa_{CD)} -2 \xi_{(A}{}^{F}\chi_{BCD)F}
\nonumber \\ & \quad - \chi_{(ABC}{}^{F}\mathcal{P}\kappa_{D)F},
\label{SIntrinsicInBrute1}
\\
%\nonumber\\
S_{(AB)}{}^{F}{}_{F} - S_{(A}{}^{F}{}_{|F|B)} & =
-\tfrac{1}{2}\mathcal{P}\kappa_{AB} +
\mathcal{P}^2\kappa_{AB}-2\mathcal{P}\xi_{AB}-4 \kappa_{AB} \Phi -
K_{AB} \xi + \tfrac{2}{3} \chi \xi_{AB} \nonumber \\ & \quad+ 6
\kappa^{FC} \Theta_{ABFC}-\mathcal{P}\kappa^{FC}\chi_{ABFC} + 2
\xi^{FC} \chi_{ABFC} -2 \mathcal{D}_{AB}\xi \nonumber \\ & \quad + 2
K_{(A}{}^{F}\xi_{B)F} - K_{(A}{}^{F}\mathcal{D}_{AB}\kappa_{B)F}-6
\kappa_{(A}{}^{F}\Phi_{B)F} -2 \chi_{(A}{}^{F}\xi_{B)F} \nonumber \\ &
\quad+
\chi_{(A}{}^{F}\mathcal{P}\kappa_{B)F}, \label{SIntrinsicInBrute2} \\
%\nonumber\\
S^{FG}{}_{FG} + S^{FG}{}_{GF} & = -2 \chi \xi
-2\chi^{FG}\mathcal{P}\kappa_{FG} + 4 \chi^{FG} \xi_{FG} -6
\kappa^{FG} \Phi_{FG} + 2 \mathcal{D}_{FG}\mathcal{P}\kappa^{FG}
\nonumber \\ & \quad - 4
\mathcal{D}_{FG}\xi^{FG}, \label{SIntrinsicInBrute3} \\
%\nonumber\\
S^{FG}{}_{FG} - S^{FG}{}_{GF} &= - K^{FG}
\mathcal{P}\kappa_{FG}-2\mathcal{P}\xi+ 2 K^{FG} \xi_{FG} + 6
\kappa^{FG} \Phi_{FG}.
%S^{F}{}_{F}{}^{G}{}_{G}=
\label{SIntrinsicInBrute4}
%\\
%\nonumber\\
%S_{(AB)}{}^{F}{}_{F} +  S_{(A}{}^{F}{}_{|F|B)} & =0
%\label{SIntrinsicInBrute5}
%\\
%\nonumber\\
%S^{FG}{}_{FG}  + S^{FG}{}_{GF} &=0
%\label{SIntrinsicInBrute6}
\end{align}
\end{subequations}
Note that in deriving expressions
\eqref{SIntrinsicInBrute1}-\eqref{SIntrinsicInBrute4} the
$H_{ABCD}|_{\mathcal{S}}=0$ and $\mathcal{P}H_{ABCD}|_{\mathcal{S}}=0$
conditions have not been used. Taking into account the
$H_{ABCD}|_{\mathcal{S}}=0$ conditions, encoded in equations
\eqref{ProjectedKillingEquation}-\eqref{NormalDerivKillingSpinor} and
\eqref{NormalDerivProjectedKillingEquation}-\eqref{DoubleNormalDerivKappaCondition},
and exploiting equations
\eqref{SenDivergenceOfSpaceSpinorAuxiliaryVector}-\eqref{SenGradSymmetrisedOfSpaceSpinorAuxiliaryVector},
the conditions encoded in
\eqref{SIntrinsicInBrute1}-\eqref{SIntrinsicInBrute4} reduce to
\begin{subequations}
\begin{eqnarray}
&& \mathcal{P}\xi = \tfrac{3}{2}K^{FG}\xi_{FG}+3\kappa^{FG}\Phi_{FG},
\label{SuperfluousConditionS1} \\
%\nonumber\\
&& \mathcal{P}\xi_{AB} =\tfrac{2}{3} \mathcal{D}_{AB}\xi- \tfrac{4}{3}
\kappa_{AB} \Phi - \tfrac{1}{3} K_{AB} \xi + \tfrac{1}{3} \chi
\xi_{AB} + K_{(A}{}^{F} \xi_{B)F} \nonumber\\ &&\qquad\qquad -
\chi_{(A}{}^{F} \xi_{B)F} +2 \kappa^{FC} \Theta_{ABFC} -
2\kappa_{(A}{}^{F} \Phi_{B)F} + \xi^{FC} \chi_{ABFC}
, \label{SuperfluousConditionS2}\\
%\nonumber\\
&&
\kappa_{(A}{}^{F}\Psi_{BCD)F}=0.\label{BuchdahlConstraintFromVanishingS}
\end{eqnarray}
\end{subequations}
Furthermore, one can verify, using the identities
\eqref{NormalDerivativeXiDerivedFromCommutator1}-\eqref{NormalDerivativeXiDerivedFromCommutator2},
that equations
\eqref{SuperfluousConditionS1}-\eqref{SuperfluousConditionS2} are
identically satisfied if the intrinsic conditions
\eqref{ProjectedKillingEquation}-\eqref{NormalDerivKillingSpinor} and
\eqref{NormalDerivProjectedKillingEquation}
-\eqref{DoubleNormalDerivKappaCondition} hold.  Additionally, observe
that, as discussed in Section \ref{TheHandPHCondition}, the vanishing
of the Buchdahl constraint \eqref{BuchdahlConstraintFromVanishingS} is
obtained through condition
\eqref{NormalDerivProjectedKillingEquation}. In other words, the
$S_{ABCD}|_{\mathcal{S}}=0$ requirement does not impose any extra
conditions than those already encoded in $H_{ABCD}|_{\mathcal{S}}=0$
and $\mathcal{P}H_{ABCD}|_{\mathcal{S}}=0$.

\medskip

Now, to analyse the conditions imposed by requiring
$\mathcal{P}S_{ABCD}=0$, observe that
\begin{equation}
\tau_{A}{}^{B'}\tau_{C}{}^{D'}\mathcal{P}S_{BB'DD'}=\mathcal{P}S_{ABCD}-K^{F}{}_{C}S_{ABFD}-K^{F}{}_{A}S_{CDFB}.
\end{equation}
 Consequently, if the conditions $S_{ABCD}|_{\mathcal{S}}=0$ hold,
 then, it is enough to analyse the restriction imposed by
 $\mathcal{P}S_{ABCD}|_{\mathcal{S}}=0$. Taking a
 $\mathcal{P}$-derivative of equations
 \eqref{SIntrinsicInBrute1}-\eqref{SIntrinsicInBrute4} and exploiting
 the space spinor formalism one obtains
\begin{subequations}
\begin{align}
\mathcal{P}S_{(ABCD)} &=-\xi\mathcal{P}\chi_{ABCD} -
\chi_{ABCD}\mathcal{P}\xi-2\mathcal{P}\mathcal{D}_{(AB}\xi_{CD)}
+\mathcal{P}\mathcal{D}_{(AB}\mathcal{P}\kappa_{CD)} \nonumber \\ &
\quad -6\kappa_{(A}{}^{F}\mathcal{P}\Theta_{BCD)F}
-3\kappa_{(AB}\mathcal{P}\Phi_{CD)}-\chi_{(AB}\mathcal{P}\xi_{CD)} +
\tfrac{1}{2}\chi_{(AB}\mathcal{P}^{2}\kappa_{CD)} \nonumber \\ & \quad
-\xi_{(AB}\mathcal{P}\chi_{CD)}
-2\xi_{(A}{}^{F}\mathcal{P}\chi_{BCD)F}
+6\Theta_{(ABC}{}^{F}\mathcal{P}\kappa_{D)F}
-3\Phi_{(AB}\mathcal{P}\kappa_{CD)} \nonumber \\ & \quad
+2\chi_{(ABC}{}^{F}\mathcal{P}\xi_{D)F}-\chi_{(ABC}{}^{F}\mathcal{P}^{2}\kappa_{D)F}
+ \tfrac{1}{2}\mathcal{P}\kappa_{(AB}\mathcal{P}\chi_{CD)} \nonumber
\\ & \quad
+\mathcal{P}\kappa_{A}{}^{F}\mathcal{P}\chi_{BCD)F}, \label{PSIntrinsicInBrute1}
\\
%\nonumber\\
 \mathcal{P}(S_{(AB)}{}^{F}{}_{F} - S_{(A}{}^{F}{}_{|F|B)}) &=
 \mathcal{P}^{3}\kappa_{AB}-2\mathcal{P}^{2}\xi_{AB}
 -\tfrac{1}{3}\mathcal{P}\kappa_{AB}\mathcal{P}\chi -\tfrac{1}{3}\chi
 \mathcal{P}^2\kappa_{AB} -4\kappa_{AB}\mathcal{P}\Phi
 -K_{AB}\mathcal{P}\xi \nonumber \\ & \quad
 +\tfrac{2}{3}\chi\mathcal{P}\xi_{AB}+6\kappa^{FC}\mathcal{P}\Theta_{ABFC}
 -\mathcal{P}\kappa^{FC}\mathcal{P}\chi_{ABFC}-2\mathcal{P}\mathcal{D}_{AB}\xi
 -4\Phi\mathcal{P}\kappa_{AB} \nonumber \\ & \quad
%%%%%%%%%%%%%%%%%%%%%%%%%%%%%%%%%%%%%%%%%%%%%%%%%%%%%%%%%%%%%%%%%%%%%%%%
%  negative space added to equation so it can fit nicely
%%%%%%%%%%%%%%%%%%%%%%%%%%%%%%%%%%%%%%%%%%%%%%%%%%%%%%%%%%%%%%%%%%%%%%%
 -\hspace{-0.5mm}\xi\mathcal{P}K_{AB}\hspace{-0.5mm}+\hspace{-0.5mm}
 \tfrac{2}{3}\xi_{AB}\mathcal{P}\chi\hspace{-0.5mm} +\hspace{-0.5mm}
 2\xi^{FC}\mathcal{P}\chi_{ABFC}
 +6\Theta_{ABFC}\mathcal{P}\kappa^{FC}\hspace{-1.5mm}-\hspace{-0.5mm}
\chi_{ABFC}\mathcal{P}\xi^{FC}
 \nonumber \\ & \quad +
 2K_{(A}{}^{F}\mathcal{P}\xi_{B)F}-K_{(A}{}^{F}\mathcal{P}^{2}\kappa_{B)F}
 -6\kappa_{(A}{}^{F}\mathcal{P}\Phi_{B)F}-2\chi_{(A}{}^{F}\mathcal{P}\xi_{B)F}
 \nonumber \\ & \quad +\chi_{(A}{}^{F}\mathcal{P}^{2}\kappa_{B)F}
 -2\xi_{(A}{}^{F}\mathcal{P}K_{B)F}+2\xi_{(A}{}{}^{F}\mathcal{P}\chi_{B)F}
+6\Phi_{(A}{}^{F}\mathcal{P}\kappa_{B)F}
 \nonumber \\ & \quad
 -\mathcal{P}K_{(A}{}^{F}\mathcal{P}\kappa_{B)F}
-\mathcal{P}\kappa_{(A}{}^{F}\mathcal{P}\chi_{B)F},
\label{PSIntrinsicInBrute2}
\\
%\nonumber\\
%%%%%%%%%%%%%%%%%%%%%%%%%%%%%%%%%%%%%%%%%%%%%%%%%%%%%%%%%%%%%%%%%%%%%%%%
%  negative space added to equation so it can fit nicely
%%%%%%%%%%%%%%%%%%%%%%%%%%%%%%%%%%%%%%%%%%%%%%%%%%%%%%%%%%%%%%%%%%%%%%%
\mathcal{P}(S^{FG}{}_{FG} + S^{FG}{}_{GF}) &=-2\chi\mathcal{P}\xi
\hspace{-0.5mm} - \hspace{-0.5mm}
2\mathcal{P}\kappa^{FG}\mathcal{P}\chi_{FG}
\hspace{-0.5mm}-\hspace{-0.5mm}
6\kappa^{FG}\mathcal{P}\Phi_{FG}+2\mathcal{P}\mathcal{D}_{FG}\mathcal{P}\kappa^{FG}
\hspace{-1.5mm}- \hspace{-0.5mm}4\mathcal{P}\mathcal{D}_{FG}\xi^{FG}
\nonumber \\ & \quad - 2\xi\mathcal{P}\chi
-2\chi^{FG}\mathcal{P}^{2}\kappa_{FG}+4\chi^{FG}\mathcal{P}\xi_{FG}
+4\xi^{FG}\mathcal{P}\chi_{FG}-6\Phi^{FG}\mathcal{P}\kappa_{FG},
\label{PSIntrinsicInBrute3}
\\
%\nonumber\\
\mathcal{P}(S^{FG}{}_{FG} - S^{FG}{}_{GF})
&=-\mathcal{P}K^{FG}\mathcal{P}\kappa_{FG}-K^{FG}\mathcal{P}^{2}\kappa_{FG}-2\mathcal{P}^{2}\xi+2K^{FG}\mathcal{P}\xi_{FG}
+ 6\kappa^{FG}\mathcal{P}\Phi_{FG} \nonumber \\ & \quad
+2\xi^{FG}\mathcal{P}K_{FG} + 6\Phi^{FG}\mathcal{P}\kappa_{FG}.
\label{PSIntrinsicInBrute4}
\end{align}
\end{subequations}

\noindent Observe that, in deriving equations
\eqref{PSIntrinsicInBrute1}-\eqref{PSIntrinsicInBrute4}, the
conditions $H_{ABCD}|_{\mathcal{S}}=0$ and
$\mathcal{P}H_{ABCD}|_{\mathcal{S}}=0$ were not used.  Similar to the
discussion leading to equations
\eqref{SuperfluousConditionS1}-\eqref{BuchdahlConstraintFromVanishingS},
exploiting $H_{ABCD}|_{\mathcal{S}}=0$ and
$\mathcal{P}H_{ABCD}|_{\mathcal{S}}=0$ leads to simpler
expressions. To implement this computation it will proof convenient to
derive some ancillary results first. This is done in
the following.

\medskip

Applying the commutator \eqref{CommutatorNormalSenDeriv} to
$\mathcal{P}\kappa_{CD}$ and exploiting the intrinsic conditions
encoded in \eqref{NormalDerivKillingSpinor} and
\eqref{DoubleNormalDerivKappaCondition} one obtains
\begin{equation*}
\mathcal{P}\mathcal{D}_{AB}\mathcal{P}\kappa_{CD} +
\mathcal{D}_{AB}\mathcal{P}\xi_{CD}
=\tfrac{1}{2}K_{AB}\mathcal{P}\xi_{CD}+\square_{AB}\xi_{CD}-\widehat{\square}_{AB}\xi_{CD}
-K_{(A}{}^{F}\mathcal{D}_{B)F}\xi_{CD} +
K_{ABFG}\mathcal{D}^{FG}\xi_{CD}.
\end{equation*}
Similarly, using again the commutator \eqref{CommutatorNormalSenDeriv}
applied now to $\xi_{CD}$ and exploiting the intrinsic conditions
encoded in \eqref{NormalDerivKillingSpinor} and
\eqref{DoubleNormalDerivKappaCondition} one obtains
\begin{equation*}
\mathcal{P}\mathcal{D}_{AB}\xi_{CD} -
\mathcal{D}_{AB}\mathcal{P}\xi_{CD}
=-\tfrac{1}{2}K_{AB}\mathcal{P}\xi_{CD}-\square_{AB}\xi_{CD}+\widehat{\square}_{AB}\xi_{CD}
+K_{(A}{}^{F}\mathcal{D}_{B)F}\xi_{CD} -
K_{ABFG}\mathcal{D}^{FG}\xi_{CD}.
\end{equation*}
Comparing the last two expressions one concludes that
\begin{equation}
 \mathcal{P}\mathcal{D}_{AB}\mathcal{P}\kappa_{CD}+\mathcal{P}\mathcal{D}_{AB}\xi_{CD}=0.
\label{NotObviousExpressionForThirdDerivatives}
\end{equation}
Additionally, observe that taking a $\mathcal{P}-$derivative to
equations
\eqref{NormalDerivativeXiDerivedFromCommutator1}-\eqref{NormalDerivativeXiDerivedFromCommutator2}
and exploiting equations
\eqref{SenDivergenceOfSpaceSpinorAuxiliaryVector}-\eqref{SenCurlOfSpaceSpinorAuxiliaryVector}
along with conditions
\eqref{ProjectedKillingEquation}-\eqref{NormalDerivKillingSpinor} and
\eqref{NormalDerivProjectedKillingEquation}-\eqref{DoubleNormalDerivKappaCondition}
we obtain
\begin{align}
\mathcal{P}^{2}\xi
=&\tfrac{3}{2}K^{AB}\mathcal{P}\xi_{AB}+3\kappa^{AB}\mathcal{P}\Phi_{AB}
+
\tfrac{3}{2}\xi^{AB}\mathcal{P}K_{AB}-3\xi^{AB}\Phi_{AB}, \label{SecondNormalDerivativeXi}
\\ \mathcal{P}^{2}\xi_{AB} =&
-\tfrac{4}{3}\kappa_{AB}\mathcal{P}\Phi-\tfrac{1}{3}K_{AB}\mathcal{P}\xi
+\tfrac{1}{3}\chi\mathcal{P}\xi_{AB}+K_{(B}{}^{C}\mathcal{P}\xi_{A)C}+2\kappa^{CD}\mathcal{P}\Theta_{ABCD}
\nonumber \\ & -2\kappa_{(B}{}^{C}\mathcal{P}\Phi_{A)C}
-\tfrac{2}{3}\mathcal{P}\mathcal{D}_{AB}\xi +
\tfrac{1}{3}\xi\mathcal{P}K_{AB}
-\chi_{(A}{}^{C}\mathcal{P}\xi_{B)C}+\tfrac{1}{3}\xi_{AB}\mathcal{P}\chi
+ \tfrac{4}{3}\Phi\xi_{AB} \nonumber \\ &
-\xi_{(A}{}^{C}\mathcal{P}K_{B)C} +\xi_{(A}{}^{C}\mathcal{P}\chi_{B)C}
+\xi^{CD}\mathcal{P}\chi_{ABCD}-2\xi^{CD}\Theta_{ABCD}+\xi_{A}{}^{C}\Phi_{BC}
\nonumber \\ &
+\chi_{ABCD}\mathcal{P}\xi^{CD}. \label{SecondNormalDerivativeXiSym}
\end{align}
 Using the above expressions along with equations
 \eqref{NotObviousExpressionForThirdDerivatives},
 \eqref{GradSenDCDeSpaceSpinorAuxiliaryVector},
 \eqref{SenDivergenceOfSpaceSpinorAuxiliaryVector}-\eqref{SenGradSymmetrisedOfSpaceSpinorAuxiliaryVector},
 \eqref{NormalDerivativeXiDerivedFromCommutator1}-\eqref{NormalDerivativeXiDerivedFromCommutator3},
 and the $H_{ABCD}|_{\mathcal{S}}$ and
 $\mathcal{P}H_{ABCD}|_{\mathcal{S}}=0$ conditions encoded in
 equations
 \eqref{ProjectedKillingEquation}-\eqref{NormalDerivKillingSpinor} and
 \eqref{NormalDerivProjectedKillingEquation}-\eqref{DoubleNormalDerivKappaCondition}
 we obtain
\begin{subequations}
\begin{eqnarray}
%\mathcal{P}S_{(ABCD)} &=
&& 6\kappa_{(A}{}^{F}\mathcal{P}\Psi_{BCD)F} + 6\Psi_{(ABC}{}^{F}\xi_{D)F}=0,\label{PSIntrinsic1}
\\
% \mathcal{P}(S_{(AB)}{}^{F}{}_{F} -  S_{(A}{}^{F}{}_{|F|B)})  &=
&& \mathcal{P}^{3}\kappa_{AB}+\mathcal{P}^{2}\xi_{AB}=0.
\label{PSIntrinsic2}
%\mathcal{P}(S^{FG}{}_{FG}  +  S^{FG}{}_{GF}) &= 0\label{PSIntrinsic3}
%\\
%\nonumber\\
%\mathcal{P}(S^{FG}{}_{FG} - S^{FG}{}_{GF}) &= 0
\label{PSIntrinsic4}
\end{eqnarray}
\end{subequations}
To simplify equation \eqref{PSIntrinsic2} we can exploit the wave
equation for $\kappa_{AB}$ as expressed in equation
\eqref{WaveEquationSpaceSpinorUseful}. Taking a
$\mathcal{P}-$derivative of the latter equations and using the
identities
\eqref{SenDivergenceOfSpaceSpinorAuxiliaryVector}-\eqref{SenGradSymmetrisedOfSpaceSpinorAuxiliaryVector}
\eqref{NormalDerivativeXiDerivedFromCommutator1}-\eqref{NormalDerivativeXiDerivedFromCommutator3},
and the $H_{ABCD}|_{\mathcal{S}}$ and
$\mathcal{P}H_{ABCD}|_{\mathcal{S}}=0$ conditions encoded in equations
\eqref{ProjectedKillingEquation}-\eqref{NormalDerivKillingSpinor} and
\eqref{NormalDerivProjectedKillingEquation}-\eqref{DoubleNormalDerivKappaCondition}
one obtains
\[
\mathcal{P}^{3}\kappa_{AB}+\mathcal{P}^{2}\xi_{AB}=0.
\]
Consequently, equation \eqref{PSIntrinsic1} contains the only
independent condition encoded by $\mathcal{P}S_{AA'BB'}|_{\mathcal{S}}=0$.
 Finally, one can exploit the conformal field equation encoded
in the zero-quantity $\eqref{ZeroQuantityWeyl}$ to express the
$\mathcal{P}$ derivative of the Weyl spinor in terms of intrinsic
quantities at $\mathcal{S}$.  To do so, let
\begin{align*}
\Lambda_{ABCD}\equiv \tau_{A}{}^{A'}\Lambda_{A'BCD}, \\ Y_{ABCD}\equiv
\tau_{D}{}^{D'}Y_{ABCD'}.
\end{align*}
Exploiting the space spinor formalism one obtains
\begin{equation*}
\Lambda_{ABCD} =\tfrac{1}{2}\mathcal{P}\Psi_{ABCD}
-\tfrac{1}{2}Y_{ABCD}+\mathcal{D}_{QD}\Psi_{ABC}{}^{Q},
\end{equation*}
from which one obtains evolution and constraint equations encoded in
\[ \Lambda_{(ABCD)}=0, \qquad  \Lambda_{AB}{}^{Q}{}_{Q}=0, \]
given explicitly by
\begin{subequations}
\begin{eqnarray}
&& \mathcal{P}\Psi_{ABCD}+2\mathcal{D}_{Q(D}\Psi_{ABC)}{}^{Q}
  -Y_{ABCD}=0,\label{ZeroQuantityWeylEvolution}\\
&&  \mathcal{D}^{PQ}\Psi_{PQAB}-\tfrac{1}{2}Y_{AB}{}^{Q}{}_{Q}=0.\label{ZeroQuantityWeylConstraint}
\end{eqnarray}
\end{subequations}

Using the evolution equation encoded in expression
\eqref{ZeroQuantityWeylEvolution}, the condition given in equation
\eqref{PSIntrinsic1} reads
\begin{equation}\label{NonTrivialCondition}
\mathcal{P}S_{(ABCD)} =6 \kappa_{(A}{}^{F}Y_{BCD)F} + 12
\kappa_{(A}{}^{F}\mathcal{D}_{|F|}{}^{G}\Psi_{BCD)G} + 6
\Psi_{(ABC}{}^{F}\xi_{D)F}
\end{equation}
\\ We are now in a position to formulate the main Theorem of this
article:
\begin{theorem}\label{MainTheorem}
Consider an initial data set for the (alternative) conformal Einstein
field equations, as encoded in the zero-quantities
\eqref{ZeroQuantitySecondDerivativeConformalFactor}-\eqref{ZeroQuantityWeyl},
on a spacelike hypersurface $\mathcal{S}$ and let
$\mathcal{U}\subset\mathcal{S}$ denote an open set.  The development
of the initial data set will have a Killing spinor in the domain of
dependence of $\mathcal{U}$ if and only if
\begin{subequations}
\begin{flalign}
&\mathcal{D}_{(AB}\kappa_{CD)}=0, \label{CS-KID1}\tag{C1}
  \\ &\kappa_{(A}{}^{F}\Psi_{BCD)F}=0, \label{CS-KID2}\tag{C2}\\ &\kappa_{(A}{}^{F}Y_{BCD)F}
  + 2 \kappa_{(A}{}^{F}\mathcal{D}_{|F|}{}^{G}\Psi_{BCD)G} +
  \Psi_{(ABC}{}^{F}\xi_{D)F}=0,\label{CS-KID3}\tag{C3}
\end{flalign}
\end{subequations}
are satisfied on $\mathcal{U}$. The Killing spinor is obtained
evolving according to the wave equation \eqref{WaveEqKillingSpinor}
with initial data satisfying conditions
\eqref{CS-KID1}-\eqref{CS-KID3} and
\begin{equation}
\mathcal{P}\kappa_{AB}=-\xi_{AB}.
\end{equation}
\end{theorem}
\begin{proof}
The prior discussion of this section establishes that the conditions
\begin{eqnarray*}
&& H_{A'ABC}=0,\\ 
&& \mathcal{P}
  H_{A'ABC}=0,\\
&& S_{AA'BB'}=0,\\ 
&&\mathcal{P}S_{AA'BB'}=0 
\end{eqnarray*}
on $\mathcal{U}\subset \mathcal{S}$ are equivalent to \eqref{CS-KID1}--\eqref{CS-KID3}. 
Hence, appealing to Proposition \ref{Prop:Propagation}, we see that if \eqref{CS-KID1}--\eqref{CS-KID3}
hold on $\mathcal{U}$, then the domain of dependence of $\mathcal{U}$ is endowed with a Killing spinor.  
\end{proof}
\begin{definition}
\emph{ The equations (\ref{CS-KID1})-(\ref{CS-KID3}) will be referred
  to as the \emph{conformal Killing spinor initial data equations},
  and a solution, $\kappa_{AB}$, thereof a \emph{Killing spinor
    candidate}.  }
\end{definition}
\begin{remark}\label{Remark:ConformalKIDsOverdetermined}
\emph{The conditions (\ref{CS-KID1})-(\ref{CS-KID3}) are a
  highly-overdetermined system of equations. It therefore follows
  that, while they are to be read as equations for the Killing spinor
  candidate, $\kappa_{AB}$, the existence of a non-trivial solution to
  the these equations places strong restrictions on the initial data and,
  consequently, on the resulting spacetime.  Observe that
  \eqref{CS-KID2} implies that the restriction of the Weyl spinor to
  $\mathcal{S}$ is algebraically special.
It will be seen in Section \ref{Sec:FurtherAnalysis} that, equation,
(\ref{CS-KID3}) places further constraints on curvature associated to
initial data for the (alternative) CFEs, in the sense of restricting
various components of the Cotton spinor, when expressed in terms of a
suitably-adapted spin dyad.  }
\end{remark}
\begin{remark}
\emph{
While the analysis in this article is carried out via the spinor
formalism, we remark here that the main results could  alternatively be
rewritten in tensorial terms; the above Theorem may be reframed in
terms of the existence of a Killing--Yano tensor (rather than of a
Killing spinor) on the spacetime development.
}
\end{remark}

The conditions \eqref{CS-KID1}-\eqref{CS-KID3} were derived
  from \eqref{eq:VanishingOfH}-\eqref{eq:VanishingOfNormalDerivS}
  exploiting the space spinor formalism adapted to a timelike
  Hermitian spinor $\tau^{AA'}$ corresponding to the normal vector to
  the initial hypersurface $\mathcal{S}$. Nevertheless, conditions
  \eqref{eq:VanishingOfH}-\eqref{eq:VanishingOfNormalDerivS} are
  irrespective of the causal nature of $\mathcal{S}$ , consequently, a
  similar analysis to that given in Section
  \ref{Sec:IntrinsicConditions} can be used to identify spinorial
  Killing initial data for the conformal Einstein field equations on a
  timelike or null hypersurface as well. 

 The initial hypersurface $\mathcal{S}$ can be chosen to
  determined by the condition $\Xi=0$ so that $\mathcal{S}$
  corresponds to the conformal boundary $\mathscr{I}$.  In this case,
  conditions \eqref{CS-KID1}-\eqref{CS-KID3} provide with conditions
  on \emph{asymptotic initial data} that ensure the existence of a
  Killing spinor in the development of this data.  This Killing spinor
  can be used to construct a conformal Killing vector in the
  unphysical spacetime $(\mathcal{M},\bmg)$ corresponding to a Killing
  vector of the physical spacetime
  $(\tilde{\mathcal{M}},\tilde{\bmg})$ ---see Lemma
  \ref{LemmaRelationKillingSpinorConformalKillingVector}. On the other hand, 
  setting $\Xi= 1$, so that we have Cauchy data for the Einstein
  field equations, the Cotton tensor (spinor) vanishes and conditions
  (\ref{CS-KID1})-(\ref{CS-KID3}) reduce to the conditions given in
  \cite{BaeVal10a,BaeVal10b,BaeVal11b}. Note that, while condition
  (\ref{CS-KID3}) trivialises in this case as it to follow as a
  consequence of (\ref{CS-KID1})-(\ref{CS-KID2}) ---see
  \cite{BaeVal12} for a detailed discussion of this.  Nevertheless, in
  the general case ($\Xi\neq 1$), (\ref{CS-KID3}) encodes non-trivial
  information about the Cotton spinor and cannot be eliminated by
  virtue of the conditions (\ref{CS-KID1})-(\ref{CS-KID2}) alone
  ---see Remark \ref{Remark:ConformalKIDsOverdetermined}.


\section{Further analysis of the conformal Killing spinor initial data equations}\label{Sec:FurtherAnalysis}

In this section the conditions \eqref{CS-KID1}-\eqref{CS-KID3} are
further analysed by expressing them in components with respect to a
spin dyad adapted to the Killing spinor $\kappa_{AB}$.  The ultimate
goal of this section is to show that, in contrast to the $\Xi= 1$ case
---see \cite{BaeVal12}, the condition \eqref{CS-KID3} is in general
non-trivial; that is to say that it does not follow as a consequence
of conditions \eqref{CS-KID1}-\eqref{CS-KID2}. Rather, we will see
that \eqref{CS-KID3} captures essential information about the Cotton
spinor, and may only be eliminated from the conformal Killing spinor
initial data equations by additionally constraining certain components
of the Cotton spinor.

\medskip 

Recalling that $Y_{ABCD}=Y_{(ABC)D}$ let $Y_{\bmi}$ with $\bmi=\bm0,...,\bm7$
 denote the components of $Y_{ABCD}$ respect to a spin dyad, $\lbrace o^{A},
\iota^{A}\rbrace$, normalised as $o_{A}\iota^{A}=1$.
In other words, let
\begin{align*}
& Y_{\bm0} =\iota^A\iota^B\iota^C\iota^D Y_{ABCD} , && Y_{\bm4} =
  \iota^A o^B o^C\iota^D Y_{ABCD},\\ & Y_{\bm1} =\iota^A\iota^B\iota^C
  o^D Y_{ABCD} , && Y_{\bm5} =\iota^A o^B o^C o^D Y_{ABCD} ,\\ &
  Y_{\bm2} =\iota^A\iota^B o^C\iota^D Y_{ABCD} , && Y_{\bm6} =o^A o^B
  o^C\iota^D Y_{ABCD} ,\\ & Y_{\bm3} =\iota^A\iota^B o^C o^D Y_{ABCD}
  , && Y_{\bm7} = o^A o^B o^C o^D Y_{ABCD}.
\end{align*}
Using the latter notation $Y_{ABCD}$ is expressed as follows
\begin{align}\label{CottonAdaptedDyad}
Y_{ABCD} &= Y_{\bm0} o_{A} o_{B} o_{C} o_{D} - Y_{\bm1} o_{A} o_{B}
o_{C} \iota_{D} - 3 Y_{\bm2} o_{D} o_{(A}o_{B}\iota_{C)}+ 3 Y_{\bm3}
\iota_{D} o_{(A}o_{B}\iota_{C)} \nonumber \\ & \quad + 3 Y_{\bm4}
o_{D} o_{(A}\iota_{B}\iota_{C)} - 3 Y_{\bm5} \iota_{D}
o_{(A}\iota_{B}\iota_{C)}- Y_{\bm6} o_{D} \iota_{A} \iota_{B}
\iota_{C}+ Y_{\bm7} \iota_{A} \iota_{B} \iota_{C} \iota_{D}.
\end{align}

The results of this section are summarised in the following Proposition:

\begin{proposition}\label{PropositionRestrictionOnCotton}
If $\kappa_{AB}\kappa^{AB}\neq 0$ then there exists a dyad, $\lbrace
o,\iota\rbrace$, and some real-valued function $\varkappa$ for which
\[\kappa_{AB}=e^\varkappa o_{(A}\iota_{B)}.\]
In terms of this adapted dyad, and assuming
(\ref{CS-KID1})--(\ref{CS-KID2}), the condition \eqref{CS-KID3} is
then equivalent to
\[Y_{\bm0}=Y_{\bm1}=Y_{\bm6}=Y_{\bm7}=0.\]
On the other hand, if $\kappa_{AB}\kappa^{AB}= 0$ then
 there exists a dyad, $\lbrace o,\iota\rbrace$, for 
which $\kappa_{AB}=o_Ao_B$, in terms of which condition 
\eqref{CS-KID3} is equivalent to 
\[
Y_{\bm2}=Y_{\bm3}=Y_{\bm4}=Y_{\bm5}=Y_{\bm6}=Y_{\bm7}=0.
\]
\end{proposition}
Cases $i)$ and $ii)$ are dealt with separately in the remainder of
this section.
\begin{remark}
\emph{Note that if the spacetime is of Type O, i.e. $\Psi_{ABCD}=0 $,
  then it follows from the conformal field equations (namely, the
  equation $\Lambda_{A'ABC}=0$) that $Y_{ABCC'}=0$ and hence that
  (\ref{CS-KID2}), (\ref{CS-KID3}) trivialise, leaving only
  (\ref{CS-KID1}).}
\end{remark}



\subsection{Type D Case: $\kappa_{AB}\kappa^{AB}\neq 0$}
\label{TypeD}

If  $\kappa_{AB}\kappa^{AB}\neq 0$  then one can choose a normalised spin dyad $\{o_{A},\iota_{B}\}$
with $o_{A}\iota^{A}=1$, adapted to $\kappa_{AB}$. In other words, such that
\begin{equation} \label{KillingSpinorAdaptedDyadOmicronIota}
\kappa_{AB}=e^\varkappa o_{(A}\iota_{B)},
\end{equation}
where $\varkappa$ is a scalar field.
Similarly, condition \eqref{CS-KID2} implies that
\begin{equation}\label{WeylAdaptedDyadOmicronIota}
\Psi_{ABCD}=\psi o_{(A}o_{B}\iota_{C}\iota_{D)},
\end{equation}
where $\psi$ is a scalar field.  Using these expressions condition
\eqref{CS-KID1} implies the following equations
\begin{subequations}
\begin{eqnarray}
&& o^{A} o^{B} o^{C} \mathcal{D}_{BC}o_{A} = 0, \label{EqA1a} \\ &&
  o^{A} o^{B} \mathcal{D}_{AB}\varkappa = -2 o^{A} o^{B} \iota^{C}
  \mathcal{D}_{BC}o_{A}, \label{EqA1b} \\ && o^{A} \iota^{B}
  \mathcal{D}_{AB}\varkappa = \tfrac{1}{2} o^{A} o^{B} \iota^{C}
  \mathcal{D}_{AB}\iota_{C} - \tfrac{1}{2} o^{A} \iota^{B} \iota^{C}
  \mathcal{D}_{BC}o_{A} \label{EqA1c}, \\ && \iota^{A} \iota^{B}
  \mathcal{D}_{AB}\varkappa = 2 o^{A} \iota^{B} \iota^{C}
  \mathcal{D}_{AC}\iota_{B}, \label{EqA1dd} \\ && \iota^{A} \iota^{B}
  \iota^{C} \mathcal{D}_{BC}\iota_{A} = 0.  \label{EqA1d}
\end{eqnarray}
\end{subequations}
Additionally, using equation
\eqref{KillingSpinorAdaptedDyadOmicronIota} the spinor $\xi_{AB}$ can
be expressed as
\begin{align}\label{SpacespinorAuxiliaryVectorInDyadOmicronIota}
e^{-\varkappa}\xi_{AB} &= \tfrac{1}{2}
o_{(A}\mathcal{D}_{B)}{}^{C}\iota_{C} - \tfrac{1}{2}
o^{C}\mathcal{D}_{(A|C|}\iota_{B)} + \tfrac{1}{2}
\iota_{(A}\mathcal{D}_{B)}{}^{C}o_{C} - \tfrac{1}{2}
\iota^{C}\mathcal{D}_{(A|C|}o_{B)} \nonumber \\ &\quad - \tfrac{1}{2}
o_{(A}\iota^{C}\mathcal{D}_{B)C}\varkappa - \tfrac{1}{2}
o^{C}\iota_{(A}\mathcal{D}_{B)C}\varkappa.
\end{align}
Using equations \eqref{WeylAdaptedDyadOmicronIota} and
\eqref{CottonAdaptedDyad} the constraint equations encoded in
$\Lambda_{AB}{}^{Q}{}_{Q}=0$ as given by
\eqref{ZeroQuantityWeylConstraint} imply
\begin{subequations}
\begin{eqnarray}
&& o^{A} o^{B} \mathcal{D}_{AB}\psi = 3 Y_5 - 3 Y_6 - 2 o^{A} \psi
  \mathcal{D}_{AB}o^{B} + 4 o^{A} o^{B} \psi \iota^{C}
  \mathcal{D}_{BC}o_{A} + 2 o^{A} o^{B} o^{C} \psi
  \mathcal{D}_{BC}\iota_{A}, \label{EqA4a} \\ && o^{A} \iota^{B}
  \mathcal{D}_{AB}\psi = - \tfrac{3}{2} Y_3 + \tfrac{3}{2} Y_4 - \psi
  \iota^{A} \mathcal{D}_{AB}o^{B} - o^{A} \psi
  \mathcal{D}_{AB}\iota^{B} - \tfrac{1}{2} o^{A} o^{B} \psi \iota^{C}
  \mathcal{D}_{AB}\iota_{C} \nonumber \\ && \qquad \qquad \qquad -
  o^{A} \psi \iota^{B} \iota^{C} \mathcal{D}_{AC}o_{B} + \tfrac{1}{2}
  o^{A} \psi \iota^{B} \iota^{C} \mathcal{D}_{BC}o_{A} + o^{A} o^{B}
  \psi \iota^{C}
  \mathcal{D}_{BC}\iota_{A}, \label{EqA4b}\\ &&\iota^{A} \iota^{B}
  \mathcal{D}_{AB}\psi = 3 Y_1 - 3 Y_2 - 2 \psi \iota^{A}
  \mathcal{D}_{AB}\iota^{B} - 4 o^{A} \psi \iota^{B} \iota^{C}
  \mathcal{D}_{AC}\iota_{B} - 2 \psi \iota^{A} \iota^{B} \iota^{C}
  \mathcal{D}_{BC}o_{A}, \label{EqA4c}
\end{eqnarray}
\end{subequations}
while condition \eqref{CS-KID3} is equivalent to
\begin{subequations}
\begin{eqnarray}
&& \psi o^{A} o^{B} o^{C} \mathcal{D}_{BC}o_{A}- Y_{\bm7}=0,\\ && \psi
  o^{A} o^{B}\iota^{C} \mathcal{D}_{AB}o_{C}- \psi o^{A}
  \mathcal{D}_{AB}o^{B} - \tfrac{1}{6} o^{A} o^{B}
  \mathcal{D}_{AB}\psi+\tfrac{1}{2} Y_{\bm5} - \tfrac{1}{6}
  Y_{\bm6}=0,\\
%%%%%%%%%%%%%%%%%%%%%%%%%%%%%%%%%%%%%%%%%%%%%%%%%%%%%%%%%%%%%%%%%%%%%%
% some negative space added to make the equation fit nicely
%%%%%%%%%%%%%%%%%%%%%%%%%%%%%%%%%%%%%%%%%%%%%%%%%%%%%%%%%%%%%%%%%%%%%%%
&& \psi o^{A}\hspace{-0.5mm} o^{B}\hspace{-0.5mm}
  \iota^{C} \hspace{-0.5mm} \mathcal{D}_{AB}\iota_{C} \hspace{-0.5mm}
  -\hspace{-0.5mm} \psi o^{A} \hspace{-0.5mm}\iota^{B}\hspace{-0.5mm}
\iota^{C} \mathcal{D}_{BC}o_{A} \hspace{-0.5mm}-\hspace{-0.5mm} 4
\psi\mathcal{D}_{AB}(\hspace{-0.5mm}
o^{A}\iota^{B} \hspace{-0.5mm})\hspace{-0.5mm}-\hspace{-0.5mm} 2 o^{A}
\hspace{-0.5mm} \iota^{B}\hspace{-0.5mm}
 \mathcal{D}_{AB}\psi\hspace{-0.5mm} -\hspace{-0.5mm}3
Y_{\bm3}\hspace{-0.5mm} +\hspace{-0.5mm} 3 Y_{\bm4} =0,\\
&& \psi \iota^{A} \iota^{B} \iota^{C} \mathcal{D}_{BC}o_{A}+ \psi \iota^{A} 
\mathcal{D}_{AB}\iota^{B}+ \tfrac{1}{6} \iota^{A} \iota^{B} \mathcal{D}_{AB}\psi
+ \tfrac{1}{2} Y_{\bm2}- \tfrac{1}{6} Y_{\bm1} =0,\\
&& \psi \iota^{A} \iota^{B} \iota^{C}\mathcal{D}_{BC}\iota_{A}- Y_{\bm0} =0.
\end{eqnarray}
\end{subequations}
A computation using equations
\eqref{SpacespinorAuxiliaryVectorInDyadOmicronIota} with
\eqref{EqA1a}-\eqref{EqA1d} and \eqref{EqA4a}-\eqref{EqA4c}, shows
that the condition \eqref{CS-KID3} implies
\begin{equation}\label{RestrictionOnCottonCase1}
Y_{\bm0}=Y_{\bm1}=Y_{\bm6}=Y_{\bm7}=0.
\end{equation}
The converse also holds. That is to say, if equation
(\ref{RestrictionOnCottonCase1}) along with
(\ref{CS-KID1})-(\ref{CS-KID2}) are satisfied, and assuming one has
initial data for the alternative CFEs ---so that, in particular,
$\Lambda_{AB}{}^{Q}{}_{Q}=0$--- then condition (\ref{CS-KID3}) holds.


\subsection{Type N Case: $\kappa_{AB}\kappa^{AB}= 0$}

If $\kappa_{AB}\kappa^{AB}= 0$ then one can choose a normalised spin
dyad $\{o_{A},\iota_{B}\}$ such that 
\begin{equation} \label{KillingSpinorAdaptedDyadOmicronOnly}
\kappa_{AB}= o_{A}o_{B}.
\end{equation}
Condition \eqref{CS-KID2} in this adapted dyad implies
\begin{equation}\label{WeylAdaptedDyadOmicronOnly}
\Psi_{ABCD}=\psi o_{(A}o_{B}o_{C}o_{D)}.
\end{equation}
Using equation \eqref{KillingSpinorAdaptedDyadOmicronOnly} one
observes that condition \eqref{CS-KID1} implies
\begin{subequations}
\begin{eqnarray}
&& o^{A} o^{B} o^{C} \mathcal{D}_{AB}o_{C} =
  0, \label{Eq5aSecond}\\ && o^{A} o^{B} \iota^{C}
  \mathcal{D}_{(AB}o_{C)} = 0, \label{Eq5bSecond} \\ && o^{A}
  \iota^{B} \iota^{C} \mathcal{D}_{(AB}o_{C)} = 0, \label{Eq5cSecond}
  \\ && \iota^{A} \iota^{B} \iota^{C} \mathcal{D}_{AB}o_{C} =
  0. \label{Eq5dSecond}
\end{eqnarray}
\end{subequations}
Additionally, using equation
\eqref{KillingSpinorAdaptedDyadOmicronOnly} the spinor $\xi_{AB}$ can
be expressed as
\begin{equation}\label{SpacespinorAuxiliaryVectorInDyadOmicronOnly}
\xi_{AB} = - \tfrac{1}{2} o^{C} \mathcal{D}_{AC}o_{B} - \tfrac{1}{2}
o_{B} \mathcal{D}_{AC}o^{C} - \tfrac{1}{2} o^{C} \mathcal{D}_{BC}o_{A}
- \tfrac{1}{2} o_{A} \mathcal{D}_{BC}o^{C}.
\end{equation}
Using equations \eqref{WeylAdaptedDyadOmicronOnly} and
\eqref{CottonAdaptedDyad} the constraint equations encoded in
$\Lambda_{AB}{}^{Q}{}_{Q}=0$ as given by equation 
\eqref{ZeroQuantityWeylConstraint} imply
\begin{subequations}
\begin{eqnarray}
&& Y_{\bm5}-Y_{\bm6} =0, \label{EqA4aSecond}\\
&& Y_{\bm3}-Y_{\bm4} =0, \label{EqA4bSecond}\\
&& o^{A} o^{B} \mathcal{D}_{AB}\psi = \tfrac{1}{2} Y_{\bm1} - 
 \tfrac{1}{2} Y_{\bm2} - 2 o^{A} \psi \mathcal{D}_{AB}o^{B} - 2 o^{A} o^{B} \iota^{C} 
\psi \mathcal{D}_{AB}o_{C},\label{EqA4cSecond}
\end{eqnarray}
\end{subequations}
Observe that in contrast with the case discussed in Section
  \ref{TypeD}, constraints \eqref{EqA4aSecond}-\eqref{EqA4bSecond}
 immediately imply algebraic dependence of various 
 components of the Cotton spinor.
In this case \eqref{CS-KID3} is equivalent to
\begin{subequations}
\begin{eqnarray}
&& Y_{\bm5}=Y_{\bm7}=0,\\
&& o^{A} o^{B} o^{C} \psi \mathcal{D}_{BC}o_{A}- \tfrac{1}{2} Y_{\bm3}=0,\\
&& o^{A} o^{B} \psi \iota^{C} \mathcal{D}_{(AB}o_{C)}- \tfrac{1}{2} Y_{\bm2}=0.
\end{eqnarray}
\end{subequations}
 A computation using equations
\eqref{SpacespinorAuxiliaryVectorInDyadOmicronOnly},
\eqref{Eq5aSecond}-\eqref{Eq5dSecond} and
\eqref{EqA4bSecond}-\eqref{EqA4cSecond} shows that condition
\eqref{CS-KID3} implies
\begin{equation}\label{RestrictionCottonCase2}
Y_{\bm2}=Y_{\bm3}=Y_{\bm4}=Y_{\bm5}=Y_{\bm6}=Y_{\bm7}=0.
\end{equation}
Again, the converse holds so that condition (\ref{CS-KID3}) may be
replaced with equation (\ref{RestrictionCottonCase2}).  Collecting
together both cases, Proposition \ref{PropositionRestrictionOnCotton}
follows immediately.


\section*{Conclusions}

In this article a \emph{conformal} version of the Killing spinor
initial data equations given in \cite{GarVal08c} are derived. By
conformal it is understood that $(\mathcal{M},\bmg)$ is conformally
related to an Einstein spacetime
$(\tilde{\mathcal{M}},\tilde{\bmg})$. Consequently, we call these
conditions the \emph{conformal Killing spinor initial data equations}.
The existence of a non-trivial solution of this system of equations is
a necessary and sufficient condition for the existence of a Killing
spinor on the development. The conditions are intrinsic to a spacelike
hypersurface $\mathcal{S}\subset\mathcal{M}$. In the
case where the conformal rescaling is trivial, $\Xi = 1$, the
conditions reduce to those given in \cite{BaeVal10b}.  These
conditions contain one differential condition and two algebraic
conditions.  The differential condition corresponds to the so-called
\emph{spatial Killing spinor equation}. The first algebraic condition
corresponds to the restriction of the Buchdahl constraint on the
initial hypersurface and the second imposes restrictions on the Cotton
spinor of the initial data set. Moreover, it was shown that, in a spin
dyad adapted to the Killing spinor, these conditions can be used along
with the conformal Einstein field equations to show that certain
components (at least half of them) of the Cotton spinor $Y_{ABCA'}$
have to vanish on the initial hypersurface $\mathcal{S}$.  Notice that
the conformal approach followed in this article ---i.e., use of the
(alternative) conformal Einstein field equations--- opens the
possibility to allow $\mathcal{S}$ to be determined by $\Xi = 0$ so
that it to corresponds to the conformal boundary $\mathscr{I}$. The
analysis given in this article already shows that in a potential
characterisation of the Kerr-de Sitter spacetime, via the existence of
Killing spinors at the conformal boundary, the Cotton spinor will play
a replant role. This is not unexpected since the conformal boundary of
the Kerr-de Sitter spacetime is conformally flat ---see
\cite{AshBonKes15a, Olz13}.  Therefore, the Cotton tensor associated
with asymptotic initial data corresponding to the Kerr-de Sitter
spacetime vanishes.  Nonetheless, future applications are not
restricted to the analysis of de-Sitter like spacetimes. To see this,
notice that, the most delicate part of the analysis consisted on
finding a system of homogeneous wave equations for $H_{A'ABC}$ and
$S_{AA'BB'}$. This system of wave equations in turn, leads to
conditions \eqref{eq:VanishingOfH}-\eqref{eq:VanishingOfNormalDerivS}
which are irrespective of the causal nature of
$\mathcal{S}$. Consequently, one could investigate the analogous
conditions to those derived in Section \ref{Sec:IntrinsicConditions}
considering a timelike or null hypersurface $\mathcal{S}$ instead. In
the latter case one could consider the conformal boundary of an
asymptotically flat spacetime. In the case of a timelike hypersurface
$\mathcal{S}$, the analogous conditions could be useful for the
 analysis of anti-de Sitter like spacetimes.


\subsection*{Acknowledgements}

We would like to thank Juan A. Valiente Kroon for useful discussion
and advice. E. Gasper\'in acknowledges support from Consejo
Nacional de Ciencia y Tecnolog\'ia (CONACyT Scholarship

\section{Appendix}
\label{Appendix}
\begin{align} 
  P_{C'A'BD} &\equiv - \tfrac{1}{18} Y_{D}{}^{AC}{}_{C'} H_{A'BAC} -
  \tfrac{1}{18} Y_{B}{}^{AC}{}_{C'} H_{A'DAC} + \tfrac{1}{36}
  \Phi_{DAC'B'} S^{AB'}{}_{BA'} \nonumber \\ & \quad + \tfrac{1}{36}
  \Phi_{BAC'B'} S^{AB'}{}_{DA'} - \tfrac{1}{24} \Phi_{BAA'C'}
  S^{AB'}{}_{DB'} - \tfrac{1}{36} \Lambda S_{BA'DC'} + \tfrac{1}{36}
  \Phi_{DAC'B'} S_{B}{}^{B'A}{}_{A'}\nonumber \\ & \quad -
  \tfrac{1}{24} \Phi_{DAA'C'} S_{B}{}^{B'A}{}_{B'} + \tfrac{1}{36}
  \bar{\Psi}_{A'C'B'D'} S_{B}{}^{B'}{}_{D}{}^{D'} - \tfrac{1}{12}
  \Lambda S_{BC'DA'} - \tfrac{1}{36} \Lambda S_{DA'BC'} \nonumber \\ &
  \quad + \tfrac{1}{36} \Phi_{BAC'B'} S_{D}{}^{B'A}{}_{A'} +
  \tfrac{1}{36} \bar{\Psi}_{A'C'B'D'} S_{D}{}^{B'}{}_{B}{}^{D'} -
  \tfrac{1}{12} \Lambda S_{DC'BA'} + \tfrac{1}{36} \Lambda
  S_{B}{}^{B'}{}_{DB'} \bar{\epsilon}_{A'C'} +\nonumber \\ & \quad
  \tfrac{1}{36} \Lambda S_{D}{}^{B'}{}_{BB'} \bar{\epsilon}_{A'C'} +
  \tfrac{1}{9} H_{B'BDC} \nabla_{AC'}\Phi^{AC}{}_{A'}{}^{B'} -
  \tfrac{1}{9} \Psi_{BDCF} \nabla_{AC'}H_{A'}{}^{ACF} \nonumber \\ &
  \quad + \tfrac{1}{3} \Lambda \nabla_{AC'}H_{A'BD}{}^{A} -
  \tfrac{1}{3} H_{A'BDA} \nabla^{A}{}_{C'}\Lambda + \tfrac{1}{9}
  \Phi_{D}{}^{A}{}_{A'}{}^{B'} \nabla_{CC'}H_{B'BA}{}^{C}
  \nonumber\\ & \quad + \tfrac{1}{9} \Phi^{AC}{}_{A'}{}^{B'}
  \nabla_{CC'}H_{B'BDA} + \tfrac{1}{9} \Phi_{B}{}^{A}{}_{A'}{}^{B'}
  \nabla_{CC'}H_{B'DA}{}^{C} - \tfrac{1}{9} H_{B'DAC}
  \nabla^{C}{}_{C'}\Phi_{B}{}^{A}{}_{A'}{}^{B'} \nonumber\\ & \quad -
  \tfrac{1}{9} H_{B'BAC} \nabla^{C}{}_{C'}\Phi_{D}{}^{A}{}_{A'}{}^{B'}
  - \tfrac{1}{9} H_{A'}{}^{ACF} \nabla_{FC'}\Psi_{BDAC} + \tfrac{1}{9}
  \Psi_{DACF} \nabla^{F}{}_{C'}H_{A'B}{}^{AC} \nonumber\\ & \quad +
  \tfrac{1}{9} \Psi_{BACF} \nabla^{F}{}_{C'}H_{A'D}{}^{AC} \nonumber
\end{align}

\begin{align} 
Q_{C'A'BD} &\equiv \tfrac{1}{18} \Phi_{DAC'B'} S^{AB'}{}_{BA'} +
\tfrac{1}{18} \Phi_{BAC'B'} S^{AB'}{}_{DA'} - \tfrac{1}{12}
\Phi_{BAA'C'} S^{AB'}{}_{DB'} - \tfrac{1}{18} \Lambda S_{BA'DC'}
\nonumber \\ & \quad + \tfrac{1}{18} \Phi_{DAC'B'}
S_{B}{}^{B'A}{}_{A'} - \tfrac{1}{12} \Phi_{DAA'C'}
S_{B}{}^{B'A}{}_{B'} + \tfrac{1}{18} \bar{\Psi}_{A'C'B'D'}
S_{B}{}^{B'}{}_{D}{}^{D'} - \tfrac{1}{6} \Lambda S_{BC'DA'} \nonumber
\\ & \quad - \tfrac{1}{18} \Lambda S_{DA'BC'} + \tfrac{1}{18}
\Phi_{BAC'B'} S_{D}{}^{B'A}{}_{A'} + \tfrac{1}{18}
\bar{\Psi}_{A'C'B'D'} S_{D}{}^{B'}{}_{B}{}^{D'} - \tfrac{1}{6} \Lambda
S_{DC'BA'} \nonumber \\ & \quad + \tfrac{1}{12} \Psi_{BDAC}
S^{AB'C}{}_{B'} \bar{\epsilon}_{A'C'} - \tfrac{7}{36} \Lambda
S_{B}{}^{B'}{}_{DB'} \bar{\epsilon}_{A'C'} - \tfrac{1}{36} \Lambda
S_{D}{}^{B'}{}_{BB'} \bar{\epsilon}_{A'C'} \nonumber \\ & \quad -
\tfrac{1}{12} \Lambda S^{AB'}{}_{AB'}
\epsilon_{BD}\bar{\epsilon}_{A'C'} - \tfrac{1}{18} H_{B'DAC}
\nabla_{BC'}\Phi^{AC}{}_{A'}{}^{B'} + \tfrac{1}{18} H_{A'}{}^{ACF}
\nabla_{BC'}\Psi_{DACF} \nonumber \\ & \quad + \tfrac{1}{18}
\Psi_{DACF} \nabla_{BC'}H_{A'}{}^{ACF} - \tfrac{1}{18}
\Phi^{AC}{}_{A'}{}^{B'} \nabla_{BC'}H_{B'DAC} -\tfrac{1}{18} H_{B'BAC}
\nabla_{DC'}\Phi^{AC}{}_{A'}{}^{B'} \nonumber \\ & \quad +
\tfrac{1}{18} H_{A'}{}^{ACF} \nabla_{DC'}\Psi_{BACF} + \tfrac{1}{18}
\Psi_{BACF} \nabla_{DC'}H_{A'}{}^{ACF} - \tfrac{1}{18}
\Phi^{AC}{}_{A'}{}^{B'} \nabla_{DC'}H_{B'BAC} \nonumber
\end{align}

\begin{align}
 U_{A'BC'D} & \equiv - \tfrac{1}{24} Y_{D}{}^{AC}{}_{C'} H_{A'BAC} -
 \tfrac{1}{24} Y_{B}{}^{AC}{}_{C'} H_{A'DAC} - \tfrac{1}{24}
 Y_{D}{}^{AC}{}_{A'} H_{C'BAC} - \tfrac{1}{24} Y_{B}{}^{AC}{}_{A'}
 H_{C'DAC}  \nonumber \\ & \quad - \tfrac{1}{12} \Psi_{BDCF}
 \nabla_{AC'}H_{A'}{}^{ACF} - \tfrac{1}{12} H_{C'}{}^{ACF}
 \nabla_{FA'}\Psi_{BDAC} - \tfrac{1}{12} H_{A'}{}^{ACF}
 \nabla_{FC'}\Psi_{BDAC} \nonumber \\ & \quad + \tfrac{1}{12}
 \Psi_{DACF} \nabla^{F}{}_{C'}H_{A'B}{}^{AC} + \tfrac{1}{12}
 \Psi_{BACF} \nabla^{F}{}_{C'}H_{A'D}{}^{AC} - \tfrac{1}{4}
 \Psi_{BDAC} S^{(A}{}_{(A'}{}^{C)}{}_{C')}\nonumber
\end{align}
% 
\bibliographystyle{/home/gasperin/Academic/References/reporthack}
\bibliography{/home/gasperin/Academic/References/GRbibJune2021a}
%

%% \begin{thebibliography}{10}

%% \bibitem{AshBonKes15a}
%% A.~{Ashtekar}, B.~{Bonga}, \& A.~{Kesavan},
%% \newblock {\em {Asymptotics with a positive cosmological constant: I. Basic
%%   framework}},
%% \newblock Classical and Quantum Gravity {\bf 32}(2), 025004 (Jan. 2015).

%% \bibitem{BaeVal10a}
%% T.~B\"{a}ckdahl \& J.~A. {Valiente Kroon},
%% \newblock {\em Geometric invariant measuring the deviation from Kerr data},
%% \newblock Phys. Rev. Lett. {\bf 104}, 231102 (2010).

%% \bibitem{BaeVal10b}
%% T.~B\"{a}ckdahl \& J.~A. {Valiente Kroon},
%% \newblock {\em On the construction of a geometric invariant measuring the
%%   deviation from Kerr data},
%% \newblock Ann. Henri Poincar\'e {\bf 11}, 1225 (2010).

%% \bibitem{BaeVal11b}
%% T.~B\"{a}ckdahl \& J.~A. {Valiente Kroon},
%% \newblock {\em The "non-Kerrness" of domains of outer communication of black
%%   holes and exteriors of stars},
%% \newblock Proc. Roy. Soc. Lond. A {\bf 467}, 1701 (2011).

%% \bibitem{BaeVal12}
%% T.~B\"{a}ckdahl \& J.~A. {Valiente Kroon},
%% \newblock {\em Constructing ``non-Kerrness'' on compact domains},
%% \newblock J. Math. Phys. {\bf 53}, 04503 (2012).

%% \bibitem{BeiChr97b}
%% R.~Beig \& P.~T. Chru\'{s}ciel,
%% \newblock {\em Killing initial data},
%% \newblock Class. Quantum Grav. {\bf 14}, A83 (1997).

%% \bibitem{ColVal16}
%% M.~J. {Cole} \& J.~A. {Valiente Kroon},
%% \newblock {\em {A geometric invariant characterising initial data for the
%%   Kerr-Newman spacetime}},
%% \newblock ArXiv e-prints  (Sept. 2016).

%% \bibitem{Fri81b}
%% H.~Friedrich,
%% \newblock {\em The asymptotic characteristic initial value problem for
%%   {Einstein}'s vacuum field equations as an initial value problem for a
%%   first-order quasilinear symmetric hyperbolic system},
%% \newblock Proc. Roy. Soc. Lond. A {\bf 378}, 401 (1981).

%% \bibitem{Fri81a}
%% H.~Friedrich,
%% \newblock {\em On the regular and the asymptotic characteristic initial value
%%   problem for {Einstein}'s vacuum field equations},
%% \newblock Proc. Roy. Soc. Lond. A {\bf 375}, 169 (1981).

%% \bibitem{Fri82}
%% H.~Friedrich,
%% \newblock {\em On the existence of analytic null asymptotically flat solutions
%%   of {Einstein}'s vacuum field equations},
%% \newblock Proc. Roy. Soc. Lond. A {\bf 381}, 361 (1982).

%% \bibitem{Fri83}
%% H.~Friedrich,
%% \newblock {\em Cauchy problems for the conformal vacuum field equations in
%%   General Relativity},
%% \newblock Comm. Math. Phys. {\bf 91}, 445 (1983).

%% \bibitem{Fri86c}
%% H.~Friedrich,
%% \newblock {\em Existence and structure of past asymptotically simple solutions
%%   of Einstein's field equations with positive cosmological constant},
%% \newblock J. Geom. Phys. {\bf 3}, 101 (1986).

%% \bibitem{Fri86b}
%% H.~Friedrich,
%% \newblock {\em On the existence of n-geodesically complete or future complete
%%   solutions of {E}instein's field equations with smooth asymptotic structure},
%% \newblock Comm. Math. Phys. {\bf 107}, 587 (1986).

%% \bibitem{Fri91}
%% H.~Friedrich,
%% \newblock {\em On the global existence and the asymptotic behaviour of
%%   solutions to the Einstein-Maxwell-Yang-Mills equations},
%% \newblock J. Diff. Geom. {\bf 34}, 275 (1991).

%% \bibitem{GasVal17}
%% E.~Gasper{\'i}n \& J.~A. Valiente~Kroon,
%% \newblock {\em Perturbations of the Asymptotic Region of the Schwarzschild--de
%%   Sitter Spacetime},
%% \newblock Annales Henri Poincar{\'e} , 1--73 (2017).

%% \bibitem{GarVal08c}
%% A.~G.-P. {G{\'o}mez-Lobo} \& J.~A. {Valiente Kroon},
%% \newblock {\em {Killing spinor initial data sets}},
%% \newblock Journal of Geometry and Physics {\bf 58}, 1186--1202 (Sept. 2008).

%% \bibitem{Mar99}
%% M.~Mars,
%% \newblock {\em A spacetime characterization of the Kerr metric},
%% \newblock Class. Quantum Grav. {\bf 16}, 2507 (1999).

%% \bibitem{Mar00}
%% M.~Mars,
%% \newblock {\em Uniqueness properties of the Kerr metric},
%% \newblock Class. Quantum Grav. {\bf 17}, 3353 (2000).

%% \bibitem{MarPaeSen16}
%% M.~{Mars}, T.-T. {Paetz}, J.~M.~M. {Senovilla}, \& W.~{Simon},
%% \newblock {\em {Characterization of (asymptotically) Kerr-de Sitter-like
%%   spacetimes at null infinity}},
%% \newblock Classical and Quantum Gravity {\bf 33}(15), 155001 (Aug. 2016).

%% \bibitem{McLBer93}
%% R.~G. McLenaghan \& N.~V. den Bergh,
%% \newblock {\em Spacetimes admitting Killing 2-spinors},
%% \newblock Classical and Quantum Gravity {\bf 10}(10), 2179 (1993).

%% \bibitem{Olz13}
%% C.~\"{O}lz,
%% \newblock {\em The global structure of Kerr-de Sitter metrics},
%% \newblock Master thesis, University of Vienna, 2013.

%% \bibitem{Pae13}
%% T.-T. Paetz,
%% \newblock {\em Conformally covariant systems of wave equations and their
%%   equivalence to Einstein's field equations},
%% \newblock In {\tt arXiv:1306.6204}, 2013.

%% \bibitem{Pae14}
%% T.-T. {Paetz},
%% \newblock {\em {Killing Initial Data on spacelike conformal boundaries}},
%% \newblock ArXiv e-prints  (Mar. 2014).

%% \bibitem{Pae14a}
%% T.-T. Paetz,
%% \newblock {\em KIDs prefer special cones},
%% \newblock Classical and Quantum Gravity {\bf 31}(8), 085007 (2014).

%% \bibitem{PenRin84}
%% R.~Penrose \& W.~Rindler,
%% \newblock {\em Spinors and space-time. {V}olume 1. {T}wo-spinor calculus and
%%   relativistic fields},
%% \newblock Cambridge University Press, 1984.

%% \bibitem{PenRin86}
%% R.~Penrose \& W.~Rindler,
%% \newblock {\em Spinors and space-time. {V}olume 2. {S}pinor and twistor methods
%%   in space-time geometry},
%% \newblock Cambridge University Press, 1986.

%% \bibitem{Rob75b}
%% D.~C. Robinson,
%% \newblock {\em Uniqueness of the Kerr black hole},
%% \newblock Phys. Rev. Lett. {\bf 34}, 905 (1975).

%% \bibitem{Sim84}
%% W.~Simon,
%% \newblock {\em Characterizations of the Kerr metric},
%% \newblock Gen. Rel. Grav. {\bf 16}, 465 (1984).

%% \bibitem{Som80}
%% P.~Sommers,
%% \newblock {\em Space spinors},
%% \newblock J. Math. Phys. {\bf 21}, 2567 (1980).

%% \bibitem{Ste91}
%% J.~Stewart,
%% \newblock {\em Advanced general relativity},
%% \newblock Cambridge University Press, 1991.

%% \bibitem{Tay96c}
%% M.~E. Taylor,
%% \newblock {\em Partial differential equations {III}: nonlinear equations},
%% \newblock Springer Verlag, 1996.

%% \bibitem{CFEbook}
%% J.~A. {Valiente Kroon},
%% \newblock {\em Conformal methods in General Relativity},
%% \newblock Cambridge University Press ---in preparation.

%% \bibitem{Wei90a}
%% G.~Weinstein,
%% \newblock {\em On rotating black holes in equilibrium in general relativity},
%% \newblock Communications on Pure and Applied Mathematics {\bf 43}(7), 903--948
%%   (1990).

%% \end{thebibliography}






\end{document}
